  Einstein's publication of the \textit{general theory of relativity} in 1915, and the discovery of a wave-like solution to the field-equations of that theory sparked a century-long quest to detect \textit{gravitational waves}.
  These illusive metric disturbances were predicted to ripple-away from some of the most energetic events in the universe, such as supernovae and colliding black holes.
  
  The quest was completed in September 2015, with the \gls{ligo} observation of a gravitational wave produced by a pair of coalescing black holes, but work to continue detecting and interpretting the signals which are detected by \gls{ligo} and its brethren is by no means complete.
  The age of gravitational wave observation has arrived, and with it the difficulties of interpretting myriad signals, differentiating them from noise, and analysing them in order to gain insight into the astrophysical systems which produced them.

  In this work I provide an overview of the history of the field of gravitational wave science: both in terms of the theoretical principles which frame it, and the attempts to build instruments which could measure them.
  I then provide a discussion of the morphologies of the signals which are searched for in current detectors' data, and the astrophysical systems which may produce such signals.
  It is of great importance that the sensitivity of both detectors and the signal analysis techniques which are used is well-understood.
  A substantial part of the novel work presented in this document discusses the development of a technique for assessing this sensitivity, through a software package called Minke.

  Knowing the sensitivity of a detector to signals from an astrophysical source allows robust limits to be placed on the rate at which these events occur.
  These rates can then be used to infer properties of astrophysical systems; this document contains a discussion of a technique which was developed by the author to allow the determination of the geometry of beamed emission from short gamma ray bursts which result from neutron star coalescences.
  This method finds that at its design sensitivity we expect the advanced LIGO detector to be able to place limits on the opening angle, $\theta$ of the beam of $\theta \in (8.10^\circ,14.95^\circ)$ under the assumption that all neutron star coalesences produce jets, and that gamma ray bursts occur at an illustrative rate of $\grbrate=10$\,Gpc$^{-3}$\,yr$^{-1}$}.

  The most efficient methods for extracting signals from noisy data, such as that produced by gravitational wave detectors, and then analysing these signals, requires robust prior knowledge of the signals' morphologies.
  The development of a new model for producing gravitational waveforms for coalescing binary black hole systems is discussed in detail in this work.
  The method which is used, Gaussian process regression, is introduced, with an overview of different methods for implementing models which use the method.
  The model, named Heron, is itself presented, and comparisons between the waveforms produced by Heron and other models which are currently used in analysis are made.
  Comparisons between the Heron model and highly accurate numerical relativity waveforms are also shown.
