\documentclass[openleft]{kentigern}

\usepackage{lipsum}
%\usepackage[toc]{glossaries}
\usepackage[acronym,toc]{glossaries}
\usepackage{siunitx}

\makeglossaries

\usepackage{type1cm}
\usepackage{lettrine}
\usepackage{physicsplus}
\usepackage{caption}

\usepackage{rotating}
\usepackage{pgfgantt}

\usepackage{setspace}
    \linespread{1.25}

\title{Year 1 Report}
\author{Daniel Williams}

%% This file is intended to contain all of the macros defining the
%% various mathematical and physical quantities used in this document in
%% order to maintain notational consistency throughout the document and
%% its glossary.

% Margin notes

\newcommand{\marginnote}[1]{
        \refstepcounter{footnote}
\footnotemark\marginpar{\footnotemark}\footnotetext{#1}}


\NewEnviron{sidefigure}[2]{
  \sidebar{
    \BODY
    \captionof{figure}{#1 \label{#2}}
  }
}

\providecommand{\msolar}{\mathrm{M{_\odot}}}
\providecommand{\mat}[1]{\mathsf{#1}}
\providecommand{\ex}{\mathbb{E}\,}

% New Operators
\DeclareMathOperator{\vary}{var}
\DeclareMathOperator{\cov}{cov}
\providecommand{\rmse}{\mathrm{RMSE}}

% Calculus operators
\providecommand{\dd}{\,\mathrm{d}}

% Software Packages
\providecommand{\imrphenomp}{\texttt{IMRPhenomP}}
\providecommand{\lalsim}{\texttt{LALSimulation}}

% New Units
\providecommand{\solMass}{\ensuremath{\mathrm{M}_{\odot}}}
%\DeclareSIUnit\parsec{pc}

% Galactic astronomy
\providecommand{\numberGalaxies}{N_\mathrm{G}}

% Gravitational wave detectors
\providecommand{\horizonDistance}{\ensuremath\mathcal{D}_{\mathrm{hor}}}

\providecommand{\GP}{\gls{gp}}

% Pipelines
\providecommand{\olib}{\texttt{oLIB}}
\providecommand{\cwb}{\texttt{cWB}}
\providecommand{\bayeswave}{\texttt{Bayeswave}}
\providecommand{\minke}{\texttt{Minke}}

% Software
\providecommand{\lalsuite}{\texttt{LALSuite}}
\providecommand{\lalsimulation}{\texttt{LALSimulation}}
\providecommand{\imrp}{\texttt{IMRPhenom\,v2}}
\providecommand{\seobnr}{\texttt{SEOBNR}}
\providecommand{\heron}{\texttt{heron}}
% latin
\providecommand{\map}{maximum \emph{a posteriori}}

% Gaussian processes
\providecommand{\set}[1]{\mathcal{#1}}
\providecommand{\gp}{\mathcal{G\!P}}
\providecommand{\GP}{Gaussian Process\renewcommand{\GP}{GP}}
\providecommand{\NR}{Numerical Relativity\renewcommand{\NR}{NR}}
\providecommand{\PE}{parameter estimation\renewcommand{\PE}{PE}}
\providecommand{\GW}{gravitational wave\renewcommand{\GW}{GW}}
\providecommand{\EI}{\mathbb{E} \mathrm{I}}

\providecommand{\trainingpoints}{\mathcal{X}}
\providecommand{\trainingobservations}{\mathcal{Y}}
\providecommand{\trainingdata}{(\trainingpoints, \trainingobservations)}

\providecommand{\kernel}[1]{\mathsf{#1}}
\providecommand{\SE}{\kernel{SE}}
\providecommand{\Con}{\kernel{C}}
\providecommand{\Lin}{\kernel{Lin}}
\providecommand{\Per}{\kernel{Per}}
\providecommand{\RQ}{\kernel{RQ}}
\providecommand{\Mat}{\kernel{M52}}

\providecommand{\numbertrainingpoints}{$12,325$}
\providecommand{\numbertrainingwaveforms}{no. Waveforms}



\theoremstyle{definition}
\newtheorem{definition}{Definition}[section]

\loadglsentries[main]{glossary/glossary.tex}

\begin{document}
\maketitle

\newpage
%

%\tableofcontents
\newpage
%\vfill
\lettrine{I}{'ve had} the pleasure of being a member of the Institute for
Gravitational Research at the University of Glasgow for a little over
seven months. It would be difficult to deny that this has been an
unexpectedly eventful period in the life of the group, and it's been
great fun to experience the excitement of the first detection of a
gravitational wave as an insider (although an insider only by the skin
of my teeth), and to see the department's work on the LISA Pathfinder
mission come to fruition just a few months later.  

The work outlined in this report comprises a number of projects which
I have spent these seven months working on, however the first
observing run of Advanced LIGO proved more disruptive to my schedule
than was, perhaps, initially envisaged.
\vfill
\newpage

%%% Local Variables: 
%%% mode: latex
%%% TeX-master: "../document"
%%% End: 

% \part{Outline \& Review of Gravitational Wave Literature}
% \label{part:introduction}


\part{Introduction}
\label{part:intro}


 \chapter{Gravitational Waves}
 \label{cha:grav-waves}

%\section{Gravitational waves}
%\label{sec:gravwaves}
\chapterprecis{\Glspl{gravitational wave} were perhaps the last of the
  predictions of Einstein's General Theory of Relativity to be
  observed; their detection was the catalyst for the beginning of a
  new era of astrophysics}

\epigraph{I guess we need to do the detection checklist...}{\textbf{Sergey Klimencko}, \emph{Internal LSC communication}, 14 September 2015}

\lettrine[lines=3]{T}{he 14 September 2016} will likely be remembered as one of the most
significant in the history of astronomy and of astrophysics. Early in
the morning of this autumn day, shortly after 9am UT, a gravitational
wave passed through the Earth, and on its way produced a sufficiently
large movement in the mirrors and test masses of the detectors of the
Laser Interferometer Gravitational-wave Observatory (\gls{LIGO}), as to be
detected.

Over five months of data analysis, detector characterisation, and
detection verification were conducted by a global team of scientists,
in the LIGO and VIRGO Scientific Collaboration (LVC). This process
resulted in a slew of journal papers being written, vast quantities of
data produced, and an enormous public outreach effort to be
launched. Eventually, the collaboration found itself in a position to
make the announcement of the first direct detection of gravitational
waves, GW150914, on 11 February 2016.

GW150914 is not the first evidence for gravitational waves, with the
Hulse-Taylor pulsar\cite{1975ApJ...195L..51H,2005ASPC..328...25W}
having provided compelling indirect evidence which lead to its
discoverers receiving the 1993 Nobel Prize.

%%% Local Variables: 
%%% mode: latex
%%% TeX-master: "../../document"
%%% End: 



% \section{Gravitational waves and general relativity}
% \label{sec:grav-waves-gener}
% Gravitational waves are one of the predictions of Einstein's 1915
Theory of General Relativity.


The Einstein field equations are non-linear, and analytical solutions
aren't possible. In order to perform calculations in general
relativity it is possible to turn to  \emph{post-Newtonian} expansion. 

The lowest-order expansion from which radiation is produced is the
quadrupole order, which depends upon the density and velocity field of
the system. The quadrupole tensor is defined
\begin{equation}
  \label{eq:1}
  Q_{jk} = \int \rho x_k x_k \dd[3]{x}
\end{equation}
and the gravitational wave is described by the three-tensor
\begin{equation}
  \label{eq:2}
  h_{jk} = \frac{2}{r} \dv[2]{Q_{jk}}{t}
\end{equation}


%%% Local Variables: 
%%% mode: latex
%%% TeX-master: "../../document"
%%% End: 


\section{GW150914: The first detection}
\label{sec:gw150914:-first-dete}
The first detection of gravitational waves was announced by the LIGO
Scientific Collaboration and the VIRGO Collaboration on 11 February
2016\cite{2016PhRvL.116f1102A}.


%%% Local Variables: 
%%% mode: latex
%%% TeX-master: "../../document"
%%% End: 



 \chapter{Detectors}
 \label{cha:detectors}

 \chapterprecis{Gravitational wave detectors are one of the great
   achievements of Twentieth and early Twenty-First Century
   science. They are the most sensitive measuring devices ever
   constructed, but they face numerous technical challenges.}

%\section{Gravitational wave detectors}
%\label{sec:detectors}

The earliest attempts to develop a gravitational wave detector were
made in the 1960s with the experiments of Jospeh Weber (1919--2000) at
the University of Maryland, from which he claimed the first detection
in 1969\cite{1969PhRvL..22.1320W,1970PhRvL..24..276W} of signals
originating in the galactic centre\cite{1970PhRvL..25..180W}. Numerous
attempts to confirm his findings were unsuccessful, including searches
in Drever's group at the University of Glasgow
\cite{1973Natur.246..340D} in the UK, at Bell Labs
\cite{1973PhRvL..31..173L,1973PhRvL..31..176G,1974PhRvL..33..794L} in
the USA, at Munich\cite{1975NCimL..12..111B,1975NCimL..12..111B} in
Germany, at Moscow\cite{1973PhLA...45..271B} in Russia, and at
Tokyo\cite{1975PhRvL..35..890H} in Japan. While Weber's original
detections were soundly refuted by the community there is little doubt
that the announcement led to a flurry of activity in the field. This
lead to the development of modern cryogenic resonant bars, such as
ALTAIR\cite{1992NCimC..15..943B}, ALLEGRO\cite{2000IJMPD...9..229M},
NAUTILUS\cite{1997APh.....7..231A}, and
EXPLORER\cite{1993PhRvD..47..362A}; and laser interferometers.

Laser interferometers, of which advanced LIGO is an implementation,
were the result of a quest for both higher sensitivities and
bandwidth. The possibility of using a Michelson interferometer to
measure the distance between test masses in order to detect
gravitational radiation originated in Moscow\cite{1963JETP...16..433G}
in 1963.  This approach was followed early-on by Scottish and German
groups as a means of improving on resonant bar sensitivities, with a
3-meter and later a 30-meter prototype detector constructed at
Garching in the late
1970s\cite{1979JPhE...12.1043B,1988PhRvD..38..423S} which used optical
delay lines, and a 1-meter prototype, and later a 10-meter instrument
were built at Glasgow in the early
1980s\cite{1979RSPSA.368...11D,1995RScI...66.4447R}, which used
Fabry-Perot cavities . The Glasgow detector was the spiritual
predecessor to the CalTech 40-meter
prototype\cite{1996PhLA..218..157A}. The increasing maturity of
technology developed by these prototypes lead to the construction of
the first generation of long-baseline detectors, starting with TAMA in
Tokyo\cite{1996JKASS..29..279K}, the joint UK-German GEO600
detector\cite{1997CQGra..14.1471L} near Hannover, and the two
kilometre-scale detectors, the joint CalTech-MIT detectors
LIGO\cite{1992Sci...256..325A}, located at two sites in the USA, and
the joint Italian-French detector VIRGO\cite{1990NIMPA.289..518B},
near Cascina. These detectors were operated during the 2000s, and
while none of them made a detection of gravitational waves, they
provided valuable astrophysical results by placing astrophysical
limits on the strength of the stochastic gravitational wave background
\cite{2014PhRvL.113w1101A}, production of gravitational waves by
pulsars\cite{2014ApJ...785..119A} and gamma ray
bursts\cite{2012ApJ...760...12A}, and the rate of compact binary
coalescence in the local
universe\cite{2012PhRvD..85h2002A,2013PhRvD..87b2002A}.

The initial-generation of detectors were upgraded during the first
half of the 2010s, leading to Advanced LIGO\cite{2015CQGra..32g4001L}
which resumed observations in September 2015, and the imminent start
of observations from the Advanced VIRGO
detector\cite{2015CQGra..32b4001A}, with the prospect of a joint run
occuring during the second half of 2016. The GEO detector was the
first of the initial detectors to be fully upgraded becoming GEO-HF
\cite{2006CQGra..23S.207W}, with improved sensitivity at high
frequencies. Japanese efforts have focussed on the development of
KAGRA (formerly LCGT), a cryogenic interferometer located deep
underground in the Kamioka mine\cite{1999IJMPD...8..557K}, although
the project has suffered from a number of set-backs. The construction
of a third LIGO detector interferometer in India using the mothballed
second detector from the Washington site has now moved into its
initial stages, with the prospect of this detector joining the network
around the end of the decade.

The second-generation detectors, specifically the two Advanced LIGO
detectors were responsible for the first discovery of gravitational
waves\cite{2016PhRvL.116m1103A}, and have successfully demonstrated
the ability of interferometry to observe the gravitational
universe. This said, future improvements in sensitivity are highly
desirable, but are likely to be even more technically challenging than
the transition from resonant bars to laser interferometers. In order
to improve the bandwidth of detectors a location free of
\emph{Newtonian noise} must be found, which ultimately mandates the
placement of an interferometer in space. There have been a number of
proposals for a space-based interferometer, with
eLISA\cite{2013GWN.....6....4A} likely to be the first to launch in
the 2020s, following the completion of a pathfinder mission during
2016\cite{2015JPhCS.610a2005A}. NEED TO UPDATE THIS. The eLISA detector will be sensitive
in the milli-hertz region of the gravitational wave spectrum, and will
be capable of observing binary inspirals at a much earlier stage in
their evolution than the advanced ground-based detectors, as well as
the galactic population of low-mass binaries, such as binary white
dwarfs. A Japanese proposal, DECIGO\cite{2011CQGra..28i4011K}, would
observe in the decihertz regime using a complex arrangement of six
spacecraft in a star-of-David confugration. There are also plans for
more sensitive detectors on the ground. The Einstein telescope is a
European proposal for an underground kilometre-scale detector in a
triangular configuration, using a ``xylophone'' configuration to
improve broadband sensitivity compared to the second-generation of
detectors; its scientific aims include providing more sensitive tests
of general relativity than are possible with the advanced
detectors\cite{2012CQGra..29l4013S}. There are also proposals for
upgrades of the advanced detectors to use squeezed light to reduce
quantum noise\cite{2015PhRvD..91f2005M}, the use of
speedmeters\cite{2014MUPB...69..519V,2002gr.qc....11088K}, or atom
interferometry\cite{2013PhRvL.110q1102G,2016PhRvD..93b1101C,2008PhRvD..78l2002D}.

At the very low-frequency limit of the gravitational wave spectrum the
bulk of detection efforts rotate around pulsar timing arrays, which
promise the detection of gravitational waves by precision measurements
of pulse arrival times from a number of pulsars distributed across the
sky. By observing correlated delays\cite{1983ApJ...265L..39H} in
arrival times the presence of a very long wavelength gravitational
wave can be inferred. There are a number of collaborations actively
producing pulsar observations with the aim of detecting gravitational
waves: the European Pulsar Timing Array
(EPTA)\cite{2013CQGra..30v4009K}, NANOGrav\cite{2009arXiv0909.1058J},
the Parkes Pulsar Timing Array (PPTA)\cite{2013PASA...30...17M}, and
the International Pulsar Timing Array (IPTA)
collaboration\cite{2013CQGra..30v4010M}.

%%% Local Variables: 
%%% mode: latex
%%% TeX-master: "../../document"
%%% End: 


% \section{Detector architectures}
% \label{sec:detect-arch}

% \section{Resonant bar detectors}
% \label{sec:reson-bar-detect}

 \section{Ground-based interferometers}
 \label{sec:ground-based-interf}
 The cutting-edge of current ground-based interferometers are the twin
Advanced LIGO detectors \cite{2015CQGra..32g4001L} located in Hanford,
WA, USA, and Livingston, LA, USA. These interferometers are Michelson
interferometers with a large number of additional components, which
allow detection of differential changes in their arm lengths (strains)
on the order of $10^{-22}$.

% \begin{figure}
% \begin{adjustwidth*}{-5.5in}{-2in}
%   \centering
%   %\begin{tikzpicture}[
	scale=0.8,
	housing/.style={top color=black!10, bottom color=black!40, draw=black!50, line width=1pt},
]

\begin{scope}[]
%%% CORNER STATION

\draw [housing] (0,0) rectangle (5,5) node at (1.5,-.4) {Corner Station};

\draw [housing] (1.5,5) rectangle (3.5,15);
\draw [housing] (1.5,17) rectangle (3.5,19);
\end{scope}

\begin{scope}[]
%%% X ARM
\draw [housing] (5,1.5) rectangle (15,3.5);
\draw [housing] (17,1.5) rectangle (19,3.5);
\end{scope}

\end{tikzpicture}
%   \caption{An interferometer.}
%   \label{fig:interferometer}
% \end{adjustwidth*}
% \end{figure}

\subsection{Detecting gravitational waves with light}
\label{sec:interferometricdetection}

Gravitational-wave detectors which use beams of light, such as
interferometers and pulsar timing arrays rely on measuring the the
travel time of a beam of electromagnetic radiation between two points,
and the effect that a gravitational wave has on this time. A full
treatment of this is given in \cite{2009LRR....12....2S}, but in
summary, if a gravitational wave is not present within a detector, the
travel time of a beam will be constant. If a gravitational waev is
introduced, which has a polarisation component $h_+(t)$ in the plane
of the beam, the change in the arrival time of the beam will be
\begin{equation}
  \label{eq:arrival-times-gw}
  \dv{t_f}{t} = 1 + \half (1 + \cos(\theta)) \qty{ 
    h_+\qty( t + [1- \cos(\theta) ] L) - h_+(t) 
  }
\end{equation}
where $\theta$ is the angle separating the detector beam and the
gravitational wave plane, and $L$ is the proper distance separating
the clocks when no gravitational wave is present.

By arranging the detector to reflect the beam back to the originating
clock, it is possible to measure the round-triop time using only one
clock. In this arrangement we must account for the gravitational wave
having a different strength one the return trip, and so equation
(\ref{eq:arrival-times-gw}) becomes 
\begin{align}
  \label{eq:three-term}
  \dv{t~{round}}{t} = 1 + \half \Big(  (& 1-\cos(\theta) )h_+ (t+2L) - (1+\cos(\theta))h_+(t) \nonumber \\ & + 2 \cos(\theta) h_+ [t+L(1 - \cos(\theta))] \Big)
\end{align}
the \emph{three-term} relation.

\subsection{Operation of a Michelson interferometer}
\label{sec:Michelson}
%
\sidebar{
\begin{center}
  \begin{tikzpicture}
    \draw [thick, red] (0,0.25) -- (3,0.25);
    \draw [thick, red] (1.1, 0.25) -- (1.1, 2.15);
    \draw [thick, red, dashed] (1.1, 0.25) -- (1.1, -0.5);
    \fill (0,0) rectangle (0.5, 0.5);
    \draw [ultra thick] (0.95, 0.1) -- +(45:.4);
    \draw [ultra thick] (3, 0) rectangle (3.2, .5);
    \draw [ultra thick] (0.8, 2.15) rectangle (1.4, 2.35);
  \end{tikzpicture}
\end{center}
\captionof{figure}{A simple Michelson interferometer, composed of a
    light source (black box), a beam splitter (heavy black line), and
    two end mirrors (white boxes). \label{fig:michelson}}
} 
%
A Michelson interferometer is an optical device which is capable of
measuring the difference in length between two optical paths to
sub-wavelength precision. A Michelson interferometer can be
constructed using a beamsplitter and two mirrors, in the configuration
presented in figure \ref{fig:michelson}. The input beam is split along
the $x$ and $y$ directions, and reflected back to the
beam-splitter. At the beam-splitter the two beams will interfere: in
the standard Michelson setup this will result in constructive
interference if the arms have identical lengths, and a beam will be
produced at the output (the dashed red line). If the arms' relative
lengths change a pattern of interference fringes will be visible at
the output of the interferometer.

\subsection{Power Recycling}
\label{sec:power-recycling}
%
\sidebar{
\begin{center}
  \begin{tikzpicture}
    \draw [ultra thick, red] (0,0.25) -- (3,0.25);
    \draw [ultra thick, red] (1.1, 0.25) -- (1.1, 2.15);
    \draw [thick, red] (-1,0.25) -- (0, 0.25);
    \draw [thick, red, dashed] (1.1, 0.25) -- (1.1, -0.5);
    \fill (-1,0) rectangle (-0.5, 0.5);
    \draw [ultra thick] (0.95, 0.1) -- +(45:.4);
    \draw [ultra thick] (3, 0) rectangle (3.2, .5);
    \draw [ultra thick] (0.8, 2.15) rectangle (1.4, 2.35);
    \draw [ultra thick] (-0.25, 0) rectangle (-0, 0.5);
  \end{tikzpicture}
\end{center}
\captionof{figure}{A Michelson interferometer with power recycling,
  using a comparable setup to figure \ref{fig:michelson}, but with a
  mirror added between the laser and the beam splitter. \label{fig:power-recycle}}
} 
%
The optimal signal-to-noise ratio can be achieved from an
interferometer when the arm lengths are configured so that when no
gravitational wave is present in the interferometer the interferometer
beams interfere destructively \cite{1978JPhE...11..710E}. If the
mirrors are low loss the light will then be reflected back towards the
laser, and by placing a mirror between the laser and the beam splitter
a resonant cavity can be formed (see figure \ref{fig:power-recycle}),
allowing the power in the interferometer to build up. This allows a
less powerful laser to be used as the input for the interferometer,
with a \SI{10}{\watt} laser capable of providing several kilowatts of
power inside the interferometer \cite{2011LRR....14....5P}.

\subsection{Signal Recycling}
\label{sec:signal-recycling}
%
\sidebar{
\begin{center}
  \begin{tikzpicture}
    \draw [thick, red] (0,0.25) -- (3,0.25);
    \draw [thick, red] (1.1, 0.25) -- (1.1, 2.15);
    \draw [thick, red] (-1,0.25) -- (0, 0.25);
    \draw [thick, red, dashed] (1.1, 0.25) -- (1.1, -1.0);
    \fill (-1,0) rectangle (-0.5, 0.5);
    \draw [ultra thick] (0.95, 0.1) -- +(45:.4);
    \draw [ultra thick] (3, 0) rectangle (3.2, .5);
    \draw [ultra thick] (0.8, 2.15) rectangle (1.4, 2.35);
    \draw [ultra thick] (0.9, -0.5) rectangle (1.3, -0.7);
  \end{tikzpicture}
\end{center}
\captionof{figure}{A Michelson interferometer with signal recycling,
  using a comparable setup to figure \ref{fig:michelson}, but with a
  mirror added between the output and the beam splitter. \label{fig:signal-recycle}}
} 
%
Signal recycling can be used to tune the bandwidth of an
interferometer, and to increase its sensitivity by re-injecting the
interferometer's output signal to the interferometer, achieving
resonance, which increases the signal-to-noise ratio of the
signal. This is possible thanks to the sidebands on the beam which are
produced by the gravitational wave not interfering destructively.

To perform signal recycling a mirror is added between the
beam-splitter and the readout port of the interferometer, with this
configuration illustrated in figure \ref{fig:signal-recycle}.

\subsection{Fabry-Perot Cavities}
\label{sec:fabry-perot-cavities}
%
\sidebar{
\vspace{-1cm}
\begin{center}
  \begin{tikzpicture}
    \draw [thick, red] (0,0.25) -- (3,0.25);
    \draw [thick, red] (1.1, 0.25) -- (1.1, 2.15);
    \draw [thick, red] (-1,0.25) -- (0, 0.25);
    \draw [thick, red, dashed] (1.1, 0.25) -- (1.1, -1.0);
    \fill (-1,0) rectangle (-0.5, 0.5);
    \draw [ultra thick] (0.95, 0.1) -- +(45:.4);
    \draw [ultra thick] (3, 0) rectangle (3.2, .5);
    \draw [ultra thick] (0.8, 2.15) rectangle (1.4, 2.35);
    \draw [ultra thick] (0.9, -0.5) rectangle (1.3, -0.7);
  \end{tikzpicture}
\end{center}
\captionof{figure}{A Michelson interferometer with signal recycling,
  using a comparable setup to figure \ref{fig:signal-recycle}, but with a
  mirror added between the beam-splitter and the end mirrors of each arm. \label{fig:fabry-perot}}
}
%
The arms of modern interferometers used in the detection of
gravitational-waves store the beam for a period of time comparable to
the timescale of the signals which are being searched for. In the case
of kilometre-scale detectors and signals with a period around
\SI{1}{\milli\second} this implies the need for the light to reflect
up-and-down the detector around $50$ times. This is achieved by laying
the reflected beams atop each other in a Fabry-Perot cavity, with a
\gls{finesse}, $\mathcal{F}=50$. A Fabry-Perot cavity is formed by
placing a mirror between the beam-splitter and the end mirror in each
arm, as illustrated in figure \ref{fig:fabry-perot}.

\subsection{Advanced LIGO}
\label{sec:advanced-ligo}

The Advanced LIGO detectors are 4-kilometre long interferometers with
Fabry-Perot cavities, with a finesse of 50. The detectors improve
their sensitivity compared to the initial generation detectors through
the use of signal recycling, a technology pioneered in the GEO
detector, and have quadruple mirror suspensions which use fused silica
fibres to provide seismic
islolation\cite{2002CQGra..19.4043R,2012CQGra..29w5004A}.
  \begin{figure}
  \centering
  \scalebox{0.5}{
  \newcommand{\convexpath}[2]{
  [   
  create hullcoords/.code={
    \global\edef\namelist{#1}
    \foreach [count=\counter] \nodename in \namelist {
      \global\edef\numberofnodes{\counter}
      \coordinate (hullcoord\counter) at (\nodename);
    }
    \coordinate (hullcoord0) at (hullcoord\numberofnodes);
    \pgfmathtruncatemacro\lastnumber{\numberofnodes+1}
    \coordinate (hullcoord\lastnumber) at (hullcoord1);
  },
  create hullcoords
  ]
  ($(hullcoord1)!#2!-90:(hullcoord0)$)
  \foreach [
  evaluate=\currentnode as \previousnode using \currentnode-1,
  evaluate=\currentnode as \nextnode using \currentnode+1
  ] \currentnode in {1,...,\numberofnodes} {
    let \p1 = ($(hullcoord\currentnode) - (hullcoord\previousnode)$),
    \n1 = {atan2(\y1,\x1) + 90},
    \p2 = ($(hullcoord\nextnode) - (hullcoord\currentnode)$),
    \n2 = {atan2(\y2,\x2) + 90},
    \n{delta} = {Mod(\n2-\n1,360) - 360}
    in 
    {arc [start angle=\n1, delta angle=\n{delta}, radius=#2]}
    -- ($(hullcoord\nextnode)!#2!-90:(hullcoord\currentnode)$) 
  }
}
\usetikzlibrary{calc} 
\begin{tikzpicture}[rotate=0]

\tikzstyle{mirror}=[fill=white]
\tikzstyle{FI}=[fill=yellow, thin]
\tikzstyle{BS}=[fill=pink, ultra thick, draw=pink, opacity=0.7]
\tikzstyle{IMC}=[draw=purple, ultra thick]
\tikzstyle{IM}=[draw=black, ultra thick]
\tikzstyle{PR}=[draw=lime, ultra thick]
\tikzstyle{IO}=[draw=gray, ultra thick]
\tikzstyle{laser}=[draw=red!80, ultra thick]
\tikzstyle{ham}=[fill=gray, opacity=0.1]
\tikzstyle{system}=[opacity=0.2]

\begin{scope} 
\draw [ham] (0,0) circle (2);
\draw (0, 2.5) node {BSC2};

\draw [BS, rotate around={45:(0,0)}] (0.5,0.1) rectangle (-0.5, -0.1) coordinate [midway] (BS1) {};

\end{scope}

\begin{scope} [xshift = 5cm]
\draw [ham] (0,0) circle (2);
\draw (0, 2.5) node {BSC3};
\end{scope}

\begin{scope} [xshift = 15cm]
\draw [ham] (0,0) circle (2);
\draw (0, 2.5) node {BSC4};
\end{scope}

\begin{scope} [yshift = 5cm]
\draw [ham] (0,0) circle (2);
\draw (0, 2.5) node {BSC1};
\end{scope}

\begin{scope} [yshift = 15cm]
\draw [ham] (0,0) circle (2);
\draw (0, 2.5) node {BSC10};
\end{scope}

\begin{scope}[xshift=-18cm]
\fill (-0.5, -0.5) rectangle (0.5, 0.5) node [midway] (PSL) {};
\draw (0., 0.75) node [above] {PSL};
\end{scope}

\begin{scope}[xshift=-15cm]
\fill [ham] (0, 0) circle (2);
\draw(0,2.5) node {HAM1};
\end{scope}

\begin{scope}[xshift=-10cm]
%%% HAM 2
\fill [ham] (0, 0) circle (2);
\draw(0,2.5) node {HAM2};
\draw [IMC, mirror, yshift = 0.75cm, rotate around={45:(0.5,0.6)}] (0,0.5) rectangle (0.75, 0.7) coordinate [midway] (MC3) {};
\draw [IMC, mirror, yshift = 0.25cm, rotate around={-45:(0.5,0.1)}] (0,0) rectangle (0.75,.2) coordinate [midway] (MC1) {};

\draw [IO, mirror, yshift = -0.95cm, xshift=0.3cm, rotate around={-45:(0.5,0.6)}] (0,0.5) -- (0, 0.7) coordinate [midway] (IO1) {};
\draw [IO, mirror, yshift =-.5cm, xshift=-0.3cm, rotate around={-45:(0.5,0.6)}] (0,0.5) -- (0, 0.7) coordinate [midway] (IO2) {};

\draw [PR, mirror,
	xshift = 0.5cm,
	 yshift = -0.63cm, rotate around={80:(0.5,0.6)}] (0,0.5) rectangle (0.75, 0.7) coordinate [ midway] (PR3) {};
\draw [PR, mirror,
	xshift = -0.4cm,
	 yshift = -1.33cm, rotate around={90:(0.5,0.6)}] (0,0.5) rectangle (0.75, 0.7) coordinate [midway] (PRM) {};

\draw [IM, mirror, xshift=-1.4cm,  yshift = 0.85cm, rotate around={25:(0.5,0.6)}] (0,0.5) rectangle (0.5, 0.7) coordinate [midway] (IM1) {};
\draw [IM, mirror, xshift=-1.7cm,  yshift = -1.5cm, rotate around={-25:(0.5,0.6)}] (0,0.5) rectangle (0.5, 0.7) coordinate [midway] (IM2) {};
\draw [IM, mirror, xshift=-1cm,  yshift = -0.2cm, rotate around={-25:(0.5,0.6)}] (0,0.5) rectangle (0.5, 0.7) coordinate [midway] (IM3) {};
\draw [IM, mirror, xshift=-1cm,  yshift = -1.65cm, rotate around={-45:(0.5,0.6)}] (0,0.5) rectangle (0.5, 0.7) coordinate [midway] (IM4) {};

\draw [IM, FI, xshift=-1.5cm,  yshift = -.60cm, rotate around={60:(0.5,0.6)}] (0,0.5) rectangle (0.3, 0.68) coordinate [midway] (FI1) {};


\end{scope}

\begin{scope}[xshift=-5cm]
%%% HAM 3
\fill [ham] (0, 0) circle (2);
\draw(0,2.5) node {HAM3};
\draw [IMC, mirror,
	xshift = -1cm,
	 yshift = 0.33cm, rotate around={90:(0.5,0.6)}] (0,0.5) rectangle (0.75, 0.7) coordinate [left=0.3cm, midway] (MC2) {};
\draw [PR, mirror,
	xshift = 0.3cm,
	 yshift = -1.33cm, rotate around={80:(0.5,0.6)}] (0,0.5) rectangle (0.75, 0.7) coordinate [midway] (PR2) {};

\end{scope}

\begin{scope}[yshift=-5cm]
\fill [ham] (0, 0) circle (2);
\draw(0,2.5) node {HAM4};
\end{scope}
\begin{scope}[yshift=-10cm]
\fill [ham] (0, 0) circle (2);
\draw(0,2.5) node {HAM5};
\end{scope}
\begin{scope}[yshift=-15cm]
\fill [ham] (0, 0) circle (2);
\draw(0,2.5) node {HAM6};
\end{scope}


\draw [laser, thick] (PSL) -- (IO1) -- (IO2) -- (MC1);
\draw [laser, thick] (MC3) -- (IM1) -- (IM2) -- (IM3) -- (IM4) -- (PRM);
\fill [black, system] \convexpath{IM1, IM3, IM4, IM2}{0.4cm};
%%% Mode-cleaner laser
\draw [laser] (MC1)--(MC2)--(MC3) -- cycle;
\fill [purple, system, thick, opacity=0.2] \convexpath{MC2, MC1, MC3}{.7cm};

\draw [laser] (PRM) -- (PR2) -- (PR3);
\fill[lime, system,] \convexpath{PR3, PR2, PRM}{0.7cm};

\draw [laser] (PR3) -- (BS1);

\end{tikzpicture}
  }
  \caption{The optical layout of the Advanced LIGO interferometer during its second operating run, based off LIGO-D0902838.}
  \label{fig:aligo-optical-layout}
\end{figure}

%%% Local Variables: 
%%% mode: latex
%%% TeX-master: "../../document"
%%% End: 




% \subsection{Space-based interferometers}
% \label{sec:space-based-interf}

% \subsection{Earth normal modes}
% \label{sec:earth-normal-modes}

% \subsection{Spacecraft telemetry}
% \label{sec:spacecraft-telemetry}



 \section{Noise sources}
 \label{sec:an-overview-noise}
 
\subsection{Shot noise}
\label{sec:shot-noise}

\subsection{Radiation pressure noise}
\label{sec:radi-press-noise}

\subsection{Thermal noise---Coatings}
\label{sec:therm-noise-coat}

\subsection{Thermal noise---suspensions}
\label{sec:therm-noise-susp}


\subsection{Newtonian Noise}
\label{sec:newtonian-noise}

Newtonian noise, or gravitational gradient noise, is the strain
produced by gravitational coupling between local mass density
variations and the test masses in the interferometer. Examples of
significant sources of Newtonian noise include clouds passing overhead
the detector, and seismic perturbations in the local ground density.

\subsection{Seismic Noise}
\label{sec:seismic-noise}

Seismic noise is the result of strain introduced into the
interferometer through movement of the ground, which can be the result
of geophysical activity, tidal activity, or anthropogenic sources of
seismic noise, such as road traffic or railways. Seismic noise is also
a source of Newtonian noise (see section \ref{sec:newtonian-noise})
due to density fluctuations as the seismic wave passes through the
ground.

% \marginpar{
%   \begin{tabular}{ccl}
%     $f$ [Hz] & $D$ [km] & Sources \\
%     0.01--1.0    &  1000         & earthquakes, microseism
%   \end{tabular}
% }

Seismic noise limits the sensitivity of the second generation
detectors at low frequencies ($f < \SI{10}{\hertz}$), but it is
present as a noise source across the passband of the detector. The
seismic noise shows a pair of notable peaks below the $\SI{1}{\hertz}$
level, one caused by ocean swell, which has a period around 4 to 30
seconds, and a second caused by standing seismic modes in the Earth
which spans the range of 30 to 1000 seconds.

Seismic isolation is used in detectors to reduce the noise level due
to seismic activity. This takes two forms: active isolation, and
passive isolation. The former is accomplished by mounting optical
components on servo-controlled systems which are controlled, via a
feedback-loop, to a seismometer. The latter is reduced by suspending
the optics as a component in a pendulum system. In the Advanced LIGO
design this involves the test masses and their associated mirrors
composing the final component in a quadruple pendulum suspension.



\subsection{Other noise sources}
\label{sec:other-noise-sources}
There are numerous additional noise sources within the interferometer.

%%% Local Variables: 
%%% mode: latex
%%% TeX-master: "../../document"
%%% End: 



 \chapter{Sources of Gravitational Waves}
 \label{cha:sourc-grav-waves}

 \chapterprecis{Gravitational waves are produced by any situation
   containing accelerating masses which are arranged in an asymmetrical
   manner, for example binary star systems, or non-spherical pulsars. A
   wide range of astrophysical sources are capable of producing
   gravitational waves, although only a handful of these are likely to
   be luminous enough to detect, or produce radiation over a frequency
   band which can be detected by current-generation detectors.}

%\section{Sources of gravitational radiation}
%\label{sec:sources}


The astrophysical sources of gravitational waves can be divided
roughly into three categories\cite{2009LRR....12....2S}:
\begin{description}
\item[Continuous] sources are expected to produce radiation
  constantly, or at least over long periods of time. The primary
  source of continuous sources for advanced LIGO are expected to be
  gravitational wave pulsars, but in detectors which are sensitive at
  lower frequencies, such as the proposed eLISA mission, the radiation
  from inspiralling binary systems should also be detectable.
\item[Transient] sources produce a strong \emph{burst} of
  gravitational waves over a period of seconds or less. These are
  sources which are primarily expected in the advanced LIGO passband,
  with compact binary coalesences and supernovae being major targets
  for burst searches in the advanced observing runs, however there are
  prospects for burst sources in the eLISA regime, for example from
  hyperbolic encounters between compact objects and stars or other
  compact objects
  \cite{2012PhRvD..86l4012B,2012PhRvD..86d4017D,2008MPLA...23...99C,2008APh....30..105C,2010MmSAI..81...87D,2005PhRvD..72h4009G,2010PhRvD..82j7501B,2011ApJ...729L..23G}.
\item[Stochastic] sources are expected to produce a background of
  gravitational waves, from the black holes at the centres of
  galaxies\cite{1980Natur.287..307B,2001astro.ph..8028P,2003ApJ...583..616J,2008MNRAS.390..192S},
  and from the universe's inflationary period\cite{1988PhRvD..37.2078A}.
\end{description}


%%% Local Variables: 
%%% mode: latex
%%% TeX-master: "../../document"
%%% End: 


 \section{Compact Binary Coalescence}
 \label{sec:cbc}
 
\sidebar{
    %% Creator: Matplotlib, PGF backend
%%
%% To include the figure in your LaTeX document, write
%%   \input{<filename>.pgf}
%%
%% Make sure the required packages are loaded in your preamble
%%   \usepackage{pgf}
%%
%% Figures using additional raster images can only be included by \input if
%% they are in the same directory as the main LaTeX file. For loading figures
%% from other directories you can use the `import` package
%%   \usepackage{import}
%% and then include the figures with
%%   \import{<path to file>}{<filename>.pgf}
%%
%% Matplotlib used the following preamble
%%   \usepackage{fontspec}
%%   \setmainfont{DejaVu Serif}
%%   \setsansfont{DejaVu Sans}
%%   \setmonofont{DejaVu Sans Mono}
%%
\begingroup%
\makeatletter%
\begin{pgfpicture}%
\pgfpathrectangle{\pgfpointorigin}{\pgfqpoint{2.500000in}{2.500000in}}%
\pgfusepath{use as bounding box, clip}%
\begin{pgfscope}%
\pgfsetbuttcap%
\pgfsetmiterjoin%
\definecolor{currentfill}{rgb}{1.000000,1.000000,1.000000}%
\pgfsetfillcolor{currentfill}%
\pgfsetlinewidth{0.000000pt}%
\definecolor{currentstroke}{rgb}{1.000000,1.000000,1.000000}%
\pgfsetstrokecolor{currentstroke}%
\pgfsetdash{}{0pt}%
\pgfpathmoveto{\pgfqpoint{0.000000in}{0.000000in}}%
\pgfpathlineto{\pgfqpoint{2.500000in}{0.000000in}}%
\pgfpathlineto{\pgfqpoint{2.500000in}{2.500000in}}%
\pgfpathlineto{\pgfqpoint{0.000000in}{2.500000in}}%
\pgfpathclose%
\pgfusepath{fill}%
\end{pgfscope}%
\begin{pgfscope}%
\pgfsetbuttcap%
\pgfsetmiterjoin%
\definecolor{currentfill}{rgb}{1.000000,1.000000,1.000000}%
\pgfsetfillcolor{currentfill}%
\pgfsetlinewidth{0.000000pt}%
\definecolor{currentstroke}{rgb}{0.000000,0.000000,0.000000}%
\pgfsetstrokecolor{currentstroke}%
\pgfsetstrokeopacity{0.000000}%
\pgfsetdash{}{0pt}%
\pgfpathmoveto{\pgfqpoint{0.651304in}{0.523750in}}%
\pgfpathlineto{\pgfqpoint{2.272067in}{0.523750in}}%
\pgfpathlineto{\pgfqpoint{2.272067in}{2.296875in}}%
\pgfpathlineto{\pgfqpoint{0.651304in}{2.296875in}}%
\pgfpathclose%
\pgfusepath{fill}%
\end{pgfscope}%
\begin{pgfscope}%
\pgfpathrectangle{\pgfqpoint{0.651304in}{0.523750in}}{\pgfqpoint{1.620763in}{1.773125in}} %
\pgfusepath{clip}%
\pgfsetrectcap%
\pgfsetroundjoin%
\pgfsetlinewidth{0.501875pt}%
\definecolor{currentstroke}{rgb}{0.800000,0.800000,0.800000}%
\pgfsetstrokecolor{currentstroke}%
\pgfsetdash{}{0pt}%
\pgfpathmoveto{\pgfqpoint{0.651304in}{0.523750in}}%
\pgfpathlineto{\pgfqpoint{0.651304in}{2.296875in}}%
\pgfusepath{stroke}%
\end{pgfscope}%
\begin{pgfscope}%
\pgfsetbuttcap%
\pgfsetroundjoin%
\definecolor{currentfill}{rgb}{0.000000,0.000000,0.000000}%
\pgfsetfillcolor{currentfill}%
\pgfsetlinewidth{0.501875pt}%
\definecolor{currentstroke}{rgb}{0.000000,0.000000,0.000000}%
\pgfsetstrokecolor{currentstroke}%
\pgfsetdash{}{0pt}%
\pgfsys@defobject{currentmarker}{\pgfqpoint{0.000000in}{0.000000in}}{\pgfqpoint{0.000000in}{0.055556in}}{%
\pgfpathmoveto{\pgfqpoint{0.000000in}{0.000000in}}%
\pgfpathlineto{\pgfqpoint{0.000000in}{0.055556in}}%
\pgfusepath{stroke,fill}%
}%
\begin{pgfscope}%
\pgfsys@transformshift{0.651304in}{0.523750in}%
\pgfsys@useobject{currentmarker}{}%
\end{pgfscope}%
\end{pgfscope}%
\begin{pgfscope}%
\pgfsetbuttcap%
\pgfsetroundjoin%
\definecolor{currentfill}{rgb}{0.000000,0.000000,0.000000}%
\pgfsetfillcolor{currentfill}%
\pgfsetlinewidth{0.501875pt}%
\definecolor{currentstroke}{rgb}{0.000000,0.000000,0.000000}%
\pgfsetstrokecolor{currentstroke}%
\pgfsetdash{}{0pt}%
\pgfsys@defobject{currentmarker}{\pgfqpoint{0.000000in}{-0.055556in}}{\pgfqpoint{0.000000in}{0.000000in}}{%
\pgfpathmoveto{\pgfqpoint{0.000000in}{0.000000in}}%
\pgfpathlineto{\pgfqpoint{0.000000in}{-0.055556in}}%
\pgfusepath{stroke,fill}%
}%
\begin{pgfscope}%
\pgfsys@transformshift{0.651304in}{2.296875in}%
\pgfsys@useobject{currentmarker}{}%
\end{pgfscope}%
\end{pgfscope}%
\begin{pgfscope}%
\pgftext[x=0.651304in,y=0.468194in,,top]{\fontsize{8.000000}{9.600000}\selectfont \(\displaystyle {10^{1}}\)}%
\end{pgfscope}%
\begin{pgfscope}%
\pgfpathrectangle{\pgfqpoint{0.651304in}{0.523750in}}{\pgfqpoint{1.620763in}{1.773125in}} %
\pgfusepath{clip}%
\pgfsetrectcap%
\pgfsetroundjoin%
\pgfsetlinewidth{0.501875pt}%
\definecolor{currentstroke}{rgb}{0.800000,0.800000,0.800000}%
\pgfsetstrokecolor{currentstroke}%
\pgfsetdash{}{0pt}%
\pgfpathmoveto{\pgfqpoint{1.461686in}{0.523750in}}%
\pgfpathlineto{\pgfqpoint{1.461686in}{2.296875in}}%
\pgfusepath{stroke}%
\end{pgfscope}%
\begin{pgfscope}%
\pgfsetbuttcap%
\pgfsetroundjoin%
\definecolor{currentfill}{rgb}{0.000000,0.000000,0.000000}%
\pgfsetfillcolor{currentfill}%
\pgfsetlinewidth{0.501875pt}%
\definecolor{currentstroke}{rgb}{0.000000,0.000000,0.000000}%
\pgfsetstrokecolor{currentstroke}%
\pgfsetdash{}{0pt}%
\pgfsys@defobject{currentmarker}{\pgfqpoint{0.000000in}{0.000000in}}{\pgfqpoint{0.000000in}{0.055556in}}{%
\pgfpathmoveto{\pgfqpoint{0.000000in}{0.000000in}}%
\pgfpathlineto{\pgfqpoint{0.000000in}{0.055556in}}%
\pgfusepath{stroke,fill}%
}%
\begin{pgfscope}%
\pgfsys@transformshift{1.461686in}{0.523750in}%
\pgfsys@useobject{currentmarker}{}%
\end{pgfscope}%
\end{pgfscope}%
\begin{pgfscope}%
\pgfsetbuttcap%
\pgfsetroundjoin%
\definecolor{currentfill}{rgb}{0.000000,0.000000,0.000000}%
\pgfsetfillcolor{currentfill}%
\pgfsetlinewidth{0.501875pt}%
\definecolor{currentstroke}{rgb}{0.000000,0.000000,0.000000}%
\pgfsetstrokecolor{currentstroke}%
\pgfsetdash{}{0pt}%
\pgfsys@defobject{currentmarker}{\pgfqpoint{0.000000in}{-0.055556in}}{\pgfqpoint{0.000000in}{0.000000in}}{%
\pgfpathmoveto{\pgfqpoint{0.000000in}{0.000000in}}%
\pgfpathlineto{\pgfqpoint{0.000000in}{-0.055556in}}%
\pgfusepath{stroke,fill}%
}%
\begin{pgfscope}%
\pgfsys@transformshift{1.461686in}{2.296875in}%
\pgfsys@useobject{currentmarker}{}%
\end{pgfscope}%
\end{pgfscope}%
\begin{pgfscope}%
\pgftext[x=1.461686in,y=0.468194in,,top]{\fontsize{8.000000}{9.600000}\selectfont \(\displaystyle {10^{2}}\)}%
\end{pgfscope}%
\begin{pgfscope}%
\pgfpathrectangle{\pgfqpoint{0.651304in}{0.523750in}}{\pgfqpoint{1.620763in}{1.773125in}} %
\pgfusepath{clip}%
\pgfsetrectcap%
\pgfsetroundjoin%
\pgfsetlinewidth{0.501875pt}%
\definecolor{currentstroke}{rgb}{0.800000,0.800000,0.800000}%
\pgfsetstrokecolor{currentstroke}%
\pgfsetdash{}{0pt}%
\pgfpathmoveto{\pgfqpoint{2.272067in}{0.523750in}}%
\pgfpathlineto{\pgfqpoint{2.272067in}{2.296875in}}%
\pgfusepath{stroke}%
\end{pgfscope}%
\begin{pgfscope}%
\pgfsetbuttcap%
\pgfsetroundjoin%
\definecolor{currentfill}{rgb}{0.000000,0.000000,0.000000}%
\pgfsetfillcolor{currentfill}%
\pgfsetlinewidth{0.501875pt}%
\definecolor{currentstroke}{rgb}{0.000000,0.000000,0.000000}%
\pgfsetstrokecolor{currentstroke}%
\pgfsetdash{}{0pt}%
\pgfsys@defobject{currentmarker}{\pgfqpoint{0.000000in}{0.000000in}}{\pgfqpoint{0.000000in}{0.055556in}}{%
\pgfpathmoveto{\pgfqpoint{0.000000in}{0.000000in}}%
\pgfpathlineto{\pgfqpoint{0.000000in}{0.055556in}}%
\pgfusepath{stroke,fill}%
}%
\begin{pgfscope}%
\pgfsys@transformshift{2.272067in}{0.523750in}%
\pgfsys@useobject{currentmarker}{}%
\end{pgfscope}%
\end{pgfscope}%
\begin{pgfscope}%
\pgfsetbuttcap%
\pgfsetroundjoin%
\definecolor{currentfill}{rgb}{0.000000,0.000000,0.000000}%
\pgfsetfillcolor{currentfill}%
\pgfsetlinewidth{0.501875pt}%
\definecolor{currentstroke}{rgb}{0.000000,0.000000,0.000000}%
\pgfsetstrokecolor{currentstroke}%
\pgfsetdash{}{0pt}%
\pgfsys@defobject{currentmarker}{\pgfqpoint{0.000000in}{-0.055556in}}{\pgfqpoint{0.000000in}{0.000000in}}{%
\pgfpathmoveto{\pgfqpoint{0.000000in}{0.000000in}}%
\pgfpathlineto{\pgfqpoint{0.000000in}{-0.055556in}}%
\pgfusepath{stroke,fill}%
}%
\begin{pgfscope}%
\pgfsys@transformshift{2.272067in}{2.296875in}%
\pgfsys@useobject{currentmarker}{}%
\end{pgfscope}%
\end{pgfscope}%
\begin{pgfscope}%
\pgftext[x=2.272067in,y=0.468194in,,top]{\fontsize{8.000000}{9.600000}\selectfont \(\displaystyle {10^{3}}\)}%
\end{pgfscope}%
\begin{pgfscope}%
\pgfsetbuttcap%
\pgfsetroundjoin%
\definecolor{currentfill}{rgb}{0.000000,0.000000,0.000000}%
\pgfsetfillcolor{currentfill}%
\pgfsetlinewidth{0.501875pt}%
\definecolor{currentstroke}{rgb}{0.000000,0.000000,0.000000}%
\pgfsetstrokecolor{currentstroke}%
\pgfsetdash{}{0pt}%
\pgfsys@defobject{currentmarker}{\pgfqpoint{0.000000in}{0.000000in}}{\pgfqpoint{0.000000in}{0.027778in}}{%
\pgfpathmoveto{\pgfqpoint{0.000000in}{0.000000in}}%
\pgfpathlineto{\pgfqpoint{0.000000in}{0.027778in}}%
\pgfusepath{stroke,fill}%
}%
\begin{pgfscope}%
\pgfsys@transformshift{0.895253in}{0.523750in}%
\pgfsys@useobject{currentmarker}{}%
\end{pgfscope}%
\end{pgfscope}%
\begin{pgfscope}%
\pgfsetbuttcap%
\pgfsetroundjoin%
\definecolor{currentfill}{rgb}{0.000000,0.000000,0.000000}%
\pgfsetfillcolor{currentfill}%
\pgfsetlinewidth{0.501875pt}%
\definecolor{currentstroke}{rgb}{0.000000,0.000000,0.000000}%
\pgfsetstrokecolor{currentstroke}%
\pgfsetdash{}{0pt}%
\pgfsys@defobject{currentmarker}{\pgfqpoint{0.000000in}{-0.027778in}}{\pgfqpoint{0.000000in}{0.000000in}}{%
\pgfpathmoveto{\pgfqpoint{0.000000in}{0.000000in}}%
\pgfpathlineto{\pgfqpoint{0.000000in}{-0.027778in}}%
\pgfusepath{stroke,fill}%
}%
\begin{pgfscope}%
\pgfsys@transformshift{0.895253in}{2.296875in}%
\pgfsys@useobject{currentmarker}{}%
\end{pgfscope}%
\end{pgfscope}%
\begin{pgfscope}%
\pgfsetbuttcap%
\pgfsetroundjoin%
\definecolor{currentfill}{rgb}{0.000000,0.000000,0.000000}%
\pgfsetfillcolor{currentfill}%
\pgfsetlinewidth{0.501875pt}%
\definecolor{currentstroke}{rgb}{0.000000,0.000000,0.000000}%
\pgfsetstrokecolor{currentstroke}%
\pgfsetdash{}{0pt}%
\pgfsys@defobject{currentmarker}{\pgfqpoint{0.000000in}{0.000000in}}{\pgfqpoint{0.000000in}{0.027778in}}{%
\pgfpathmoveto{\pgfqpoint{0.000000in}{0.000000in}}%
\pgfpathlineto{\pgfqpoint{0.000000in}{0.027778in}}%
\pgfusepath{stroke,fill}%
}%
\begin{pgfscope}%
\pgfsys@transformshift{1.037954in}{0.523750in}%
\pgfsys@useobject{currentmarker}{}%
\end{pgfscope}%
\end{pgfscope}%
\begin{pgfscope}%
\pgfsetbuttcap%
\pgfsetroundjoin%
\definecolor{currentfill}{rgb}{0.000000,0.000000,0.000000}%
\pgfsetfillcolor{currentfill}%
\pgfsetlinewidth{0.501875pt}%
\definecolor{currentstroke}{rgb}{0.000000,0.000000,0.000000}%
\pgfsetstrokecolor{currentstroke}%
\pgfsetdash{}{0pt}%
\pgfsys@defobject{currentmarker}{\pgfqpoint{0.000000in}{-0.027778in}}{\pgfqpoint{0.000000in}{0.000000in}}{%
\pgfpathmoveto{\pgfqpoint{0.000000in}{0.000000in}}%
\pgfpathlineto{\pgfqpoint{0.000000in}{-0.027778in}}%
\pgfusepath{stroke,fill}%
}%
\begin{pgfscope}%
\pgfsys@transformshift{1.037954in}{2.296875in}%
\pgfsys@useobject{currentmarker}{}%
\end{pgfscope}%
\end{pgfscope}%
\begin{pgfscope}%
\pgfsetbuttcap%
\pgfsetroundjoin%
\definecolor{currentfill}{rgb}{0.000000,0.000000,0.000000}%
\pgfsetfillcolor{currentfill}%
\pgfsetlinewidth{0.501875pt}%
\definecolor{currentstroke}{rgb}{0.000000,0.000000,0.000000}%
\pgfsetstrokecolor{currentstroke}%
\pgfsetdash{}{0pt}%
\pgfsys@defobject{currentmarker}{\pgfqpoint{0.000000in}{0.000000in}}{\pgfqpoint{0.000000in}{0.027778in}}{%
\pgfpathmoveto{\pgfqpoint{0.000000in}{0.000000in}}%
\pgfpathlineto{\pgfqpoint{0.000000in}{0.027778in}}%
\pgfusepath{stroke,fill}%
}%
\begin{pgfscope}%
\pgfsys@transformshift{1.139202in}{0.523750in}%
\pgfsys@useobject{currentmarker}{}%
\end{pgfscope}%
\end{pgfscope}%
\begin{pgfscope}%
\pgfsetbuttcap%
\pgfsetroundjoin%
\definecolor{currentfill}{rgb}{0.000000,0.000000,0.000000}%
\pgfsetfillcolor{currentfill}%
\pgfsetlinewidth{0.501875pt}%
\definecolor{currentstroke}{rgb}{0.000000,0.000000,0.000000}%
\pgfsetstrokecolor{currentstroke}%
\pgfsetdash{}{0pt}%
\pgfsys@defobject{currentmarker}{\pgfqpoint{0.000000in}{-0.027778in}}{\pgfqpoint{0.000000in}{0.000000in}}{%
\pgfpathmoveto{\pgfqpoint{0.000000in}{0.000000in}}%
\pgfpathlineto{\pgfqpoint{0.000000in}{-0.027778in}}%
\pgfusepath{stroke,fill}%
}%
\begin{pgfscope}%
\pgfsys@transformshift{1.139202in}{2.296875in}%
\pgfsys@useobject{currentmarker}{}%
\end{pgfscope}%
\end{pgfscope}%
\begin{pgfscope}%
\pgfsetbuttcap%
\pgfsetroundjoin%
\definecolor{currentfill}{rgb}{0.000000,0.000000,0.000000}%
\pgfsetfillcolor{currentfill}%
\pgfsetlinewidth{0.501875pt}%
\definecolor{currentstroke}{rgb}{0.000000,0.000000,0.000000}%
\pgfsetstrokecolor{currentstroke}%
\pgfsetdash{}{0pt}%
\pgfsys@defobject{currentmarker}{\pgfqpoint{0.000000in}{0.000000in}}{\pgfqpoint{0.000000in}{0.027778in}}{%
\pgfpathmoveto{\pgfqpoint{0.000000in}{0.000000in}}%
\pgfpathlineto{\pgfqpoint{0.000000in}{0.027778in}}%
\pgfusepath{stroke,fill}%
}%
\begin{pgfscope}%
\pgfsys@transformshift{1.217736in}{0.523750in}%
\pgfsys@useobject{currentmarker}{}%
\end{pgfscope}%
\end{pgfscope}%
\begin{pgfscope}%
\pgfsetbuttcap%
\pgfsetroundjoin%
\definecolor{currentfill}{rgb}{0.000000,0.000000,0.000000}%
\pgfsetfillcolor{currentfill}%
\pgfsetlinewidth{0.501875pt}%
\definecolor{currentstroke}{rgb}{0.000000,0.000000,0.000000}%
\pgfsetstrokecolor{currentstroke}%
\pgfsetdash{}{0pt}%
\pgfsys@defobject{currentmarker}{\pgfqpoint{0.000000in}{-0.027778in}}{\pgfqpoint{0.000000in}{0.000000in}}{%
\pgfpathmoveto{\pgfqpoint{0.000000in}{0.000000in}}%
\pgfpathlineto{\pgfqpoint{0.000000in}{-0.027778in}}%
\pgfusepath{stroke,fill}%
}%
\begin{pgfscope}%
\pgfsys@transformshift{1.217736in}{2.296875in}%
\pgfsys@useobject{currentmarker}{}%
\end{pgfscope}%
\end{pgfscope}%
\begin{pgfscope}%
\pgfsetbuttcap%
\pgfsetroundjoin%
\definecolor{currentfill}{rgb}{0.000000,0.000000,0.000000}%
\pgfsetfillcolor{currentfill}%
\pgfsetlinewidth{0.501875pt}%
\definecolor{currentstroke}{rgb}{0.000000,0.000000,0.000000}%
\pgfsetstrokecolor{currentstroke}%
\pgfsetdash{}{0pt}%
\pgfsys@defobject{currentmarker}{\pgfqpoint{0.000000in}{0.000000in}}{\pgfqpoint{0.000000in}{0.027778in}}{%
\pgfpathmoveto{\pgfqpoint{0.000000in}{0.000000in}}%
\pgfpathlineto{\pgfqpoint{0.000000in}{0.027778in}}%
\pgfusepath{stroke,fill}%
}%
\begin{pgfscope}%
\pgfsys@transformshift{1.281904in}{0.523750in}%
\pgfsys@useobject{currentmarker}{}%
\end{pgfscope}%
\end{pgfscope}%
\begin{pgfscope}%
\pgfsetbuttcap%
\pgfsetroundjoin%
\definecolor{currentfill}{rgb}{0.000000,0.000000,0.000000}%
\pgfsetfillcolor{currentfill}%
\pgfsetlinewidth{0.501875pt}%
\definecolor{currentstroke}{rgb}{0.000000,0.000000,0.000000}%
\pgfsetstrokecolor{currentstroke}%
\pgfsetdash{}{0pt}%
\pgfsys@defobject{currentmarker}{\pgfqpoint{0.000000in}{-0.027778in}}{\pgfqpoint{0.000000in}{0.000000in}}{%
\pgfpathmoveto{\pgfqpoint{0.000000in}{0.000000in}}%
\pgfpathlineto{\pgfqpoint{0.000000in}{-0.027778in}}%
\pgfusepath{stroke,fill}%
}%
\begin{pgfscope}%
\pgfsys@transformshift{1.281904in}{2.296875in}%
\pgfsys@useobject{currentmarker}{}%
\end{pgfscope}%
\end{pgfscope}%
\begin{pgfscope}%
\pgfsetbuttcap%
\pgfsetroundjoin%
\definecolor{currentfill}{rgb}{0.000000,0.000000,0.000000}%
\pgfsetfillcolor{currentfill}%
\pgfsetlinewidth{0.501875pt}%
\definecolor{currentstroke}{rgb}{0.000000,0.000000,0.000000}%
\pgfsetstrokecolor{currentstroke}%
\pgfsetdash{}{0pt}%
\pgfsys@defobject{currentmarker}{\pgfqpoint{0.000000in}{0.000000in}}{\pgfqpoint{0.000000in}{0.027778in}}{%
\pgfpathmoveto{\pgfqpoint{0.000000in}{0.000000in}}%
\pgfpathlineto{\pgfqpoint{0.000000in}{0.027778in}}%
\pgfusepath{stroke,fill}%
}%
\begin{pgfscope}%
\pgfsys@transformshift{1.336156in}{0.523750in}%
\pgfsys@useobject{currentmarker}{}%
\end{pgfscope}%
\end{pgfscope}%
\begin{pgfscope}%
\pgfsetbuttcap%
\pgfsetroundjoin%
\definecolor{currentfill}{rgb}{0.000000,0.000000,0.000000}%
\pgfsetfillcolor{currentfill}%
\pgfsetlinewidth{0.501875pt}%
\definecolor{currentstroke}{rgb}{0.000000,0.000000,0.000000}%
\pgfsetstrokecolor{currentstroke}%
\pgfsetdash{}{0pt}%
\pgfsys@defobject{currentmarker}{\pgfqpoint{0.000000in}{-0.027778in}}{\pgfqpoint{0.000000in}{0.000000in}}{%
\pgfpathmoveto{\pgfqpoint{0.000000in}{0.000000in}}%
\pgfpathlineto{\pgfqpoint{0.000000in}{-0.027778in}}%
\pgfusepath{stroke,fill}%
}%
\begin{pgfscope}%
\pgfsys@transformshift{1.336156in}{2.296875in}%
\pgfsys@useobject{currentmarker}{}%
\end{pgfscope}%
\end{pgfscope}%
\begin{pgfscope}%
\pgfsetbuttcap%
\pgfsetroundjoin%
\definecolor{currentfill}{rgb}{0.000000,0.000000,0.000000}%
\pgfsetfillcolor{currentfill}%
\pgfsetlinewidth{0.501875pt}%
\definecolor{currentstroke}{rgb}{0.000000,0.000000,0.000000}%
\pgfsetstrokecolor{currentstroke}%
\pgfsetdash{}{0pt}%
\pgfsys@defobject{currentmarker}{\pgfqpoint{0.000000in}{0.000000in}}{\pgfqpoint{0.000000in}{0.027778in}}{%
\pgfpathmoveto{\pgfqpoint{0.000000in}{0.000000in}}%
\pgfpathlineto{\pgfqpoint{0.000000in}{0.027778in}}%
\pgfusepath{stroke,fill}%
}%
\begin{pgfscope}%
\pgfsys@transformshift{1.383152in}{0.523750in}%
\pgfsys@useobject{currentmarker}{}%
\end{pgfscope}%
\end{pgfscope}%
\begin{pgfscope}%
\pgfsetbuttcap%
\pgfsetroundjoin%
\definecolor{currentfill}{rgb}{0.000000,0.000000,0.000000}%
\pgfsetfillcolor{currentfill}%
\pgfsetlinewidth{0.501875pt}%
\definecolor{currentstroke}{rgb}{0.000000,0.000000,0.000000}%
\pgfsetstrokecolor{currentstroke}%
\pgfsetdash{}{0pt}%
\pgfsys@defobject{currentmarker}{\pgfqpoint{0.000000in}{-0.027778in}}{\pgfqpoint{0.000000in}{0.000000in}}{%
\pgfpathmoveto{\pgfqpoint{0.000000in}{0.000000in}}%
\pgfpathlineto{\pgfqpoint{0.000000in}{-0.027778in}}%
\pgfusepath{stroke,fill}%
}%
\begin{pgfscope}%
\pgfsys@transformshift{1.383152in}{2.296875in}%
\pgfsys@useobject{currentmarker}{}%
\end{pgfscope}%
\end{pgfscope}%
\begin{pgfscope}%
\pgfsetbuttcap%
\pgfsetroundjoin%
\definecolor{currentfill}{rgb}{0.000000,0.000000,0.000000}%
\pgfsetfillcolor{currentfill}%
\pgfsetlinewidth{0.501875pt}%
\definecolor{currentstroke}{rgb}{0.000000,0.000000,0.000000}%
\pgfsetstrokecolor{currentstroke}%
\pgfsetdash{}{0pt}%
\pgfsys@defobject{currentmarker}{\pgfqpoint{0.000000in}{0.000000in}}{\pgfqpoint{0.000000in}{0.027778in}}{%
\pgfpathmoveto{\pgfqpoint{0.000000in}{0.000000in}}%
\pgfpathlineto{\pgfqpoint{0.000000in}{0.027778in}}%
\pgfusepath{stroke,fill}%
}%
\begin{pgfscope}%
\pgfsys@transformshift{1.424605in}{0.523750in}%
\pgfsys@useobject{currentmarker}{}%
\end{pgfscope}%
\end{pgfscope}%
\begin{pgfscope}%
\pgfsetbuttcap%
\pgfsetroundjoin%
\definecolor{currentfill}{rgb}{0.000000,0.000000,0.000000}%
\pgfsetfillcolor{currentfill}%
\pgfsetlinewidth{0.501875pt}%
\definecolor{currentstroke}{rgb}{0.000000,0.000000,0.000000}%
\pgfsetstrokecolor{currentstroke}%
\pgfsetdash{}{0pt}%
\pgfsys@defobject{currentmarker}{\pgfqpoint{0.000000in}{-0.027778in}}{\pgfqpoint{0.000000in}{0.000000in}}{%
\pgfpathmoveto{\pgfqpoint{0.000000in}{0.000000in}}%
\pgfpathlineto{\pgfqpoint{0.000000in}{-0.027778in}}%
\pgfusepath{stroke,fill}%
}%
\begin{pgfscope}%
\pgfsys@transformshift{1.424605in}{2.296875in}%
\pgfsys@useobject{currentmarker}{}%
\end{pgfscope}%
\end{pgfscope}%
\begin{pgfscope}%
\pgfsetbuttcap%
\pgfsetroundjoin%
\definecolor{currentfill}{rgb}{0.000000,0.000000,0.000000}%
\pgfsetfillcolor{currentfill}%
\pgfsetlinewidth{0.501875pt}%
\definecolor{currentstroke}{rgb}{0.000000,0.000000,0.000000}%
\pgfsetstrokecolor{currentstroke}%
\pgfsetdash{}{0pt}%
\pgfsys@defobject{currentmarker}{\pgfqpoint{0.000000in}{0.000000in}}{\pgfqpoint{0.000000in}{0.027778in}}{%
\pgfpathmoveto{\pgfqpoint{0.000000in}{0.000000in}}%
\pgfpathlineto{\pgfqpoint{0.000000in}{0.027778in}}%
\pgfusepath{stroke,fill}%
}%
\begin{pgfscope}%
\pgfsys@transformshift{1.705635in}{0.523750in}%
\pgfsys@useobject{currentmarker}{}%
\end{pgfscope}%
\end{pgfscope}%
\begin{pgfscope}%
\pgfsetbuttcap%
\pgfsetroundjoin%
\definecolor{currentfill}{rgb}{0.000000,0.000000,0.000000}%
\pgfsetfillcolor{currentfill}%
\pgfsetlinewidth{0.501875pt}%
\definecolor{currentstroke}{rgb}{0.000000,0.000000,0.000000}%
\pgfsetstrokecolor{currentstroke}%
\pgfsetdash{}{0pt}%
\pgfsys@defobject{currentmarker}{\pgfqpoint{0.000000in}{-0.027778in}}{\pgfqpoint{0.000000in}{0.000000in}}{%
\pgfpathmoveto{\pgfqpoint{0.000000in}{0.000000in}}%
\pgfpathlineto{\pgfqpoint{0.000000in}{-0.027778in}}%
\pgfusepath{stroke,fill}%
}%
\begin{pgfscope}%
\pgfsys@transformshift{1.705635in}{2.296875in}%
\pgfsys@useobject{currentmarker}{}%
\end{pgfscope}%
\end{pgfscope}%
\begin{pgfscope}%
\pgfsetbuttcap%
\pgfsetroundjoin%
\definecolor{currentfill}{rgb}{0.000000,0.000000,0.000000}%
\pgfsetfillcolor{currentfill}%
\pgfsetlinewidth{0.501875pt}%
\definecolor{currentstroke}{rgb}{0.000000,0.000000,0.000000}%
\pgfsetstrokecolor{currentstroke}%
\pgfsetdash{}{0pt}%
\pgfsys@defobject{currentmarker}{\pgfqpoint{0.000000in}{0.000000in}}{\pgfqpoint{0.000000in}{0.027778in}}{%
\pgfpathmoveto{\pgfqpoint{0.000000in}{0.000000in}}%
\pgfpathlineto{\pgfqpoint{0.000000in}{0.027778in}}%
\pgfusepath{stroke,fill}%
}%
\begin{pgfscope}%
\pgfsys@transformshift{1.848336in}{0.523750in}%
\pgfsys@useobject{currentmarker}{}%
\end{pgfscope}%
\end{pgfscope}%
\begin{pgfscope}%
\pgfsetbuttcap%
\pgfsetroundjoin%
\definecolor{currentfill}{rgb}{0.000000,0.000000,0.000000}%
\pgfsetfillcolor{currentfill}%
\pgfsetlinewidth{0.501875pt}%
\definecolor{currentstroke}{rgb}{0.000000,0.000000,0.000000}%
\pgfsetstrokecolor{currentstroke}%
\pgfsetdash{}{0pt}%
\pgfsys@defobject{currentmarker}{\pgfqpoint{0.000000in}{-0.027778in}}{\pgfqpoint{0.000000in}{0.000000in}}{%
\pgfpathmoveto{\pgfqpoint{0.000000in}{0.000000in}}%
\pgfpathlineto{\pgfqpoint{0.000000in}{-0.027778in}}%
\pgfusepath{stroke,fill}%
}%
\begin{pgfscope}%
\pgfsys@transformshift{1.848336in}{2.296875in}%
\pgfsys@useobject{currentmarker}{}%
\end{pgfscope}%
\end{pgfscope}%
\begin{pgfscope}%
\pgfsetbuttcap%
\pgfsetroundjoin%
\definecolor{currentfill}{rgb}{0.000000,0.000000,0.000000}%
\pgfsetfillcolor{currentfill}%
\pgfsetlinewidth{0.501875pt}%
\definecolor{currentstroke}{rgb}{0.000000,0.000000,0.000000}%
\pgfsetstrokecolor{currentstroke}%
\pgfsetdash{}{0pt}%
\pgfsys@defobject{currentmarker}{\pgfqpoint{0.000000in}{0.000000in}}{\pgfqpoint{0.000000in}{0.027778in}}{%
\pgfpathmoveto{\pgfqpoint{0.000000in}{0.000000in}}%
\pgfpathlineto{\pgfqpoint{0.000000in}{0.027778in}}%
\pgfusepath{stroke,fill}%
}%
\begin{pgfscope}%
\pgfsys@transformshift{1.949584in}{0.523750in}%
\pgfsys@useobject{currentmarker}{}%
\end{pgfscope}%
\end{pgfscope}%
\begin{pgfscope}%
\pgfsetbuttcap%
\pgfsetroundjoin%
\definecolor{currentfill}{rgb}{0.000000,0.000000,0.000000}%
\pgfsetfillcolor{currentfill}%
\pgfsetlinewidth{0.501875pt}%
\definecolor{currentstroke}{rgb}{0.000000,0.000000,0.000000}%
\pgfsetstrokecolor{currentstroke}%
\pgfsetdash{}{0pt}%
\pgfsys@defobject{currentmarker}{\pgfqpoint{0.000000in}{-0.027778in}}{\pgfqpoint{0.000000in}{0.000000in}}{%
\pgfpathmoveto{\pgfqpoint{0.000000in}{0.000000in}}%
\pgfpathlineto{\pgfqpoint{0.000000in}{-0.027778in}}%
\pgfusepath{stroke,fill}%
}%
\begin{pgfscope}%
\pgfsys@transformshift{1.949584in}{2.296875in}%
\pgfsys@useobject{currentmarker}{}%
\end{pgfscope}%
\end{pgfscope}%
\begin{pgfscope}%
\pgfsetbuttcap%
\pgfsetroundjoin%
\definecolor{currentfill}{rgb}{0.000000,0.000000,0.000000}%
\pgfsetfillcolor{currentfill}%
\pgfsetlinewidth{0.501875pt}%
\definecolor{currentstroke}{rgb}{0.000000,0.000000,0.000000}%
\pgfsetstrokecolor{currentstroke}%
\pgfsetdash{}{0pt}%
\pgfsys@defobject{currentmarker}{\pgfqpoint{0.000000in}{0.000000in}}{\pgfqpoint{0.000000in}{0.027778in}}{%
\pgfpathmoveto{\pgfqpoint{0.000000in}{0.000000in}}%
\pgfpathlineto{\pgfqpoint{0.000000in}{0.027778in}}%
\pgfusepath{stroke,fill}%
}%
\begin{pgfscope}%
\pgfsys@transformshift{2.028118in}{0.523750in}%
\pgfsys@useobject{currentmarker}{}%
\end{pgfscope}%
\end{pgfscope}%
\begin{pgfscope}%
\pgfsetbuttcap%
\pgfsetroundjoin%
\definecolor{currentfill}{rgb}{0.000000,0.000000,0.000000}%
\pgfsetfillcolor{currentfill}%
\pgfsetlinewidth{0.501875pt}%
\definecolor{currentstroke}{rgb}{0.000000,0.000000,0.000000}%
\pgfsetstrokecolor{currentstroke}%
\pgfsetdash{}{0pt}%
\pgfsys@defobject{currentmarker}{\pgfqpoint{0.000000in}{-0.027778in}}{\pgfqpoint{0.000000in}{0.000000in}}{%
\pgfpathmoveto{\pgfqpoint{0.000000in}{0.000000in}}%
\pgfpathlineto{\pgfqpoint{0.000000in}{-0.027778in}}%
\pgfusepath{stroke,fill}%
}%
\begin{pgfscope}%
\pgfsys@transformshift{2.028118in}{2.296875in}%
\pgfsys@useobject{currentmarker}{}%
\end{pgfscope}%
\end{pgfscope}%
\begin{pgfscope}%
\pgfsetbuttcap%
\pgfsetroundjoin%
\definecolor{currentfill}{rgb}{0.000000,0.000000,0.000000}%
\pgfsetfillcolor{currentfill}%
\pgfsetlinewidth{0.501875pt}%
\definecolor{currentstroke}{rgb}{0.000000,0.000000,0.000000}%
\pgfsetstrokecolor{currentstroke}%
\pgfsetdash{}{0pt}%
\pgfsys@defobject{currentmarker}{\pgfqpoint{0.000000in}{0.000000in}}{\pgfqpoint{0.000000in}{0.027778in}}{%
\pgfpathmoveto{\pgfqpoint{0.000000in}{0.000000in}}%
\pgfpathlineto{\pgfqpoint{0.000000in}{0.027778in}}%
\pgfusepath{stroke,fill}%
}%
\begin{pgfscope}%
\pgfsys@transformshift{2.092285in}{0.523750in}%
\pgfsys@useobject{currentmarker}{}%
\end{pgfscope}%
\end{pgfscope}%
\begin{pgfscope}%
\pgfsetbuttcap%
\pgfsetroundjoin%
\definecolor{currentfill}{rgb}{0.000000,0.000000,0.000000}%
\pgfsetfillcolor{currentfill}%
\pgfsetlinewidth{0.501875pt}%
\definecolor{currentstroke}{rgb}{0.000000,0.000000,0.000000}%
\pgfsetstrokecolor{currentstroke}%
\pgfsetdash{}{0pt}%
\pgfsys@defobject{currentmarker}{\pgfqpoint{0.000000in}{-0.027778in}}{\pgfqpoint{0.000000in}{0.000000in}}{%
\pgfpathmoveto{\pgfqpoint{0.000000in}{0.000000in}}%
\pgfpathlineto{\pgfqpoint{0.000000in}{-0.027778in}}%
\pgfusepath{stroke,fill}%
}%
\begin{pgfscope}%
\pgfsys@transformshift{2.092285in}{2.296875in}%
\pgfsys@useobject{currentmarker}{}%
\end{pgfscope}%
\end{pgfscope}%
\begin{pgfscope}%
\pgfsetbuttcap%
\pgfsetroundjoin%
\definecolor{currentfill}{rgb}{0.000000,0.000000,0.000000}%
\pgfsetfillcolor{currentfill}%
\pgfsetlinewidth{0.501875pt}%
\definecolor{currentstroke}{rgb}{0.000000,0.000000,0.000000}%
\pgfsetstrokecolor{currentstroke}%
\pgfsetdash{}{0pt}%
\pgfsys@defobject{currentmarker}{\pgfqpoint{0.000000in}{0.000000in}}{\pgfqpoint{0.000000in}{0.027778in}}{%
\pgfpathmoveto{\pgfqpoint{0.000000in}{0.000000in}}%
\pgfpathlineto{\pgfqpoint{0.000000in}{0.027778in}}%
\pgfusepath{stroke,fill}%
}%
\begin{pgfscope}%
\pgfsys@transformshift{2.146538in}{0.523750in}%
\pgfsys@useobject{currentmarker}{}%
\end{pgfscope}%
\end{pgfscope}%
\begin{pgfscope}%
\pgfsetbuttcap%
\pgfsetroundjoin%
\definecolor{currentfill}{rgb}{0.000000,0.000000,0.000000}%
\pgfsetfillcolor{currentfill}%
\pgfsetlinewidth{0.501875pt}%
\definecolor{currentstroke}{rgb}{0.000000,0.000000,0.000000}%
\pgfsetstrokecolor{currentstroke}%
\pgfsetdash{}{0pt}%
\pgfsys@defobject{currentmarker}{\pgfqpoint{0.000000in}{-0.027778in}}{\pgfqpoint{0.000000in}{0.000000in}}{%
\pgfpathmoveto{\pgfqpoint{0.000000in}{0.000000in}}%
\pgfpathlineto{\pgfqpoint{0.000000in}{-0.027778in}}%
\pgfusepath{stroke,fill}%
}%
\begin{pgfscope}%
\pgfsys@transformshift{2.146538in}{2.296875in}%
\pgfsys@useobject{currentmarker}{}%
\end{pgfscope}%
\end{pgfscope}%
\begin{pgfscope}%
\pgfsetbuttcap%
\pgfsetroundjoin%
\definecolor{currentfill}{rgb}{0.000000,0.000000,0.000000}%
\pgfsetfillcolor{currentfill}%
\pgfsetlinewidth{0.501875pt}%
\definecolor{currentstroke}{rgb}{0.000000,0.000000,0.000000}%
\pgfsetstrokecolor{currentstroke}%
\pgfsetdash{}{0pt}%
\pgfsys@defobject{currentmarker}{\pgfqpoint{0.000000in}{0.000000in}}{\pgfqpoint{0.000000in}{0.027778in}}{%
\pgfpathmoveto{\pgfqpoint{0.000000in}{0.000000in}}%
\pgfpathlineto{\pgfqpoint{0.000000in}{0.027778in}}%
\pgfusepath{stroke,fill}%
}%
\begin{pgfscope}%
\pgfsys@transformshift{2.193533in}{0.523750in}%
\pgfsys@useobject{currentmarker}{}%
\end{pgfscope}%
\end{pgfscope}%
\begin{pgfscope}%
\pgfsetbuttcap%
\pgfsetroundjoin%
\definecolor{currentfill}{rgb}{0.000000,0.000000,0.000000}%
\pgfsetfillcolor{currentfill}%
\pgfsetlinewidth{0.501875pt}%
\definecolor{currentstroke}{rgb}{0.000000,0.000000,0.000000}%
\pgfsetstrokecolor{currentstroke}%
\pgfsetdash{}{0pt}%
\pgfsys@defobject{currentmarker}{\pgfqpoint{0.000000in}{-0.027778in}}{\pgfqpoint{0.000000in}{0.000000in}}{%
\pgfpathmoveto{\pgfqpoint{0.000000in}{0.000000in}}%
\pgfpathlineto{\pgfqpoint{0.000000in}{-0.027778in}}%
\pgfusepath{stroke,fill}%
}%
\begin{pgfscope}%
\pgfsys@transformshift{2.193533in}{2.296875in}%
\pgfsys@useobject{currentmarker}{}%
\end{pgfscope}%
\end{pgfscope}%
\begin{pgfscope}%
\pgfsetbuttcap%
\pgfsetroundjoin%
\definecolor{currentfill}{rgb}{0.000000,0.000000,0.000000}%
\pgfsetfillcolor{currentfill}%
\pgfsetlinewidth{0.501875pt}%
\definecolor{currentstroke}{rgb}{0.000000,0.000000,0.000000}%
\pgfsetstrokecolor{currentstroke}%
\pgfsetdash{}{0pt}%
\pgfsys@defobject{currentmarker}{\pgfqpoint{0.000000in}{0.000000in}}{\pgfqpoint{0.000000in}{0.027778in}}{%
\pgfpathmoveto{\pgfqpoint{0.000000in}{0.000000in}}%
\pgfpathlineto{\pgfqpoint{0.000000in}{0.027778in}}%
\pgfusepath{stroke,fill}%
}%
\begin{pgfscope}%
\pgfsys@transformshift{2.234986in}{0.523750in}%
\pgfsys@useobject{currentmarker}{}%
\end{pgfscope}%
\end{pgfscope}%
\begin{pgfscope}%
\pgfsetbuttcap%
\pgfsetroundjoin%
\definecolor{currentfill}{rgb}{0.000000,0.000000,0.000000}%
\pgfsetfillcolor{currentfill}%
\pgfsetlinewidth{0.501875pt}%
\definecolor{currentstroke}{rgb}{0.000000,0.000000,0.000000}%
\pgfsetstrokecolor{currentstroke}%
\pgfsetdash{}{0pt}%
\pgfsys@defobject{currentmarker}{\pgfqpoint{0.000000in}{-0.027778in}}{\pgfqpoint{0.000000in}{0.000000in}}{%
\pgfpathmoveto{\pgfqpoint{0.000000in}{0.000000in}}%
\pgfpathlineto{\pgfqpoint{0.000000in}{-0.027778in}}%
\pgfusepath{stroke,fill}%
}%
\begin{pgfscope}%
\pgfsys@transformshift{2.234986in}{2.296875in}%
\pgfsys@useobject{currentmarker}{}%
\end{pgfscope}%
\end{pgfscope}%
\begin{pgfscope}%
\pgftext[x=1.461686in,y=0.291220in,,top]{\fontsize{8.000000}{9.600000}\selectfont Frequency [Hz]}%
\end{pgfscope}%
\begin{pgfscope}%
\pgfpathrectangle{\pgfqpoint{0.651304in}{0.523750in}}{\pgfqpoint{1.620763in}{1.773125in}} %
\pgfusepath{clip}%
\pgfsetrectcap%
\pgfsetroundjoin%
\pgfsetlinewidth{0.501875pt}%
\definecolor{currentstroke}{rgb}{0.800000,0.800000,0.800000}%
\pgfsetstrokecolor{currentstroke}%
\pgfsetdash{}{0pt}%
\pgfpathmoveto{\pgfqpoint{0.651304in}{0.523750in}}%
\pgfpathlineto{\pgfqpoint{2.272067in}{0.523750in}}%
\pgfusepath{stroke}%
\end{pgfscope}%
\begin{pgfscope}%
\pgfsetbuttcap%
\pgfsetroundjoin%
\definecolor{currentfill}{rgb}{0.000000,0.000000,0.000000}%
\pgfsetfillcolor{currentfill}%
\pgfsetlinewidth{0.501875pt}%
\definecolor{currentstroke}{rgb}{0.000000,0.000000,0.000000}%
\pgfsetstrokecolor{currentstroke}%
\pgfsetdash{}{0pt}%
\pgfsys@defobject{currentmarker}{\pgfqpoint{0.000000in}{0.000000in}}{\pgfqpoint{0.055556in}{0.000000in}}{%
\pgfpathmoveto{\pgfqpoint{0.000000in}{0.000000in}}%
\pgfpathlineto{\pgfqpoint{0.055556in}{0.000000in}}%
\pgfusepath{stroke,fill}%
}%
\begin{pgfscope}%
\pgfsys@transformshift{0.651304in}{0.523750in}%
\pgfsys@useobject{currentmarker}{}%
\end{pgfscope}%
\end{pgfscope}%
\begin{pgfscope}%
\pgfsetbuttcap%
\pgfsetroundjoin%
\definecolor{currentfill}{rgb}{0.000000,0.000000,0.000000}%
\pgfsetfillcolor{currentfill}%
\pgfsetlinewidth{0.501875pt}%
\definecolor{currentstroke}{rgb}{0.000000,0.000000,0.000000}%
\pgfsetstrokecolor{currentstroke}%
\pgfsetdash{}{0pt}%
\pgfsys@defobject{currentmarker}{\pgfqpoint{-0.055556in}{0.000000in}}{\pgfqpoint{0.000000in}{0.000000in}}{%
\pgfpathmoveto{\pgfqpoint{0.000000in}{0.000000in}}%
\pgfpathlineto{\pgfqpoint{-0.055556in}{0.000000in}}%
\pgfusepath{stroke,fill}%
}%
\begin{pgfscope}%
\pgfsys@transformshift{2.272067in}{0.523750in}%
\pgfsys@useobject{currentmarker}{}%
\end{pgfscope}%
\end{pgfscope}%
\begin{pgfscope}%
\pgftext[x=0.595748in,y=0.523750in,right,]{\fontsize{8.000000}{9.600000}\selectfont \(\displaystyle {10^{-23}}\)}%
\end{pgfscope}%
\begin{pgfscope}%
\pgfpathrectangle{\pgfqpoint{0.651304in}{0.523750in}}{\pgfqpoint{1.620763in}{1.773125in}} %
\pgfusepath{clip}%
\pgfsetrectcap%
\pgfsetroundjoin%
\pgfsetlinewidth{0.501875pt}%
\definecolor{currentstroke}{rgb}{0.800000,0.800000,0.800000}%
\pgfsetstrokecolor{currentstroke}%
\pgfsetdash{}{0pt}%
\pgfpathmoveto{\pgfqpoint{0.651304in}{0.967031in}}%
\pgfpathlineto{\pgfqpoint{2.272067in}{0.967031in}}%
\pgfusepath{stroke}%
\end{pgfscope}%
\begin{pgfscope}%
\pgfsetbuttcap%
\pgfsetroundjoin%
\definecolor{currentfill}{rgb}{0.000000,0.000000,0.000000}%
\pgfsetfillcolor{currentfill}%
\pgfsetlinewidth{0.501875pt}%
\definecolor{currentstroke}{rgb}{0.000000,0.000000,0.000000}%
\pgfsetstrokecolor{currentstroke}%
\pgfsetdash{}{0pt}%
\pgfsys@defobject{currentmarker}{\pgfqpoint{0.000000in}{0.000000in}}{\pgfqpoint{0.055556in}{0.000000in}}{%
\pgfpathmoveto{\pgfqpoint{0.000000in}{0.000000in}}%
\pgfpathlineto{\pgfqpoint{0.055556in}{0.000000in}}%
\pgfusepath{stroke,fill}%
}%
\begin{pgfscope}%
\pgfsys@transformshift{0.651304in}{0.967031in}%
\pgfsys@useobject{currentmarker}{}%
\end{pgfscope}%
\end{pgfscope}%
\begin{pgfscope}%
\pgfsetbuttcap%
\pgfsetroundjoin%
\definecolor{currentfill}{rgb}{0.000000,0.000000,0.000000}%
\pgfsetfillcolor{currentfill}%
\pgfsetlinewidth{0.501875pt}%
\definecolor{currentstroke}{rgb}{0.000000,0.000000,0.000000}%
\pgfsetstrokecolor{currentstroke}%
\pgfsetdash{}{0pt}%
\pgfsys@defobject{currentmarker}{\pgfqpoint{-0.055556in}{0.000000in}}{\pgfqpoint{0.000000in}{0.000000in}}{%
\pgfpathmoveto{\pgfqpoint{0.000000in}{0.000000in}}%
\pgfpathlineto{\pgfqpoint{-0.055556in}{0.000000in}}%
\pgfusepath{stroke,fill}%
}%
\begin{pgfscope}%
\pgfsys@transformshift{2.272067in}{0.967031in}%
\pgfsys@useobject{currentmarker}{}%
\end{pgfscope}%
\end{pgfscope}%
\begin{pgfscope}%
\pgftext[x=0.595748in,y=0.967031in,right,]{\fontsize{8.000000}{9.600000}\selectfont \(\displaystyle {10^{-22}}\)}%
\end{pgfscope}%
\begin{pgfscope}%
\pgfpathrectangle{\pgfqpoint{0.651304in}{0.523750in}}{\pgfqpoint{1.620763in}{1.773125in}} %
\pgfusepath{clip}%
\pgfsetrectcap%
\pgfsetroundjoin%
\pgfsetlinewidth{0.501875pt}%
\definecolor{currentstroke}{rgb}{0.800000,0.800000,0.800000}%
\pgfsetstrokecolor{currentstroke}%
\pgfsetdash{}{0pt}%
\pgfpathmoveto{\pgfqpoint{0.651304in}{1.410313in}}%
\pgfpathlineto{\pgfqpoint{2.272067in}{1.410313in}}%
\pgfusepath{stroke}%
\end{pgfscope}%
\begin{pgfscope}%
\pgfsetbuttcap%
\pgfsetroundjoin%
\definecolor{currentfill}{rgb}{0.000000,0.000000,0.000000}%
\pgfsetfillcolor{currentfill}%
\pgfsetlinewidth{0.501875pt}%
\definecolor{currentstroke}{rgb}{0.000000,0.000000,0.000000}%
\pgfsetstrokecolor{currentstroke}%
\pgfsetdash{}{0pt}%
\pgfsys@defobject{currentmarker}{\pgfqpoint{0.000000in}{0.000000in}}{\pgfqpoint{0.055556in}{0.000000in}}{%
\pgfpathmoveto{\pgfqpoint{0.000000in}{0.000000in}}%
\pgfpathlineto{\pgfqpoint{0.055556in}{0.000000in}}%
\pgfusepath{stroke,fill}%
}%
\begin{pgfscope}%
\pgfsys@transformshift{0.651304in}{1.410313in}%
\pgfsys@useobject{currentmarker}{}%
\end{pgfscope}%
\end{pgfscope}%
\begin{pgfscope}%
\pgfsetbuttcap%
\pgfsetroundjoin%
\definecolor{currentfill}{rgb}{0.000000,0.000000,0.000000}%
\pgfsetfillcolor{currentfill}%
\pgfsetlinewidth{0.501875pt}%
\definecolor{currentstroke}{rgb}{0.000000,0.000000,0.000000}%
\pgfsetstrokecolor{currentstroke}%
\pgfsetdash{}{0pt}%
\pgfsys@defobject{currentmarker}{\pgfqpoint{-0.055556in}{0.000000in}}{\pgfqpoint{0.000000in}{0.000000in}}{%
\pgfpathmoveto{\pgfqpoint{0.000000in}{0.000000in}}%
\pgfpathlineto{\pgfqpoint{-0.055556in}{0.000000in}}%
\pgfusepath{stroke,fill}%
}%
\begin{pgfscope}%
\pgfsys@transformshift{2.272067in}{1.410313in}%
\pgfsys@useobject{currentmarker}{}%
\end{pgfscope}%
\end{pgfscope}%
\begin{pgfscope}%
\pgftext[x=0.595748in,y=1.410313in,right,]{\fontsize{8.000000}{9.600000}\selectfont \(\displaystyle {10^{-21}}\)}%
\end{pgfscope}%
\begin{pgfscope}%
\pgfpathrectangle{\pgfqpoint{0.651304in}{0.523750in}}{\pgfqpoint{1.620763in}{1.773125in}} %
\pgfusepath{clip}%
\pgfsetrectcap%
\pgfsetroundjoin%
\pgfsetlinewidth{0.501875pt}%
\definecolor{currentstroke}{rgb}{0.800000,0.800000,0.800000}%
\pgfsetstrokecolor{currentstroke}%
\pgfsetdash{}{0pt}%
\pgfpathmoveto{\pgfqpoint{0.651304in}{1.853594in}}%
\pgfpathlineto{\pgfqpoint{2.272067in}{1.853594in}}%
\pgfusepath{stroke}%
\end{pgfscope}%
\begin{pgfscope}%
\pgfsetbuttcap%
\pgfsetroundjoin%
\definecolor{currentfill}{rgb}{0.000000,0.000000,0.000000}%
\pgfsetfillcolor{currentfill}%
\pgfsetlinewidth{0.501875pt}%
\definecolor{currentstroke}{rgb}{0.000000,0.000000,0.000000}%
\pgfsetstrokecolor{currentstroke}%
\pgfsetdash{}{0pt}%
\pgfsys@defobject{currentmarker}{\pgfqpoint{0.000000in}{0.000000in}}{\pgfqpoint{0.055556in}{0.000000in}}{%
\pgfpathmoveto{\pgfqpoint{0.000000in}{0.000000in}}%
\pgfpathlineto{\pgfqpoint{0.055556in}{0.000000in}}%
\pgfusepath{stroke,fill}%
}%
\begin{pgfscope}%
\pgfsys@transformshift{0.651304in}{1.853594in}%
\pgfsys@useobject{currentmarker}{}%
\end{pgfscope}%
\end{pgfscope}%
\begin{pgfscope}%
\pgfsetbuttcap%
\pgfsetroundjoin%
\definecolor{currentfill}{rgb}{0.000000,0.000000,0.000000}%
\pgfsetfillcolor{currentfill}%
\pgfsetlinewidth{0.501875pt}%
\definecolor{currentstroke}{rgb}{0.000000,0.000000,0.000000}%
\pgfsetstrokecolor{currentstroke}%
\pgfsetdash{}{0pt}%
\pgfsys@defobject{currentmarker}{\pgfqpoint{-0.055556in}{0.000000in}}{\pgfqpoint{0.000000in}{0.000000in}}{%
\pgfpathmoveto{\pgfqpoint{0.000000in}{0.000000in}}%
\pgfpathlineto{\pgfqpoint{-0.055556in}{0.000000in}}%
\pgfusepath{stroke,fill}%
}%
\begin{pgfscope}%
\pgfsys@transformshift{2.272067in}{1.853594in}%
\pgfsys@useobject{currentmarker}{}%
\end{pgfscope}%
\end{pgfscope}%
\begin{pgfscope}%
\pgftext[x=0.595748in,y=1.853594in,right,]{\fontsize{8.000000}{9.600000}\selectfont \(\displaystyle {10^{-20}}\)}%
\end{pgfscope}%
\begin{pgfscope}%
\pgfpathrectangle{\pgfqpoint{0.651304in}{0.523750in}}{\pgfqpoint{1.620763in}{1.773125in}} %
\pgfusepath{clip}%
\pgfsetrectcap%
\pgfsetroundjoin%
\pgfsetlinewidth{0.501875pt}%
\definecolor{currentstroke}{rgb}{0.800000,0.800000,0.800000}%
\pgfsetstrokecolor{currentstroke}%
\pgfsetdash{}{0pt}%
\pgfpathmoveto{\pgfqpoint{0.651304in}{2.296875in}}%
\pgfpathlineto{\pgfqpoint{2.272067in}{2.296875in}}%
\pgfusepath{stroke}%
\end{pgfscope}%
\begin{pgfscope}%
\pgfsetbuttcap%
\pgfsetroundjoin%
\definecolor{currentfill}{rgb}{0.000000,0.000000,0.000000}%
\pgfsetfillcolor{currentfill}%
\pgfsetlinewidth{0.501875pt}%
\definecolor{currentstroke}{rgb}{0.000000,0.000000,0.000000}%
\pgfsetstrokecolor{currentstroke}%
\pgfsetdash{}{0pt}%
\pgfsys@defobject{currentmarker}{\pgfqpoint{0.000000in}{0.000000in}}{\pgfqpoint{0.055556in}{0.000000in}}{%
\pgfpathmoveto{\pgfqpoint{0.000000in}{0.000000in}}%
\pgfpathlineto{\pgfqpoint{0.055556in}{0.000000in}}%
\pgfusepath{stroke,fill}%
}%
\begin{pgfscope}%
\pgfsys@transformshift{0.651304in}{2.296875in}%
\pgfsys@useobject{currentmarker}{}%
\end{pgfscope}%
\end{pgfscope}%
\begin{pgfscope}%
\pgfsetbuttcap%
\pgfsetroundjoin%
\definecolor{currentfill}{rgb}{0.000000,0.000000,0.000000}%
\pgfsetfillcolor{currentfill}%
\pgfsetlinewidth{0.501875pt}%
\definecolor{currentstroke}{rgb}{0.000000,0.000000,0.000000}%
\pgfsetstrokecolor{currentstroke}%
\pgfsetdash{}{0pt}%
\pgfsys@defobject{currentmarker}{\pgfqpoint{-0.055556in}{0.000000in}}{\pgfqpoint{0.000000in}{0.000000in}}{%
\pgfpathmoveto{\pgfqpoint{0.000000in}{0.000000in}}%
\pgfpathlineto{\pgfqpoint{-0.055556in}{0.000000in}}%
\pgfusepath{stroke,fill}%
}%
\begin{pgfscope}%
\pgfsys@transformshift{2.272067in}{2.296875in}%
\pgfsys@useobject{currentmarker}{}%
\end{pgfscope}%
\end{pgfscope}%
\begin{pgfscope}%
\pgftext[x=0.595748in,y=2.296875in,right,]{\fontsize{8.000000}{9.600000}\selectfont \(\displaystyle {10^{-19}}\)}%
\end{pgfscope}%
\begin{pgfscope}%
\pgfsetbuttcap%
\pgfsetroundjoin%
\definecolor{currentfill}{rgb}{0.000000,0.000000,0.000000}%
\pgfsetfillcolor{currentfill}%
\pgfsetlinewidth{0.501875pt}%
\definecolor{currentstroke}{rgb}{0.000000,0.000000,0.000000}%
\pgfsetstrokecolor{currentstroke}%
\pgfsetdash{}{0pt}%
\pgfsys@defobject{currentmarker}{\pgfqpoint{0.000000in}{0.000000in}}{\pgfqpoint{0.027778in}{0.000000in}}{%
\pgfpathmoveto{\pgfqpoint{0.000000in}{0.000000in}}%
\pgfpathlineto{\pgfqpoint{0.027778in}{0.000000in}}%
\pgfusepath{stroke,fill}%
}%
\begin{pgfscope}%
\pgfsys@transformshift{0.651304in}{0.657191in}%
\pgfsys@useobject{currentmarker}{}%
\end{pgfscope}%
\end{pgfscope}%
\begin{pgfscope}%
\pgfsetbuttcap%
\pgfsetroundjoin%
\definecolor{currentfill}{rgb}{0.000000,0.000000,0.000000}%
\pgfsetfillcolor{currentfill}%
\pgfsetlinewidth{0.501875pt}%
\definecolor{currentstroke}{rgb}{0.000000,0.000000,0.000000}%
\pgfsetstrokecolor{currentstroke}%
\pgfsetdash{}{0pt}%
\pgfsys@defobject{currentmarker}{\pgfqpoint{-0.027778in}{0.000000in}}{\pgfqpoint{0.000000in}{0.000000in}}{%
\pgfpathmoveto{\pgfqpoint{0.000000in}{0.000000in}}%
\pgfpathlineto{\pgfqpoint{-0.027778in}{0.000000in}}%
\pgfusepath{stroke,fill}%
}%
\begin{pgfscope}%
\pgfsys@transformshift{2.272067in}{0.657191in}%
\pgfsys@useobject{currentmarker}{}%
\end{pgfscope}%
\end{pgfscope}%
\begin{pgfscope}%
\pgfsetbuttcap%
\pgfsetroundjoin%
\definecolor{currentfill}{rgb}{0.000000,0.000000,0.000000}%
\pgfsetfillcolor{currentfill}%
\pgfsetlinewidth{0.501875pt}%
\definecolor{currentstroke}{rgb}{0.000000,0.000000,0.000000}%
\pgfsetstrokecolor{currentstroke}%
\pgfsetdash{}{0pt}%
\pgfsys@defobject{currentmarker}{\pgfqpoint{0.000000in}{0.000000in}}{\pgfqpoint{0.027778in}{0.000000in}}{%
\pgfpathmoveto{\pgfqpoint{0.000000in}{0.000000in}}%
\pgfpathlineto{\pgfqpoint{0.027778in}{0.000000in}}%
\pgfusepath{stroke,fill}%
}%
\begin{pgfscope}%
\pgfsys@transformshift{0.651304in}{0.735249in}%
\pgfsys@useobject{currentmarker}{}%
\end{pgfscope}%
\end{pgfscope}%
\begin{pgfscope}%
\pgfsetbuttcap%
\pgfsetroundjoin%
\definecolor{currentfill}{rgb}{0.000000,0.000000,0.000000}%
\pgfsetfillcolor{currentfill}%
\pgfsetlinewidth{0.501875pt}%
\definecolor{currentstroke}{rgb}{0.000000,0.000000,0.000000}%
\pgfsetstrokecolor{currentstroke}%
\pgfsetdash{}{0pt}%
\pgfsys@defobject{currentmarker}{\pgfqpoint{-0.027778in}{0.000000in}}{\pgfqpoint{0.000000in}{0.000000in}}{%
\pgfpathmoveto{\pgfqpoint{0.000000in}{0.000000in}}%
\pgfpathlineto{\pgfqpoint{-0.027778in}{0.000000in}}%
\pgfusepath{stroke,fill}%
}%
\begin{pgfscope}%
\pgfsys@transformshift{2.272067in}{0.735249in}%
\pgfsys@useobject{currentmarker}{}%
\end{pgfscope}%
\end{pgfscope}%
\begin{pgfscope}%
\pgfsetbuttcap%
\pgfsetroundjoin%
\definecolor{currentfill}{rgb}{0.000000,0.000000,0.000000}%
\pgfsetfillcolor{currentfill}%
\pgfsetlinewidth{0.501875pt}%
\definecolor{currentstroke}{rgb}{0.000000,0.000000,0.000000}%
\pgfsetstrokecolor{currentstroke}%
\pgfsetdash{}{0pt}%
\pgfsys@defobject{currentmarker}{\pgfqpoint{0.000000in}{0.000000in}}{\pgfqpoint{0.027778in}{0.000000in}}{%
\pgfpathmoveto{\pgfqpoint{0.000000in}{0.000000in}}%
\pgfpathlineto{\pgfqpoint{0.027778in}{0.000000in}}%
\pgfusepath{stroke,fill}%
}%
\begin{pgfscope}%
\pgfsys@transformshift{0.651304in}{0.790632in}%
\pgfsys@useobject{currentmarker}{}%
\end{pgfscope}%
\end{pgfscope}%
\begin{pgfscope}%
\pgfsetbuttcap%
\pgfsetroundjoin%
\definecolor{currentfill}{rgb}{0.000000,0.000000,0.000000}%
\pgfsetfillcolor{currentfill}%
\pgfsetlinewidth{0.501875pt}%
\definecolor{currentstroke}{rgb}{0.000000,0.000000,0.000000}%
\pgfsetstrokecolor{currentstroke}%
\pgfsetdash{}{0pt}%
\pgfsys@defobject{currentmarker}{\pgfqpoint{-0.027778in}{0.000000in}}{\pgfqpoint{0.000000in}{0.000000in}}{%
\pgfpathmoveto{\pgfqpoint{0.000000in}{0.000000in}}%
\pgfpathlineto{\pgfqpoint{-0.027778in}{0.000000in}}%
\pgfusepath{stroke,fill}%
}%
\begin{pgfscope}%
\pgfsys@transformshift{2.272067in}{0.790632in}%
\pgfsys@useobject{currentmarker}{}%
\end{pgfscope}%
\end{pgfscope}%
\begin{pgfscope}%
\pgfsetbuttcap%
\pgfsetroundjoin%
\definecolor{currentfill}{rgb}{0.000000,0.000000,0.000000}%
\pgfsetfillcolor{currentfill}%
\pgfsetlinewidth{0.501875pt}%
\definecolor{currentstroke}{rgb}{0.000000,0.000000,0.000000}%
\pgfsetstrokecolor{currentstroke}%
\pgfsetdash{}{0pt}%
\pgfsys@defobject{currentmarker}{\pgfqpoint{0.000000in}{0.000000in}}{\pgfqpoint{0.027778in}{0.000000in}}{%
\pgfpathmoveto{\pgfqpoint{0.000000in}{0.000000in}}%
\pgfpathlineto{\pgfqpoint{0.027778in}{0.000000in}}%
\pgfusepath{stroke,fill}%
}%
\begin{pgfscope}%
\pgfsys@transformshift{0.651304in}{0.833590in}%
\pgfsys@useobject{currentmarker}{}%
\end{pgfscope}%
\end{pgfscope}%
\begin{pgfscope}%
\pgfsetbuttcap%
\pgfsetroundjoin%
\definecolor{currentfill}{rgb}{0.000000,0.000000,0.000000}%
\pgfsetfillcolor{currentfill}%
\pgfsetlinewidth{0.501875pt}%
\definecolor{currentstroke}{rgb}{0.000000,0.000000,0.000000}%
\pgfsetstrokecolor{currentstroke}%
\pgfsetdash{}{0pt}%
\pgfsys@defobject{currentmarker}{\pgfqpoint{-0.027778in}{0.000000in}}{\pgfqpoint{0.000000in}{0.000000in}}{%
\pgfpathmoveto{\pgfqpoint{0.000000in}{0.000000in}}%
\pgfpathlineto{\pgfqpoint{-0.027778in}{0.000000in}}%
\pgfusepath{stroke,fill}%
}%
\begin{pgfscope}%
\pgfsys@transformshift{2.272067in}{0.833590in}%
\pgfsys@useobject{currentmarker}{}%
\end{pgfscope}%
\end{pgfscope}%
\begin{pgfscope}%
\pgfsetbuttcap%
\pgfsetroundjoin%
\definecolor{currentfill}{rgb}{0.000000,0.000000,0.000000}%
\pgfsetfillcolor{currentfill}%
\pgfsetlinewidth{0.501875pt}%
\definecolor{currentstroke}{rgb}{0.000000,0.000000,0.000000}%
\pgfsetstrokecolor{currentstroke}%
\pgfsetdash{}{0pt}%
\pgfsys@defobject{currentmarker}{\pgfqpoint{0.000000in}{0.000000in}}{\pgfqpoint{0.027778in}{0.000000in}}{%
\pgfpathmoveto{\pgfqpoint{0.000000in}{0.000000in}}%
\pgfpathlineto{\pgfqpoint{0.027778in}{0.000000in}}%
\pgfusepath{stroke,fill}%
}%
\begin{pgfscope}%
\pgfsys@transformshift{0.651304in}{0.868690in}%
\pgfsys@useobject{currentmarker}{}%
\end{pgfscope}%
\end{pgfscope}%
\begin{pgfscope}%
\pgfsetbuttcap%
\pgfsetroundjoin%
\definecolor{currentfill}{rgb}{0.000000,0.000000,0.000000}%
\pgfsetfillcolor{currentfill}%
\pgfsetlinewidth{0.501875pt}%
\definecolor{currentstroke}{rgb}{0.000000,0.000000,0.000000}%
\pgfsetstrokecolor{currentstroke}%
\pgfsetdash{}{0pt}%
\pgfsys@defobject{currentmarker}{\pgfqpoint{-0.027778in}{0.000000in}}{\pgfqpoint{0.000000in}{0.000000in}}{%
\pgfpathmoveto{\pgfqpoint{0.000000in}{0.000000in}}%
\pgfpathlineto{\pgfqpoint{-0.027778in}{0.000000in}}%
\pgfusepath{stroke,fill}%
}%
\begin{pgfscope}%
\pgfsys@transformshift{2.272067in}{0.868690in}%
\pgfsys@useobject{currentmarker}{}%
\end{pgfscope}%
\end{pgfscope}%
\begin{pgfscope}%
\pgfsetbuttcap%
\pgfsetroundjoin%
\definecolor{currentfill}{rgb}{0.000000,0.000000,0.000000}%
\pgfsetfillcolor{currentfill}%
\pgfsetlinewidth{0.501875pt}%
\definecolor{currentstroke}{rgb}{0.000000,0.000000,0.000000}%
\pgfsetstrokecolor{currentstroke}%
\pgfsetdash{}{0pt}%
\pgfsys@defobject{currentmarker}{\pgfqpoint{0.000000in}{0.000000in}}{\pgfqpoint{0.027778in}{0.000000in}}{%
\pgfpathmoveto{\pgfqpoint{0.000000in}{0.000000in}}%
\pgfpathlineto{\pgfqpoint{0.027778in}{0.000000in}}%
\pgfusepath{stroke,fill}%
}%
\begin{pgfscope}%
\pgfsys@transformshift{0.651304in}{0.898366in}%
\pgfsys@useobject{currentmarker}{}%
\end{pgfscope}%
\end{pgfscope}%
\begin{pgfscope}%
\pgfsetbuttcap%
\pgfsetroundjoin%
\definecolor{currentfill}{rgb}{0.000000,0.000000,0.000000}%
\pgfsetfillcolor{currentfill}%
\pgfsetlinewidth{0.501875pt}%
\definecolor{currentstroke}{rgb}{0.000000,0.000000,0.000000}%
\pgfsetstrokecolor{currentstroke}%
\pgfsetdash{}{0pt}%
\pgfsys@defobject{currentmarker}{\pgfqpoint{-0.027778in}{0.000000in}}{\pgfqpoint{0.000000in}{0.000000in}}{%
\pgfpathmoveto{\pgfqpoint{0.000000in}{0.000000in}}%
\pgfpathlineto{\pgfqpoint{-0.027778in}{0.000000in}}%
\pgfusepath{stroke,fill}%
}%
\begin{pgfscope}%
\pgfsys@transformshift{2.272067in}{0.898366in}%
\pgfsys@useobject{currentmarker}{}%
\end{pgfscope}%
\end{pgfscope}%
\begin{pgfscope}%
\pgfsetbuttcap%
\pgfsetroundjoin%
\definecolor{currentfill}{rgb}{0.000000,0.000000,0.000000}%
\pgfsetfillcolor{currentfill}%
\pgfsetlinewidth{0.501875pt}%
\definecolor{currentstroke}{rgb}{0.000000,0.000000,0.000000}%
\pgfsetstrokecolor{currentstroke}%
\pgfsetdash{}{0pt}%
\pgfsys@defobject{currentmarker}{\pgfqpoint{0.000000in}{0.000000in}}{\pgfqpoint{0.027778in}{0.000000in}}{%
\pgfpathmoveto{\pgfqpoint{0.000000in}{0.000000in}}%
\pgfpathlineto{\pgfqpoint{0.027778in}{0.000000in}}%
\pgfusepath{stroke,fill}%
}%
\begin{pgfscope}%
\pgfsys@transformshift{0.651304in}{0.924073in}%
\pgfsys@useobject{currentmarker}{}%
\end{pgfscope}%
\end{pgfscope}%
\begin{pgfscope}%
\pgfsetbuttcap%
\pgfsetroundjoin%
\definecolor{currentfill}{rgb}{0.000000,0.000000,0.000000}%
\pgfsetfillcolor{currentfill}%
\pgfsetlinewidth{0.501875pt}%
\definecolor{currentstroke}{rgb}{0.000000,0.000000,0.000000}%
\pgfsetstrokecolor{currentstroke}%
\pgfsetdash{}{0pt}%
\pgfsys@defobject{currentmarker}{\pgfqpoint{-0.027778in}{0.000000in}}{\pgfqpoint{0.000000in}{0.000000in}}{%
\pgfpathmoveto{\pgfqpoint{0.000000in}{0.000000in}}%
\pgfpathlineto{\pgfqpoint{-0.027778in}{0.000000in}}%
\pgfusepath{stroke,fill}%
}%
\begin{pgfscope}%
\pgfsys@transformshift{2.272067in}{0.924073in}%
\pgfsys@useobject{currentmarker}{}%
\end{pgfscope}%
\end{pgfscope}%
\begin{pgfscope}%
\pgfsetbuttcap%
\pgfsetroundjoin%
\definecolor{currentfill}{rgb}{0.000000,0.000000,0.000000}%
\pgfsetfillcolor{currentfill}%
\pgfsetlinewidth{0.501875pt}%
\definecolor{currentstroke}{rgb}{0.000000,0.000000,0.000000}%
\pgfsetstrokecolor{currentstroke}%
\pgfsetdash{}{0pt}%
\pgfsys@defobject{currentmarker}{\pgfqpoint{0.000000in}{0.000000in}}{\pgfqpoint{0.027778in}{0.000000in}}{%
\pgfpathmoveto{\pgfqpoint{0.000000in}{0.000000in}}%
\pgfpathlineto{\pgfqpoint{0.027778in}{0.000000in}}%
\pgfusepath{stroke,fill}%
}%
\begin{pgfscope}%
\pgfsys@transformshift{0.651304in}{0.946748in}%
\pgfsys@useobject{currentmarker}{}%
\end{pgfscope}%
\end{pgfscope}%
\begin{pgfscope}%
\pgfsetbuttcap%
\pgfsetroundjoin%
\definecolor{currentfill}{rgb}{0.000000,0.000000,0.000000}%
\pgfsetfillcolor{currentfill}%
\pgfsetlinewidth{0.501875pt}%
\definecolor{currentstroke}{rgb}{0.000000,0.000000,0.000000}%
\pgfsetstrokecolor{currentstroke}%
\pgfsetdash{}{0pt}%
\pgfsys@defobject{currentmarker}{\pgfqpoint{-0.027778in}{0.000000in}}{\pgfqpoint{0.000000in}{0.000000in}}{%
\pgfpathmoveto{\pgfqpoint{0.000000in}{0.000000in}}%
\pgfpathlineto{\pgfqpoint{-0.027778in}{0.000000in}}%
\pgfusepath{stroke,fill}%
}%
\begin{pgfscope}%
\pgfsys@transformshift{2.272067in}{0.946748in}%
\pgfsys@useobject{currentmarker}{}%
\end{pgfscope}%
\end{pgfscope}%
\begin{pgfscope}%
\pgfsetbuttcap%
\pgfsetroundjoin%
\definecolor{currentfill}{rgb}{0.000000,0.000000,0.000000}%
\pgfsetfillcolor{currentfill}%
\pgfsetlinewidth{0.501875pt}%
\definecolor{currentstroke}{rgb}{0.000000,0.000000,0.000000}%
\pgfsetstrokecolor{currentstroke}%
\pgfsetdash{}{0pt}%
\pgfsys@defobject{currentmarker}{\pgfqpoint{0.000000in}{0.000000in}}{\pgfqpoint{0.027778in}{0.000000in}}{%
\pgfpathmoveto{\pgfqpoint{0.000000in}{0.000000in}}%
\pgfpathlineto{\pgfqpoint{0.027778in}{0.000000in}}%
\pgfusepath{stroke,fill}%
}%
\begin{pgfscope}%
\pgfsys@transformshift{0.651304in}{1.100472in}%
\pgfsys@useobject{currentmarker}{}%
\end{pgfscope}%
\end{pgfscope}%
\begin{pgfscope}%
\pgfsetbuttcap%
\pgfsetroundjoin%
\definecolor{currentfill}{rgb}{0.000000,0.000000,0.000000}%
\pgfsetfillcolor{currentfill}%
\pgfsetlinewidth{0.501875pt}%
\definecolor{currentstroke}{rgb}{0.000000,0.000000,0.000000}%
\pgfsetstrokecolor{currentstroke}%
\pgfsetdash{}{0pt}%
\pgfsys@defobject{currentmarker}{\pgfqpoint{-0.027778in}{0.000000in}}{\pgfqpoint{0.000000in}{0.000000in}}{%
\pgfpathmoveto{\pgfqpoint{0.000000in}{0.000000in}}%
\pgfpathlineto{\pgfqpoint{-0.027778in}{0.000000in}}%
\pgfusepath{stroke,fill}%
}%
\begin{pgfscope}%
\pgfsys@transformshift{2.272067in}{1.100472in}%
\pgfsys@useobject{currentmarker}{}%
\end{pgfscope}%
\end{pgfscope}%
\begin{pgfscope}%
\pgfsetbuttcap%
\pgfsetroundjoin%
\definecolor{currentfill}{rgb}{0.000000,0.000000,0.000000}%
\pgfsetfillcolor{currentfill}%
\pgfsetlinewidth{0.501875pt}%
\definecolor{currentstroke}{rgb}{0.000000,0.000000,0.000000}%
\pgfsetstrokecolor{currentstroke}%
\pgfsetdash{}{0pt}%
\pgfsys@defobject{currentmarker}{\pgfqpoint{0.000000in}{0.000000in}}{\pgfqpoint{0.027778in}{0.000000in}}{%
\pgfpathmoveto{\pgfqpoint{0.000000in}{0.000000in}}%
\pgfpathlineto{\pgfqpoint{0.027778in}{0.000000in}}%
\pgfusepath{stroke,fill}%
}%
\begin{pgfscope}%
\pgfsys@transformshift{0.651304in}{1.178530in}%
\pgfsys@useobject{currentmarker}{}%
\end{pgfscope}%
\end{pgfscope}%
\begin{pgfscope}%
\pgfsetbuttcap%
\pgfsetroundjoin%
\definecolor{currentfill}{rgb}{0.000000,0.000000,0.000000}%
\pgfsetfillcolor{currentfill}%
\pgfsetlinewidth{0.501875pt}%
\definecolor{currentstroke}{rgb}{0.000000,0.000000,0.000000}%
\pgfsetstrokecolor{currentstroke}%
\pgfsetdash{}{0pt}%
\pgfsys@defobject{currentmarker}{\pgfqpoint{-0.027778in}{0.000000in}}{\pgfqpoint{0.000000in}{0.000000in}}{%
\pgfpathmoveto{\pgfqpoint{0.000000in}{0.000000in}}%
\pgfpathlineto{\pgfqpoint{-0.027778in}{0.000000in}}%
\pgfusepath{stroke,fill}%
}%
\begin{pgfscope}%
\pgfsys@transformshift{2.272067in}{1.178530in}%
\pgfsys@useobject{currentmarker}{}%
\end{pgfscope}%
\end{pgfscope}%
\begin{pgfscope}%
\pgfsetbuttcap%
\pgfsetroundjoin%
\definecolor{currentfill}{rgb}{0.000000,0.000000,0.000000}%
\pgfsetfillcolor{currentfill}%
\pgfsetlinewidth{0.501875pt}%
\definecolor{currentstroke}{rgb}{0.000000,0.000000,0.000000}%
\pgfsetstrokecolor{currentstroke}%
\pgfsetdash{}{0pt}%
\pgfsys@defobject{currentmarker}{\pgfqpoint{0.000000in}{0.000000in}}{\pgfqpoint{0.027778in}{0.000000in}}{%
\pgfpathmoveto{\pgfqpoint{0.000000in}{0.000000in}}%
\pgfpathlineto{\pgfqpoint{0.027778in}{0.000000in}}%
\pgfusepath{stroke,fill}%
}%
\begin{pgfscope}%
\pgfsys@transformshift{0.651304in}{1.233913in}%
\pgfsys@useobject{currentmarker}{}%
\end{pgfscope}%
\end{pgfscope}%
\begin{pgfscope}%
\pgfsetbuttcap%
\pgfsetroundjoin%
\definecolor{currentfill}{rgb}{0.000000,0.000000,0.000000}%
\pgfsetfillcolor{currentfill}%
\pgfsetlinewidth{0.501875pt}%
\definecolor{currentstroke}{rgb}{0.000000,0.000000,0.000000}%
\pgfsetstrokecolor{currentstroke}%
\pgfsetdash{}{0pt}%
\pgfsys@defobject{currentmarker}{\pgfqpoint{-0.027778in}{0.000000in}}{\pgfqpoint{0.000000in}{0.000000in}}{%
\pgfpathmoveto{\pgfqpoint{0.000000in}{0.000000in}}%
\pgfpathlineto{\pgfqpoint{-0.027778in}{0.000000in}}%
\pgfusepath{stroke,fill}%
}%
\begin{pgfscope}%
\pgfsys@transformshift{2.272067in}{1.233913in}%
\pgfsys@useobject{currentmarker}{}%
\end{pgfscope}%
\end{pgfscope}%
\begin{pgfscope}%
\pgfsetbuttcap%
\pgfsetroundjoin%
\definecolor{currentfill}{rgb}{0.000000,0.000000,0.000000}%
\pgfsetfillcolor{currentfill}%
\pgfsetlinewidth{0.501875pt}%
\definecolor{currentstroke}{rgb}{0.000000,0.000000,0.000000}%
\pgfsetstrokecolor{currentstroke}%
\pgfsetdash{}{0pt}%
\pgfsys@defobject{currentmarker}{\pgfqpoint{0.000000in}{0.000000in}}{\pgfqpoint{0.027778in}{0.000000in}}{%
\pgfpathmoveto{\pgfqpoint{0.000000in}{0.000000in}}%
\pgfpathlineto{\pgfqpoint{0.027778in}{0.000000in}}%
\pgfusepath{stroke,fill}%
}%
\begin{pgfscope}%
\pgfsys@transformshift{0.651304in}{1.276872in}%
\pgfsys@useobject{currentmarker}{}%
\end{pgfscope}%
\end{pgfscope}%
\begin{pgfscope}%
\pgfsetbuttcap%
\pgfsetroundjoin%
\definecolor{currentfill}{rgb}{0.000000,0.000000,0.000000}%
\pgfsetfillcolor{currentfill}%
\pgfsetlinewidth{0.501875pt}%
\definecolor{currentstroke}{rgb}{0.000000,0.000000,0.000000}%
\pgfsetstrokecolor{currentstroke}%
\pgfsetdash{}{0pt}%
\pgfsys@defobject{currentmarker}{\pgfqpoint{-0.027778in}{0.000000in}}{\pgfqpoint{0.000000in}{0.000000in}}{%
\pgfpathmoveto{\pgfqpoint{0.000000in}{0.000000in}}%
\pgfpathlineto{\pgfqpoint{-0.027778in}{0.000000in}}%
\pgfusepath{stroke,fill}%
}%
\begin{pgfscope}%
\pgfsys@transformshift{2.272067in}{1.276872in}%
\pgfsys@useobject{currentmarker}{}%
\end{pgfscope}%
\end{pgfscope}%
\begin{pgfscope}%
\pgfsetbuttcap%
\pgfsetroundjoin%
\definecolor{currentfill}{rgb}{0.000000,0.000000,0.000000}%
\pgfsetfillcolor{currentfill}%
\pgfsetlinewidth{0.501875pt}%
\definecolor{currentstroke}{rgb}{0.000000,0.000000,0.000000}%
\pgfsetstrokecolor{currentstroke}%
\pgfsetdash{}{0pt}%
\pgfsys@defobject{currentmarker}{\pgfqpoint{0.000000in}{0.000000in}}{\pgfqpoint{0.027778in}{0.000000in}}{%
\pgfpathmoveto{\pgfqpoint{0.000000in}{0.000000in}}%
\pgfpathlineto{\pgfqpoint{0.027778in}{0.000000in}}%
\pgfusepath{stroke,fill}%
}%
\begin{pgfscope}%
\pgfsys@transformshift{0.651304in}{1.311971in}%
\pgfsys@useobject{currentmarker}{}%
\end{pgfscope}%
\end{pgfscope}%
\begin{pgfscope}%
\pgfsetbuttcap%
\pgfsetroundjoin%
\definecolor{currentfill}{rgb}{0.000000,0.000000,0.000000}%
\pgfsetfillcolor{currentfill}%
\pgfsetlinewidth{0.501875pt}%
\definecolor{currentstroke}{rgb}{0.000000,0.000000,0.000000}%
\pgfsetstrokecolor{currentstroke}%
\pgfsetdash{}{0pt}%
\pgfsys@defobject{currentmarker}{\pgfqpoint{-0.027778in}{0.000000in}}{\pgfqpoint{0.000000in}{0.000000in}}{%
\pgfpathmoveto{\pgfqpoint{0.000000in}{0.000000in}}%
\pgfpathlineto{\pgfqpoint{-0.027778in}{0.000000in}}%
\pgfusepath{stroke,fill}%
}%
\begin{pgfscope}%
\pgfsys@transformshift{2.272067in}{1.311971in}%
\pgfsys@useobject{currentmarker}{}%
\end{pgfscope}%
\end{pgfscope}%
\begin{pgfscope}%
\pgfsetbuttcap%
\pgfsetroundjoin%
\definecolor{currentfill}{rgb}{0.000000,0.000000,0.000000}%
\pgfsetfillcolor{currentfill}%
\pgfsetlinewidth{0.501875pt}%
\definecolor{currentstroke}{rgb}{0.000000,0.000000,0.000000}%
\pgfsetstrokecolor{currentstroke}%
\pgfsetdash{}{0pt}%
\pgfsys@defobject{currentmarker}{\pgfqpoint{0.000000in}{0.000000in}}{\pgfqpoint{0.027778in}{0.000000in}}{%
\pgfpathmoveto{\pgfqpoint{0.000000in}{0.000000in}}%
\pgfpathlineto{\pgfqpoint{0.027778in}{0.000000in}}%
\pgfusepath{stroke,fill}%
}%
\begin{pgfscope}%
\pgfsys@transformshift{0.651304in}{1.341647in}%
\pgfsys@useobject{currentmarker}{}%
\end{pgfscope}%
\end{pgfscope}%
\begin{pgfscope}%
\pgfsetbuttcap%
\pgfsetroundjoin%
\definecolor{currentfill}{rgb}{0.000000,0.000000,0.000000}%
\pgfsetfillcolor{currentfill}%
\pgfsetlinewidth{0.501875pt}%
\definecolor{currentstroke}{rgb}{0.000000,0.000000,0.000000}%
\pgfsetstrokecolor{currentstroke}%
\pgfsetdash{}{0pt}%
\pgfsys@defobject{currentmarker}{\pgfqpoint{-0.027778in}{0.000000in}}{\pgfqpoint{0.000000in}{0.000000in}}{%
\pgfpathmoveto{\pgfqpoint{0.000000in}{0.000000in}}%
\pgfpathlineto{\pgfqpoint{-0.027778in}{0.000000in}}%
\pgfusepath{stroke,fill}%
}%
\begin{pgfscope}%
\pgfsys@transformshift{2.272067in}{1.341647in}%
\pgfsys@useobject{currentmarker}{}%
\end{pgfscope}%
\end{pgfscope}%
\begin{pgfscope}%
\pgfsetbuttcap%
\pgfsetroundjoin%
\definecolor{currentfill}{rgb}{0.000000,0.000000,0.000000}%
\pgfsetfillcolor{currentfill}%
\pgfsetlinewidth{0.501875pt}%
\definecolor{currentstroke}{rgb}{0.000000,0.000000,0.000000}%
\pgfsetstrokecolor{currentstroke}%
\pgfsetdash{}{0pt}%
\pgfsys@defobject{currentmarker}{\pgfqpoint{0.000000in}{0.000000in}}{\pgfqpoint{0.027778in}{0.000000in}}{%
\pgfpathmoveto{\pgfqpoint{0.000000in}{0.000000in}}%
\pgfpathlineto{\pgfqpoint{0.027778in}{0.000000in}}%
\pgfusepath{stroke,fill}%
}%
\begin{pgfscope}%
\pgfsys@transformshift{0.651304in}{1.367354in}%
\pgfsys@useobject{currentmarker}{}%
\end{pgfscope}%
\end{pgfscope}%
\begin{pgfscope}%
\pgfsetbuttcap%
\pgfsetroundjoin%
\definecolor{currentfill}{rgb}{0.000000,0.000000,0.000000}%
\pgfsetfillcolor{currentfill}%
\pgfsetlinewidth{0.501875pt}%
\definecolor{currentstroke}{rgb}{0.000000,0.000000,0.000000}%
\pgfsetstrokecolor{currentstroke}%
\pgfsetdash{}{0pt}%
\pgfsys@defobject{currentmarker}{\pgfqpoint{-0.027778in}{0.000000in}}{\pgfqpoint{0.000000in}{0.000000in}}{%
\pgfpathmoveto{\pgfqpoint{0.000000in}{0.000000in}}%
\pgfpathlineto{\pgfqpoint{-0.027778in}{0.000000in}}%
\pgfusepath{stroke,fill}%
}%
\begin{pgfscope}%
\pgfsys@transformshift{2.272067in}{1.367354in}%
\pgfsys@useobject{currentmarker}{}%
\end{pgfscope}%
\end{pgfscope}%
\begin{pgfscope}%
\pgfsetbuttcap%
\pgfsetroundjoin%
\definecolor{currentfill}{rgb}{0.000000,0.000000,0.000000}%
\pgfsetfillcolor{currentfill}%
\pgfsetlinewidth{0.501875pt}%
\definecolor{currentstroke}{rgb}{0.000000,0.000000,0.000000}%
\pgfsetstrokecolor{currentstroke}%
\pgfsetdash{}{0pt}%
\pgfsys@defobject{currentmarker}{\pgfqpoint{0.000000in}{0.000000in}}{\pgfqpoint{0.027778in}{0.000000in}}{%
\pgfpathmoveto{\pgfqpoint{0.000000in}{0.000000in}}%
\pgfpathlineto{\pgfqpoint{0.027778in}{0.000000in}}%
\pgfusepath{stroke,fill}%
}%
\begin{pgfscope}%
\pgfsys@transformshift{0.651304in}{1.390029in}%
\pgfsys@useobject{currentmarker}{}%
\end{pgfscope}%
\end{pgfscope}%
\begin{pgfscope}%
\pgfsetbuttcap%
\pgfsetroundjoin%
\definecolor{currentfill}{rgb}{0.000000,0.000000,0.000000}%
\pgfsetfillcolor{currentfill}%
\pgfsetlinewidth{0.501875pt}%
\definecolor{currentstroke}{rgb}{0.000000,0.000000,0.000000}%
\pgfsetstrokecolor{currentstroke}%
\pgfsetdash{}{0pt}%
\pgfsys@defobject{currentmarker}{\pgfqpoint{-0.027778in}{0.000000in}}{\pgfqpoint{0.000000in}{0.000000in}}{%
\pgfpathmoveto{\pgfqpoint{0.000000in}{0.000000in}}%
\pgfpathlineto{\pgfqpoint{-0.027778in}{0.000000in}}%
\pgfusepath{stroke,fill}%
}%
\begin{pgfscope}%
\pgfsys@transformshift{2.272067in}{1.390029in}%
\pgfsys@useobject{currentmarker}{}%
\end{pgfscope}%
\end{pgfscope}%
\begin{pgfscope}%
\pgfsetbuttcap%
\pgfsetroundjoin%
\definecolor{currentfill}{rgb}{0.000000,0.000000,0.000000}%
\pgfsetfillcolor{currentfill}%
\pgfsetlinewidth{0.501875pt}%
\definecolor{currentstroke}{rgb}{0.000000,0.000000,0.000000}%
\pgfsetstrokecolor{currentstroke}%
\pgfsetdash{}{0pt}%
\pgfsys@defobject{currentmarker}{\pgfqpoint{0.000000in}{0.000000in}}{\pgfqpoint{0.027778in}{0.000000in}}{%
\pgfpathmoveto{\pgfqpoint{0.000000in}{0.000000in}}%
\pgfpathlineto{\pgfqpoint{0.027778in}{0.000000in}}%
\pgfusepath{stroke,fill}%
}%
\begin{pgfscope}%
\pgfsys@transformshift{0.651304in}{1.543753in}%
\pgfsys@useobject{currentmarker}{}%
\end{pgfscope}%
\end{pgfscope}%
\begin{pgfscope}%
\pgfsetbuttcap%
\pgfsetroundjoin%
\definecolor{currentfill}{rgb}{0.000000,0.000000,0.000000}%
\pgfsetfillcolor{currentfill}%
\pgfsetlinewidth{0.501875pt}%
\definecolor{currentstroke}{rgb}{0.000000,0.000000,0.000000}%
\pgfsetstrokecolor{currentstroke}%
\pgfsetdash{}{0pt}%
\pgfsys@defobject{currentmarker}{\pgfqpoint{-0.027778in}{0.000000in}}{\pgfqpoint{0.000000in}{0.000000in}}{%
\pgfpathmoveto{\pgfqpoint{0.000000in}{0.000000in}}%
\pgfpathlineto{\pgfqpoint{-0.027778in}{0.000000in}}%
\pgfusepath{stroke,fill}%
}%
\begin{pgfscope}%
\pgfsys@transformshift{2.272067in}{1.543753in}%
\pgfsys@useobject{currentmarker}{}%
\end{pgfscope}%
\end{pgfscope}%
\begin{pgfscope}%
\pgfsetbuttcap%
\pgfsetroundjoin%
\definecolor{currentfill}{rgb}{0.000000,0.000000,0.000000}%
\pgfsetfillcolor{currentfill}%
\pgfsetlinewidth{0.501875pt}%
\definecolor{currentstroke}{rgb}{0.000000,0.000000,0.000000}%
\pgfsetstrokecolor{currentstroke}%
\pgfsetdash{}{0pt}%
\pgfsys@defobject{currentmarker}{\pgfqpoint{0.000000in}{0.000000in}}{\pgfqpoint{0.027778in}{0.000000in}}{%
\pgfpathmoveto{\pgfqpoint{0.000000in}{0.000000in}}%
\pgfpathlineto{\pgfqpoint{0.027778in}{0.000000in}}%
\pgfusepath{stroke,fill}%
}%
\begin{pgfscope}%
\pgfsys@transformshift{0.651304in}{1.621811in}%
\pgfsys@useobject{currentmarker}{}%
\end{pgfscope}%
\end{pgfscope}%
\begin{pgfscope}%
\pgfsetbuttcap%
\pgfsetroundjoin%
\definecolor{currentfill}{rgb}{0.000000,0.000000,0.000000}%
\pgfsetfillcolor{currentfill}%
\pgfsetlinewidth{0.501875pt}%
\definecolor{currentstroke}{rgb}{0.000000,0.000000,0.000000}%
\pgfsetstrokecolor{currentstroke}%
\pgfsetdash{}{0pt}%
\pgfsys@defobject{currentmarker}{\pgfqpoint{-0.027778in}{0.000000in}}{\pgfqpoint{0.000000in}{0.000000in}}{%
\pgfpathmoveto{\pgfqpoint{0.000000in}{0.000000in}}%
\pgfpathlineto{\pgfqpoint{-0.027778in}{0.000000in}}%
\pgfusepath{stroke,fill}%
}%
\begin{pgfscope}%
\pgfsys@transformshift{2.272067in}{1.621811in}%
\pgfsys@useobject{currentmarker}{}%
\end{pgfscope}%
\end{pgfscope}%
\begin{pgfscope}%
\pgfsetbuttcap%
\pgfsetroundjoin%
\definecolor{currentfill}{rgb}{0.000000,0.000000,0.000000}%
\pgfsetfillcolor{currentfill}%
\pgfsetlinewidth{0.501875pt}%
\definecolor{currentstroke}{rgb}{0.000000,0.000000,0.000000}%
\pgfsetstrokecolor{currentstroke}%
\pgfsetdash{}{0pt}%
\pgfsys@defobject{currentmarker}{\pgfqpoint{0.000000in}{0.000000in}}{\pgfqpoint{0.027778in}{0.000000in}}{%
\pgfpathmoveto{\pgfqpoint{0.000000in}{0.000000in}}%
\pgfpathlineto{\pgfqpoint{0.027778in}{0.000000in}}%
\pgfusepath{stroke,fill}%
}%
\begin{pgfscope}%
\pgfsys@transformshift{0.651304in}{1.677194in}%
\pgfsys@useobject{currentmarker}{}%
\end{pgfscope}%
\end{pgfscope}%
\begin{pgfscope}%
\pgfsetbuttcap%
\pgfsetroundjoin%
\definecolor{currentfill}{rgb}{0.000000,0.000000,0.000000}%
\pgfsetfillcolor{currentfill}%
\pgfsetlinewidth{0.501875pt}%
\definecolor{currentstroke}{rgb}{0.000000,0.000000,0.000000}%
\pgfsetstrokecolor{currentstroke}%
\pgfsetdash{}{0pt}%
\pgfsys@defobject{currentmarker}{\pgfqpoint{-0.027778in}{0.000000in}}{\pgfqpoint{0.000000in}{0.000000in}}{%
\pgfpathmoveto{\pgfqpoint{0.000000in}{0.000000in}}%
\pgfpathlineto{\pgfqpoint{-0.027778in}{0.000000in}}%
\pgfusepath{stroke,fill}%
}%
\begin{pgfscope}%
\pgfsys@transformshift{2.272067in}{1.677194in}%
\pgfsys@useobject{currentmarker}{}%
\end{pgfscope}%
\end{pgfscope}%
\begin{pgfscope}%
\pgfsetbuttcap%
\pgfsetroundjoin%
\definecolor{currentfill}{rgb}{0.000000,0.000000,0.000000}%
\pgfsetfillcolor{currentfill}%
\pgfsetlinewidth{0.501875pt}%
\definecolor{currentstroke}{rgb}{0.000000,0.000000,0.000000}%
\pgfsetstrokecolor{currentstroke}%
\pgfsetdash{}{0pt}%
\pgfsys@defobject{currentmarker}{\pgfqpoint{0.000000in}{0.000000in}}{\pgfqpoint{0.027778in}{0.000000in}}{%
\pgfpathmoveto{\pgfqpoint{0.000000in}{0.000000in}}%
\pgfpathlineto{\pgfqpoint{0.027778in}{0.000000in}}%
\pgfusepath{stroke,fill}%
}%
\begin{pgfscope}%
\pgfsys@transformshift{0.651304in}{1.720153in}%
\pgfsys@useobject{currentmarker}{}%
\end{pgfscope}%
\end{pgfscope}%
\begin{pgfscope}%
\pgfsetbuttcap%
\pgfsetroundjoin%
\definecolor{currentfill}{rgb}{0.000000,0.000000,0.000000}%
\pgfsetfillcolor{currentfill}%
\pgfsetlinewidth{0.501875pt}%
\definecolor{currentstroke}{rgb}{0.000000,0.000000,0.000000}%
\pgfsetstrokecolor{currentstroke}%
\pgfsetdash{}{0pt}%
\pgfsys@defobject{currentmarker}{\pgfqpoint{-0.027778in}{0.000000in}}{\pgfqpoint{0.000000in}{0.000000in}}{%
\pgfpathmoveto{\pgfqpoint{0.000000in}{0.000000in}}%
\pgfpathlineto{\pgfqpoint{-0.027778in}{0.000000in}}%
\pgfusepath{stroke,fill}%
}%
\begin{pgfscope}%
\pgfsys@transformshift{2.272067in}{1.720153in}%
\pgfsys@useobject{currentmarker}{}%
\end{pgfscope}%
\end{pgfscope}%
\begin{pgfscope}%
\pgfsetbuttcap%
\pgfsetroundjoin%
\definecolor{currentfill}{rgb}{0.000000,0.000000,0.000000}%
\pgfsetfillcolor{currentfill}%
\pgfsetlinewidth{0.501875pt}%
\definecolor{currentstroke}{rgb}{0.000000,0.000000,0.000000}%
\pgfsetstrokecolor{currentstroke}%
\pgfsetdash{}{0pt}%
\pgfsys@defobject{currentmarker}{\pgfqpoint{0.000000in}{0.000000in}}{\pgfqpoint{0.027778in}{0.000000in}}{%
\pgfpathmoveto{\pgfqpoint{0.000000in}{0.000000in}}%
\pgfpathlineto{\pgfqpoint{0.027778in}{0.000000in}}%
\pgfusepath{stroke,fill}%
}%
\begin{pgfscope}%
\pgfsys@transformshift{0.651304in}{1.755252in}%
\pgfsys@useobject{currentmarker}{}%
\end{pgfscope}%
\end{pgfscope}%
\begin{pgfscope}%
\pgfsetbuttcap%
\pgfsetroundjoin%
\definecolor{currentfill}{rgb}{0.000000,0.000000,0.000000}%
\pgfsetfillcolor{currentfill}%
\pgfsetlinewidth{0.501875pt}%
\definecolor{currentstroke}{rgb}{0.000000,0.000000,0.000000}%
\pgfsetstrokecolor{currentstroke}%
\pgfsetdash{}{0pt}%
\pgfsys@defobject{currentmarker}{\pgfqpoint{-0.027778in}{0.000000in}}{\pgfqpoint{0.000000in}{0.000000in}}{%
\pgfpathmoveto{\pgfqpoint{0.000000in}{0.000000in}}%
\pgfpathlineto{\pgfqpoint{-0.027778in}{0.000000in}}%
\pgfusepath{stroke,fill}%
}%
\begin{pgfscope}%
\pgfsys@transformshift{2.272067in}{1.755252in}%
\pgfsys@useobject{currentmarker}{}%
\end{pgfscope}%
\end{pgfscope}%
\begin{pgfscope}%
\pgfsetbuttcap%
\pgfsetroundjoin%
\definecolor{currentfill}{rgb}{0.000000,0.000000,0.000000}%
\pgfsetfillcolor{currentfill}%
\pgfsetlinewidth{0.501875pt}%
\definecolor{currentstroke}{rgb}{0.000000,0.000000,0.000000}%
\pgfsetstrokecolor{currentstroke}%
\pgfsetdash{}{0pt}%
\pgfsys@defobject{currentmarker}{\pgfqpoint{0.000000in}{0.000000in}}{\pgfqpoint{0.027778in}{0.000000in}}{%
\pgfpathmoveto{\pgfqpoint{0.000000in}{0.000000in}}%
\pgfpathlineto{\pgfqpoint{0.027778in}{0.000000in}}%
\pgfusepath{stroke,fill}%
}%
\begin{pgfscope}%
\pgfsys@transformshift{0.651304in}{1.784929in}%
\pgfsys@useobject{currentmarker}{}%
\end{pgfscope}%
\end{pgfscope}%
\begin{pgfscope}%
\pgfsetbuttcap%
\pgfsetroundjoin%
\definecolor{currentfill}{rgb}{0.000000,0.000000,0.000000}%
\pgfsetfillcolor{currentfill}%
\pgfsetlinewidth{0.501875pt}%
\definecolor{currentstroke}{rgb}{0.000000,0.000000,0.000000}%
\pgfsetstrokecolor{currentstroke}%
\pgfsetdash{}{0pt}%
\pgfsys@defobject{currentmarker}{\pgfqpoint{-0.027778in}{0.000000in}}{\pgfqpoint{0.000000in}{0.000000in}}{%
\pgfpathmoveto{\pgfqpoint{0.000000in}{0.000000in}}%
\pgfpathlineto{\pgfqpoint{-0.027778in}{0.000000in}}%
\pgfusepath{stroke,fill}%
}%
\begin{pgfscope}%
\pgfsys@transformshift{2.272067in}{1.784929in}%
\pgfsys@useobject{currentmarker}{}%
\end{pgfscope}%
\end{pgfscope}%
\begin{pgfscope}%
\pgfsetbuttcap%
\pgfsetroundjoin%
\definecolor{currentfill}{rgb}{0.000000,0.000000,0.000000}%
\pgfsetfillcolor{currentfill}%
\pgfsetlinewidth{0.501875pt}%
\definecolor{currentstroke}{rgb}{0.000000,0.000000,0.000000}%
\pgfsetstrokecolor{currentstroke}%
\pgfsetdash{}{0pt}%
\pgfsys@defobject{currentmarker}{\pgfqpoint{0.000000in}{0.000000in}}{\pgfqpoint{0.027778in}{0.000000in}}{%
\pgfpathmoveto{\pgfqpoint{0.000000in}{0.000000in}}%
\pgfpathlineto{\pgfqpoint{0.027778in}{0.000000in}}%
\pgfusepath{stroke,fill}%
}%
\begin{pgfscope}%
\pgfsys@transformshift{0.651304in}{1.810635in}%
\pgfsys@useobject{currentmarker}{}%
\end{pgfscope}%
\end{pgfscope}%
\begin{pgfscope}%
\pgfsetbuttcap%
\pgfsetroundjoin%
\definecolor{currentfill}{rgb}{0.000000,0.000000,0.000000}%
\pgfsetfillcolor{currentfill}%
\pgfsetlinewidth{0.501875pt}%
\definecolor{currentstroke}{rgb}{0.000000,0.000000,0.000000}%
\pgfsetstrokecolor{currentstroke}%
\pgfsetdash{}{0pt}%
\pgfsys@defobject{currentmarker}{\pgfqpoint{-0.027778in}{0.000000in}}{\pgfqpoint{0.000000in}{0.000000in}}{%
\pgfpathmoveto{\pgfqpoint{0.000000in}{0.000000in}}%
\pgfpathlineto{\pgfqpoint{-0.027778in}{0.000000in}}%
\pgfusepath{stroke,fill}%
}%
\begin{pgfscope}%
\pgfsys@transformshift{2.272067in}{1.810635in}%
\pgfsys@useobject{currentmarker}{}%
\end{pgfscope}%
\end{pgfscope}%
\begin{pgfscope}%
\pgfsetbuttcap%
\pgfsetroundjoin%
\definecolor{currentfill}{rgb}{0.000000,0.000000,0.000000}%
\pgfsetfillcolor{currentfill}%
\pgfsetlinewidth{0.501875pt}%
\definecolor{currentstroke}{rgb}{0.000000,0.000000,0.000000}%
\pgfsetstrokecolor{currentstroke}%
\pgfsetdash{}{0pt}%
\pgfsys@defobject{currentmarker}{\pgfqpoint{0.000000in}{0.000000in}}{\pgfqpoint{0.027778in}{0.000000in}}{%
\pgfpathmoveto{\pgfqpoint{0.000000in}{0.000000in}}%
\pgfpathlineto{\pgfqpoint{0.027778in}{0.000000in}}%
\pgfusepath{stroke,fill}%
}%
\begin{pgfscope}%
\pgfsys@transformshift{0.651304in}{1.833310in}%
\pgfsys@useobject{currentmarker}{}%
\end{pgfscope}%
\end{pgfscope}%
\begin{pgfscope}%
\pgfsetbuttcap%
\pgfsetroundjoin%
\definecolor{currentfill}{rgb}{0.000000,0.000000,0.000000}%
\pgfsetfillcolor{currentfill}%
\pgfsetlinewidth{0.501875pt}%
\definecolor{currentstroke}{rgb}{0.000000,0.000000,0.000000}%
\pgfsetstrokecolor{currentstroke}%
\pgfsetdash{}{0pt}%
\pgfsys@defobject{currentmarker}{\pgfqpoint{-0.027778in}{0.000000in}}{\pgfqpoint{0.000000in}{0.000000in}}{%
\pgfpathmoveto{\pgfqpoint{0.000000in}{0.000000in}}%
\pgfpathlineto{\pgfqpoint{-0.027778in}{0.000000in}}%
\pgfusepath{stroke,fill}%
}%
\begin{pgfscope}%
\pgfsys@transformshift{2.272067in}{1.833310in}%
\pgfsys@useobject{currentmarker}{}%
\end{pgfscope}%
\end{pgfscope}%
\begin{pgfscope}%
\pgfsetbuttcap%
\pgfsetroundjoin%
\definecolor{currentfill}{rgb}{0.000000,0.000000,0.000000}%
\pgfsetfillcolor{currentfill}%
\pgfsetlinewidth{0.501875pt}%
\definecolor{currentstroke}{rgb}{0.000000,0.000000,0.000000}%
\pgfsetstrokecolor{currentstroke}%
\pgfsetdash{}{0pt}%
\pgfsys@defobject{currentmarker}{\pgfqpoint{0.000000in}{0.000000in}}{\pgfqpoint{0.027778in}{0.000000in}}{%
\pgfpathmoveto{\pgfqpoint{0.000000in}{0.000000in}}%
\pgfpathlineto{\pgfqpoint{0.027778in}{0.000000in}}%
\pgfusepath{stroke,fill}%
}%
\begin{pgfscope}%
\pgfsys@transformshift{0.651304in}{1.987035in}%
\pgfsys@useobject{currentmarker}{}%
\end{pgfscope}%
\end{pgfscope}%
\begin{pgfscope}%
\pgfsetbuttcap%
\pgfsetroundjoin%
\definecolor{currentfill}{rgb}{0.000000,0.000000,0.000000}%
\pgfsetfillcolor{currentfill}%
\pgfsetlinewidth{0.501875pt}%
\definecolor{currentstroke}{rgb}{0.000000,0.000000,0.000000}%
\pgfsetstrokecolor{currentstroke}%
\pgfsetdash{}{0pt}%
\pgfsys@defobject{currentmarker}{\pgfqpoint{-0.027778in}{0.000000in}}{\pgfqpoint{0.000000in}{0.000000in}}{%
\pgfpathmoveto{\pgfqpoint{0.000000in}{0.000000in}}%
\pgfpathlineto{\pgfqpoint{-0.027778in}{0.000000in}}%
\pgfusepath{stroke,fill}%
}%
\begin{pgfscope}%
\pgfsys@transformshift{2.272067in}{1.987035in}%
\pgfsys@useobject{currentmarker}{}%
\end{pgfscope}%
\end{pgfscope}%
\begin{pgfscope}%
\pgfsetbuttcap%
\pgfsetroundjoin%
\definecolor{currentfill}{rgb}{0.000000,0.000000,0.000000}%
\pgfsetfillcolor{currentfill}%
\pgfsetlinewidth{0.501875pt}%
\definecolor{currentstroke}{rgb}{0.000000,0.000000,0.000000}%
\pgfsetstrokecolor{currentstroke}%
\pgfsetdash{}{0pt}%
\pgfsys@defobject{currentmarker}{\pgfqpoint{0.000000in}{0.000000in}}{\pgfqpoint{0.027778in}{0.000000in}}{%
\pgfpathmoveto{\pgfqpoint{0.000000in}{0.000000in}}%
\pgfpathlineto{\pgfqpoint{0.027778in}{0.000000in}}%
\pgfusepath{stroke,fill}%
}%
\begin{pgfscope}%
\pgfsys@transformshift{0.651304in}{2.065093in}%
\pgfsys@useobject{currentmarker}{}%
\end{pgfscope}%
\end{pgfscope}%
\begin{pgfscope}%
\pgfsetbuttcap%
\pgfsetroundjoin%
\definecolor{currentfill}{rgb}{0.000000,0.000000,0.000000}%
\pgfsetfillcolor{currentfill}%
\pgfsetlinewidth{0.501875pt}%
\definecolor{currentstroke}{rgb}{0.000000,0.000000,0.000000}%
\pgfsetstrokecolor{currentstroke}%
\pgfsetdash{}{0pt}%
\pgfsys@defobject{currentmarker}{\pgfqpoint{-0.027778in}{0.000000in}}{\pgfqpoint{0.000000in}{0.000000in}}{%
\pgfpathmoveto{\pgfqpoint{0.000000in}{0.000000in}}%
\pgfpathlineto{\pgfqpoint{-0.027778in}{0.000000in}}%
\pgfusepath{stroke,fill}%
}%
\begin{pgfscope}%
\pgfsys@transformshift{2.272067in}{2.065093in}%
\pgfsys@useobject{currentmarker}{}%
\end{pgfscope}%
\end{pgfscope}%
\begin{pgfscope}%
\pgfsetbuttcap%
\pgfsetroundjoin%
\definecolor{currentfill}{rgb}{0.000000,0.000000,0.000000}%
\pgfsetfillcolor{currentfill}%
\pgfsetlinewidth{0.501875pt}%
\definecolor{currentstroke}{rgb}{0.000000,0.000000,0.000000}%
\pgfsetstrokecolor{currentstroke}%
\pgfsetdash{}{0pt}%
\pgfsys@defobject{currentmarker}{\pgfqpoint{0.000000in}{0.000000in}}{\pgfqpoint{0.027778in}{0.000000in}}{%
\pgfpathmoveto{\pgfqpoint{0.000000in}{0.000000in}}%
\pgfpathlineto{\pgfqpoint{0.027778in}{0.000000in}}%
\pgfusepath{stroke,fill}%
}%
\begin{pgfscope}%
\pgfsys@transformshift{0.651304in}{2.120476in}%
\pgfsys@useobject{currentmarker}{}%
\end{pgfscope}%
\end{pgfscope}%
\begin{pgfscope}%
\pgfsetbuttcap%
\pgfsetroundjoin%
\definecolor{currentfill}{rgb}{0.000000,0.000000,0.000000}%
\pgfsetfillcolor{currentfill}%
\pgfsetlinewidth{0.501875pt}%
\definecolor{currentstroke}{rgb}{0.000000,0.000000,0.000000}%
\pgfsetstrokecolor{currentstroke}%
\pgfsetdash{}{0pt}%
\pgfsys@defobject{currentmarker}{\pgfqpoint{-0.027778in}{0.000000in}}{\pgfqpoint{0.000000in}{0.000000in}}{%
\pgfpathmoveto{\pgfqpoint{0.000000in}{0.000000in}}%
\pgfpathlineto{\pgfqpoint{-0.027778in}{0.000000in}}%
\pgfusepath{stroke,fill}%
}%
\begin{pgfscope}%
\pgfsys@transformshift{2.272067in}{2.120476in}%
\pgfsys@useobject{currentmarker}{}%
\end{pgfscope}%
\end{pgfscope}%
\begin{pgfscope}%
\pgfsetbuttcap%
\pgfsetroundjoin%
\definecolor{currentfill}{rgb}{0.000000,0.000000,0.000000}%
\pgfsetfillcolor{currentfill}%
\pgfsetlinewidth{0.501875pt}%
\definecolor{currentstroke}{rgb}{0.000000,0.000000,0.000000}%
\pgfsetstrokecolor{currentstroke}%
\pgfsetdash{}{0pt}%
\pgfsys@defobject{currentmarker}{\pgfqpoint{0.000000in}{0.000000in}}{\pgfqpoint{0.027778in}{0.000000in}}{%
\pgfpathmoveto{\pgfqpoint{0.000000in}{0.000000in}}%
\pgfpathlineto{\pgfqpoint{0.027778in}{0.000000in}}%
\pgfusepath{stroke,fill}%
}%
\begin{pgfscope}%
\pgfsys@transformshift{0.651304in}{2.163434in}%
\pgfsys@useobject{currentmarker}{}%
\end{pgfscope}%
\end{pgfscope}%
\begin{pgfscope}%
\pgfsetbuttcap%
\pgfsetroundjoin%
\definecolor{currentfill}{rgb}{0.000000,0.000000,0.000000}%
\pgfsetfillcolor{currentfill}%
\pgfsetlinewidth{0.501875pt}%
\definecolor{currentstroke}{rgb}{0.000000,0.000000,0.000000}%
\pgfsetstrokecolor{currentstroke}%
\pgfsetdash{}{0pt}%
\pgfsys@defobject{currentmarker}{\pgfqpoint{-0.027778in}{0.000000in}}{\pgfqpoint{0.000000in}{0.000000in}}{%
\pgfpathmoveto{\pgfqpoint{0.000000in}{0.000000in}}%
\pgfpathlineto{\pgfqpoint{-0.027778in}{0.000000in}}%
\pgfusepath{stroke,fill}%
}%
\begin{pgfscope}%
\pgfsys@transformshift{2.272067in}{2.163434in}%
\pgfsys@useobject{currentmarker}{}%
\end{pgfscope}%
\end{pgfscope}%
\begin{pgfscope}%
\pgfsetbuttcap%
\pgfsetroundjoin%
\definecolor{currentfill}{rgb}{0.000000,0.000000,0.000000}%
\pgfsetfillcolor{currentfill}%
\pgfsetlinewidth{0.501875pt}%
\definecolor{currentstroke}{rgb}{0.000000,0.000000,0.000000}%
\pgfsetstrokecolor{currentstroke}%
\pgfsetdash{}{0pt}%
\pgfsys@defobject{currentmarker}{\pgfqpoint{0.000000in}{0.000000in}}{\pgfqpoint{0.027778in}{0.000000in}}{%
\pgfpathmoveto{\pgfqpoint{0.000000in}{0.000000in}}%
\pgfpathlineto{\pgfqpoint{0.027778in}{0.000000in}}%
\pgfusepath{stroke,fill}%
}%
\begin{pgfscope}%
\pgfsys@transformshift{0.651304in}{2.198534in}%
\pgfsys@useobject{currentmarker}{}%
\end{pgfscope}%
\end{pgfscope}%
\begin{pgfscope}%
\pgfsetbuttcap%
\pgfsetroundjoin%
\definecolor{currentfill}{rgb}{0.000000,0.000000,0.000000}%
\pgfsetfillcolor{currentfill}%
\pgfsetlinewidth{0.501875pt}%
\definecolor{currentstroke}{rgb}{0.000000,0.000000,0.000000}%
\pgfsetstrokecolor{currentstroke}%
\pgfsetdash{}{0pt}%
\pgfsys@defobject{currentmarker}{\pgfqpoint{-0.027778in}{0.000000in}}{\pgfqpoint{0.000000in}{0.000000in}}{%
\pgfpathmoveto{\pgfqpoint{0.000000in}{0.000000in}}%
\pgfpathlineto{\pgfqpoint{-0.027778in}{0.000000in}}%
\pgfusepath{stroke,fill}%
}%
\begin{pgfscope}%
\pgfsys@transformshift{2.272067in}{2.198534in}%
\pgfsys@useobject{currentmarker}{}%
\end{pgfscope}%
\end{pgfscope}%
\begin{pgfscope}%
\pgfsetbuttcap%
\pgfsetroundjoin%
\definecolor{currentfill}{rgb}{0.000000,0.000000,0.000000}%
\pgfsetfillcolor{currentfill}%
\pgfsetlinewidth{0.501875pt}%
\definecolor{currentstroke}{rgb}{0.000000,0.000000,0.000000}%
\pgfsetstrokecolor{currentstroke}%
\pgfsetdash{}{0pt}%
\pgfsys@defobject{currentmarker}{\pgfqpoint{0.000000in}{0.000000in}}{\pgfqpoint{0.027778in}{0.000000in}}{%
\pgfpathmoveto{\pgfqpoint{0.000000in}{0.000000in}}%
\pgfpathlineto{\pgfqpoint{0.027778in}{0.000000in}}%
\pgfusepath{stroke,fill}%
}%
\begin{pgfscope}%
\pgfsys@transformshift{0.651304in}{2.228210in}%
\pgfsys@useobject{currentmarker}{}%
\end{pgfscope}%
\end{pgfscope}%
\begin{pgfscope}%
\pgfsetbuttcap%
\pgfsetroundjoin%
\definecolor{currentfill}{rgb}{0.000000,0.000000,0.000000}%
\pgfsetfillcolor{currentfill}%
\pgfsetlinewidth{0.501875pt}%
\definecolor{currentstroke}{rgb}{0.000000,0.000000,0.000000}%
\pgfsetstrokecolor{currentstroke}%
\pgfsetdash{}{0pt}%
\pgfsys@defobject{currentmarker}{\pgfqpoint{-0.027778in}{0.000000in}}{\pgfqpoint{0.000000in}{0.000000in}}{%
\pgfpathmoveto{\pgfqpoint{0.000000in}{0.000000in}}%
\pgfpathlineto{\pgfqpoint{-0.027778in}{0.000000in}}%
\pgfusepath{stroke,fill}%
}%
\begin{pgfscope}%
\pgfsys@transformshift{2.272067in}{2.228210in}%
\pgfsys@useobject{currentmarker}{}%
\end{pgfscope}%
\end{pgfscope}%
\begin{pgfscope}%
\pgfsetbuttcap%
\pgfsetroundjoin%
\definecolor{currentfill}{rgb}{0.000000,0.000000,0.000000}%
\pgfsetfillcolor{currentfill}%
\pgfsetlinewidth{0.501875pt}%
\definecolor{currentstroke}{rgb}{0.000000,0.000000,0.000000}%
\pgfsetstrokecolor{currentstroke}%
\pgfsetdash{}{0pt}%
\pgfsys@defobject{currentmarker}{\pgfqpoint{0.000000in}{0.000000in}}{\pgfqpoint{0.027778in}{0.000000in}}{%
\pgfpathmoveto{\pgfqpoint{0.000000in}{0.000000in}}%
\pgfpathlineto{\pgfqpoint{0.027778in}{0.000000in}}%
\pgfusepath{stroke,fill}%
}%
\begin{pgfscope}%
\pgfsys@transformshift{0.651304in}{2.253917in}%
\pgfsys@useobject{currentmarker}{}%
\end{pgfscope}%
\end{pgfscope}%
\begin{pgfscope}%
\pgfsetbuttcap%
\pgfsetroundjoin%
\definecolor{currentfill}{rgb}{0.000000,0.000000,0.000000}%
\pgfsetfillcolor{currentfill}%
\pgfsetlinewidth{0.501875pt}%
\definecolor{currentstroke}{rgb}{0.000000,0.000000,0.000000}%
\pgfsetstrokecolor{currentstroke}%
\pgfsetdash{}{0pt}%
\pgfsys@defobject{currentmarker}{\pgfqpoint{-0.027778in}{0.000000in}}{\pgfqpoint{0.000000in}{0.000000in}}{%
\pgfpathmoveto{\pgfqpoint{0.000000in}{0.000000in}}%
\pgfpathlineto{\pgfqpoint{-0.027778in}{0.000000in}}%
\pgfusepath{stroke,fill}%
}%
\begin{pgfscope}%
\pgfsys@transformshift{2.272067in}{2.253917in}%
\pgfsys@useobject{currentmarker}{}%
\end{pgfscope}%
\end{pgfscope}%
\begin{pgfscope}%
\pgfsetbuttcap%
\pgfsetroundjoin%
\definecolor{currentfill}{rgb}{0.000000,0.000000,0.000000}%
\pgfsetfillcolor{currentfill}%
\pgfsetlinewidth{0.501875pt}%
\definecolor{currentstroke}{rgb}{0.000000,0.000000,0.000000}%
\pgfsetstrokecolor{currentstroke}%
\pgfsetdash{}{0pt}%
\pgfsys@defobject{currentmarker}{\pgfqpoint{0.000000in}{0.000000in}}{\pgfqpoint{0.027778in}{0.000000in}}{%
\pgfpathmoveto{\pgfqpoint{0.000000in}{0.000000in}}%
\pgfpathlineto{\pgfqpoint{0.027778in}{0.000000in}}%
\pgfusepath{stroke,fill}%
}%
\begin{pgfscope}%
\pgfsys@transformshift{0.651304in}{2.276592in}%
\pgfsys@useobject{currentmarker}{}%
\end{pgfscope}%
\end{pgfscope}%
\begin{pgfscope}%
\pgfsetbuttcap%
\pgfsetroundjoin%
\definecolor{currentfill}{rgb}{0.000000,0.000000,0.000000}%
\pgfsetfillcolor{currentfill}%
\pgfsetlinewidth{0.501875pt}%
\definecolor{currentstroke}{rgb}{0.000000,0.000000,0.000000}%
\pgfsetstrokecolor{currentstroke}%
\pgfsetdash{}{0pt}%
\pgfsys@defobject{currentmarker}{\pgfqpoint{-0.027778in}{0.000000in}}{\pgfqpoint{0.000000in}{0.000000in}}{%
\pgfpathmoveto{\pgfqpoint{0.000000in}{0.000000in}}%
\pgfpathlineto{\pgfqpoint{-0.027778in}{0.000000in}}%
\pgfusepath{stroke,fill}%
}%
\begin{pgfscope}%
\pgfsys@transformshift{2.272067in}{2.276592in}%
\pgfsys@useobject{currentmarker}{}%
\end{pgfscope}%
\end{pgfscope}%
\begin{pgfscope}%
\pgftext[x=0.219206in,y=1.410312in,,bottom,rotate=90.000000]{\fontsize{8.000000}{9.600000}\selectfont Characteristic Strain}%
\end{pgfscope}%
\begin{pgfscope}%
\pgfpathrectangle{\pgfqpoint{0.651304in}{0.523750in}}{\pgfqpoint{1.620763in}{1.773125in}} %
\pgfusepath{clip}%
\pgfsetrectcap%
\pgfsetroundjoin%
\pgfsetlinewidth{2.007500pt}%
\definecolor{currentstroke}{rgb}{0.886275,0.290196,0.200000}%
\pgfsetstrokecolor{currentstroke}%
\pgfsetdash{}{0pt}%
\pgfpathmoveto{\pgfqpoint{0.895415in}{1.091009in}}%
\pgfpathlineto{\pgfqpoint{1.146335in}{0.874591in}}%
\pgfpathlineto{\pgfqpoint{1.205985in}{0.827561in}}%
\pgfpathlineto{\pgfqpoint{1.250074in}{0.796566in}}%
\pgfpathlineto{\pgfqpoint{1.286707in}{0.774453in}}%
\pgfpathlineto{\pgfqpoint{1.319774in}{0.758021in}}%
\pgfpathlineto{\pgfqpoint{1.351220in}{0.745795in}}%
\pgfpathlineto{\pgfqpoint{1.382990in}{0.736758in}}%
\pgfpathlineto{\pgfqpoint{1.417029in}{0.730334in}}%
\pgfpathlineto{\pgfqpoint{1.457228in}{0.726033in}}%
\pgfpathlineto{\pgfqpoint{1.517527in}{0.723111in}}%
\pgfpathlineto{\pgfqpoint{1.610568in}{0.718295in}}%
\pgfpathlineto{\pgfqpoint{1.717549in}{0.711292in}}%
\pgfpathlineto{\pgfqpoint{1.740890in}{0.713827in}}%
\pgfpathlineto{\pgfqpoint{1.760665in}{0.719036in}}%
\pgfpathlineto{\pgfqpoint{1.779144in}{0.727119in}}%
\pgfpathlineto{\pgfqpoint{1.797622in}{0.738638in}}%
\pgfpathlineto{\pgfqpoint{1.817398in}{0.754716in}}%
\pgfpathlineto{\pgfqpoint{1.840415in}{0.777644in}}%
\pgfpathlineto{\pgfqpoint{1.869916in}{0.811737in}}%
\pgfpathlineto{\pgfqpoint{1.922758in}{0.878509in}}%
\pgfpathlineto{\pgfqpoint{1.997645in}{0.971865in}}%
\pgfpathlineto{\pgfqpoint{2.055350in}{1.038650in}}%
\pgfpathlineto{\pgfqpoint{2.115324in}{1.103276in}}%
\pgfpathlineto{\pgfqpoint{2.181782in}{1.170258in}}%
\pgfpathlineto{\pgfqpoint{2.258614in}{1.243197in}}%
\pgfpathlineto{\pgfqpoint{2.285956in}{1.268277in}}%
\pgfpathlineto{\pgfqpoint{2.285956in}{1.268277in}}%
\pgfusepath{stroke}%
\end{pgfscope}%
\begin{pgfscope}%
\pgfpathrectangle{\pgfqpoint{0.651304in}{0.523750in}}{\pgfqpoint{1.620763in}{1.773125in}} %
\pgfusepath{clip}%
\pgfsetrectcap%
\pgfsetroundjoin%
\pgfsetlinewidth{2.007500pt}%
\definecolor{currentstroke}{rgb}{0.203922,0.541176,0.741176}%
\pgfsetstrokecolor{currentstroke}%
\pgfsetdash{}{0pt}%
\pgfpathmoveto{\pgfqpoint{0.637415in}{1.695049in}}%
\pgfpathlineto{\pgfqpoint{2.220962in}{1.189762in}}%
\pgfpathlineto{\pgfqpoint{2.220962in}{1.189762in}}%
\pgfusepath{stroke}%
\end{pgfscope}%
\begin{pgfscope}%
\pgfsetrectcap%
\pgfsetmiterjoin%
\pgfsetlinewidth{1.003750pt}%
\definecolor{currentstroke}{rgb}{0.737255,0.737255,0.737255}%
\pgfsetstrokecolor{currentstroke}%
\pgfsetdash{}{0pt}%
\pgfpathmoveto{\pgfqpoint{0.651304in}{2.296875in}}%
\pgfpathlineto{\pgfqpoint{2.272067in}{2.296875in}}%
\pgfusepath{stroke}%
\end{pgfscope}%
\begin{pgfscope}%
\pgfsetrectcap%
\pgfsetmiterjoin%
\pgfsetlinewidth{1.003750pt}%
\definecolor{currentstroke}{rgb}{0.737255,0.737255,0.737255}%
\pgfsetstrokecolor{currentstroke}%
\pgfsetdash{}{0pt}%
\pgfpathmoveto{\pgfqpoint{2.272067in}{0.523750in}}%
\pgfpathlineto{\pgfqpoint{2.272067in}{2.296875in}}%
\pgfusepath{stroke}%
\end{pgfscope}%
\begin{pgfscope}%
\pgfsetrectcap%
\pgfsetmiterjoin%
\pgfsetlinewidth{1.003750pt}%
\definecolor{currentstroke}{rgb}{0.737255,0.737255,0.737255}%
\pgfsetstrokecolor{currentstroke}%
\pgfsetdash{}{0pt}%
\pgfpathmoveto{\pgfqpoint{0.651304in}{0.523750in}}%
\pgfpathlineto{\pgfqpoint{2.272067in}{0.523750in}}%
\pgfusepath{stroke}%
\end{pgfscope}%
\begin{pgfscope}%
\pgfsetrectcap%
\pgfsetmiterjoin%
\pgfsetlinewidth{1.003750pt}%
\definecolor{currentstroke}{rgb}{0.737255,0.737255,0.737255}%
\pgfsetstrokecolor{currentstroke}%
\pgfsetdash{}{0pt}%
\pgfpathmoveto{\pgfqpoint{0.651304in}{0.523750in}}%
\pgfpathlineto{\pgfqpoint{0.651304in}{2.296875in}}%
\pgfusepath{stroke}%
\end{pgfscope}%
\begin{pgfscope}%
\pgfsetbuttcap%
\pgfsetmiterjoin%
\definecolor{currentfill}{rgb}{1.000000,1.000000,1.000000}%
\pgfsetfillcolor{currentfill}%
\pgfsetlinewidth{0.501875pt}%
\definecolor{currentstroke}{rgb}{0.737255,0.737255,0.737255}%
\pgfsetstrokecolor{currentstroke}%
\pgfsetdash{}{0pt}%
\pgfpathmoveto{\pgfqpoint{1.452742in}{1.881814in}}%
\pgfpathlineto{\pgfqpoint{2.216512in}{1.881814in}}%
\pgfpathlineto{\pgfqpoint{2.216512in}{2.241319in}}%
\pgfpathlineto{\pgfqpoint{1.452742in}{2.241319in}}%
\pgfpathclose%
\pgfusepath{stroke,fill}%
\end{pgfscope}%
\begin{pgfscope}%
\pgfsetrectcap%
\pgfsetroundjoin%
\pgfsetlinewidth{2.007500pt}%
\definecolor{currentstroke}{rgb}{0.886275,0.290196,0.200000}%
\pgfsetstrokecolor{currentstroke}%
\pgfsetdash{}{0pt}%
\pgfpathmoveto{\pgfqpoint{1.497187in}{2.151345in}}%
\pgfpathlineto{\pgfqpoint{1.719409in}{2.151345in}}%
\pgfusepath{stroke}%
\end{pgfscope}%
\begin{pgfscope}%
\pgftext[x=1.808298in,y=2.112457in,left,base]{\fontsize{8.000000}{9.600000}\selectfont aLIGO}%
\end{pgfscope}%
\begin{pgfscope}%
\pgfsetrectcap%
\pgfsetroundjoin%
\pgfsetlinewidth{2.007500pt}%
\definecolor{currentstroke}{rgb}{0.203922,0.541176,0.741176}%
\pgfsetstrokecolor{currentstroke}%
\pgfsetdash{}{0pt}%
\pgfpathmoveto{\pgfqpoint{1.497187in}{1.988260in}}%
\pgfpathlineto{\pgfqpoint{1.719409in}{1.988260in}}%
\pgfusepath{stroke}%
\end{pgfscope}%
\begin{pgfscope}%
\pgftext[x=1.808298in,y=1.949371in,left,base]{\fontsize{8.000000}{9.600000}\selectfont CBC}%
\end{pgfscope}%
\end{pgfpicture}%
\makeatother%
\endgroup%

    \captionof{figure}{The frequency spectrum of a compact binary coalescence, alongside the power spectrum of the advanced LIGO detector.}
  } 

  A pair of orbiting objects can also produce gravitational waves, and
  as they do so their orbit loses energy, causing the radius of the
  orbit to shrink\cite{1995PhRvL..74.3515B}. The objects spiral in
  towards each other. This behaviour has already been observed in the
  Hulse-Taylor
  pulsar\cite{1975ApJ...195L..51H,2005ASPC..328...25W}, a system
  of two neutron stars---one of which is a pulsar--- which, through
  precise pulsar timing measurements, have been inferred to be
  inspiralling, and producing gravitational waves.

  The binary system will continue to lose energy via gravitational
  radiation until it reaches its \emph{innermost stable circular
    orbit}, after which the objects will merge, and eventually
  coalesce. This coalescence will be a powerful source of
  gravitational waves, and the chirp produced by the coalescence may
  be sufficiently luminous to be detected by current, advanced
  detectors, (attempts to detect signals from such coalescing systems
  were also made during the initial run of the
  detectors\cite{2012PhRvD..85h2002A}) and would be characterised by
  the distinct pattern of the pseudo-sinusoidal inspiral waveform,
  followed by a bright burst of radiation, and then a sinusoidal
  ``ringdown'' as the post-coalescence remnant
  vibrates\cite{2009LRR....12....2S}. Binary coalescences are thus
  classified as transient, or burst sources.

The potential objects which may be involved in an observable binary
coalescence are black holes and neutron stars: both compact
objects. White dwarf binaries may also be observable, and these
systems are expected to be much more abundant than either neutron star
or black hole binaries, but their emission lies within the passband of
LISA---a planned space-based gravitational wave observatory. These
never reach a last stable orbit, as it lies within their physical
diameter, and so the inspiral component of the waveform is the
principle source of gravitational waves. These are expected to be so
numerous in LISA results\cite{2013GWN.....6....4A} that entirely new statistical methods will be
needed to process the observations, and to allow observations of other
phenomena to be made in their background.

Supermassive binary black holes (SMBBH) are believed to collide and
merge as part of the merging processes of galaxies. Again these
binaries should produce signals within the passband of LISA\cite{2012CQGra..29l4016A}, and
should be so spectacuarly strong that they are visible in the LISA
data without the \emph{matched filtering} techniques which are
required to extract other signals from the data\cite{2009LRR....12....2S}. The observation of
these objects would provide much-needed information about the
evolution of galaxies and of super-massive black holes.

Inspiralling compact binaries can act as a cosmological distance
measure: they have two parameters, their period, and the rate at which
that period changes (which is calculated by measuring the \emph{chirp
  mass} of the system) which characterise the system, and the
amplitude of the gravitational waves produced is dependent only on the
chirp mass of the source, and the distance from the observer to the
object. As a result it is possible to determine the distance to an
inspiralling system simply by determining the chirp mass and measuring
the brightness of the event. This would provide an additional means of
measuring cosmic acceleration, and, in the LISA era, this would allow
the measurement of acceleration at high redshift using high-mass
binary black holes.


%%% Local Variables: 
%%% mode: latex
%%% TeX-master: "../../document"
%%% End: 


 \section{Supernovae}
 \label{sec:sn}
 
\subsection{Core-collapse supernovae}
\label{sec:core-coll-supern}


Core collapse supernova (CCSNe) are driven by the release of
gravitational energy as a massive star's core collapses. Progenitor
stars of CCSNe have zero-age-main-sequence (ZAMS) masses in the range
$8\,\msolar \leq M \leq 130\,\msolar$. Much of this energy is stored
as heat in the protoneutron star remnant, around 99\% of the released
energy is carried-off by neutrinos, around 1\% provides the kinetic
energy of the explosion, while less than $0.01\%$ of the energy is
extracted as electromagnetic and gravitational radiation
\cite{2009CQGra..26f3001O}.

When the iron core of a star exceeds the Chandrasekhar mass it becomes
unstable, and undegoes gravitational collapse, and is compressed until
the neutron degeneracy pressure is able to halt arrest the
collapse. At this point the core becomes stiff, and the inner core
rebounds---a phase of the supernova known as ``core bounce''. The
stiff, ultra-dense remnant of the collapse is a proto-neutronstar
(PNS).

Gravitational waves are expected to be emitted in a number of periods
during the collapse, for example during a rotating collapse, and the
core-bounce which follows it; pulsations of the PNS
\cite{1966ApJ...145..514M}; and anisotropic neutrino emission
\cite{1979ApJ...231Q.644E,1978ApJ...223.1037E,1978Natur.274..565T}.

In order to predict the gravitational waveforms which would be
produced by a CCSN detailed numerical modelling must be completed,
with the most modern results from
Scheidegger\cite{2010A&A...514A..51S}, modelling rotating,
axisymmetric collapses in three dimensions, and
Dimmelmeier\cite{2008PhRvD..78f4056D} in two dimensions; and
M\"uller\cite{2012A&A...537A..63M} and Ott\cite{2013ApJ...768..115O},
modelling neutrino-driven supernovae in three dimensions.

It is possible that core-collapse supernovae could have been detected
with the initial LIGO detector\cite{2009LRR....12....2S}, although
none were. At design sensitivity the three-detector network of
Advanced LIGO and Advanced VIRGO should be able to detect CCSNe to a
distance of around $\SI{5.5}{\kilo pc}$, in the case of
neutrino-driven explosions, while rapidly-rotating core-collapses will
be detectable to $\SI{50}{\kilo pc}$, the distance to the Large
Magellanic Cloud. Extreme emission scenarios may be detectable as far
as $\SI{0.77}{\mega pc}$, the distance to M31\cite{2016PhRvD..93d2002G}.

\subsection{Type Ia supernovae}
\label{sec:type-ia-supernovae}

Type Ia supernovae (SNe Ia) are believed to be the result of
white-dwarfs in binary systems accreting enough matter to exceed the
Chandrasekhar-mass, and undergoing catastrophic
coore-collapse\cite{2013MNRAS.429.1156S}, however the evolution of the
binary systems which are the progenitors of Type Ia supernovae is
poorly understood. Recent work\cite{2015PhRvD..92l4013S} implies that
the gravitational wave emission from a Type Ia supernova would produce
decihertz gravitational-waves, peaking at a frequency around
\SI{0.4}{\hertz}. This would position SNe Ia as a target for the
proposed DECIGO and BBO space-based observatories.

%%% Local Variables: 
%%% mode: latex
%%% TeX-master: "../../document"
%%% End: 


 \section{Gravitational-wave Pulsars}
 \label{sec:pulsar}
 \input{tex/2-sources/2-3-pulsar}

 \section{Stochastic Sources}
 \label{sec:stochastic-sources}
 
While the gravitational wave detections which we expect to make with
the aLIGO and aVIRGO detectors in the next five years all correspond
to radiation from specific sources, in the LISA era we expect to
encounter a new phenomenon: the gravitational wave
background\cite{2009LRR....12....2S}. This is likely to come from a
number of sources. LISA will be sensitive to compact binary systems
for a much longer period of their evolution, and so rather than just
detecting the final few orbits and the coalescense of objects, we will
be able to observe decades or more of their inspiral. Given the large
population of binary objects in the universe, and the ability of LISA
to detect white dwarf binaries, we can expect the existence of a rich
gravitational wave background at low frequencies: so much so that the
data-handling task will be many orders of magnitude more complicated
than that for the advanced era detectors.  Super-massive binary black
holes throughout the universe will also contribute to this background.

Currently-favoured theories of cosmology favour the existence of a
period of cosmological inflation, a process which should have produced
very low-frequency gravitational waves.  These were the focus of the
ultimately-refuted BICEP2 announcement\cite{2014PhRvL.112x1101B} of
the discovery of gravitational waves.

%%% Local Variables: 
%%% mode: latex
%%% TeX-master: "../../document"
%%% End: 


 \chapter{Data analysis of gravitational wave data}
 \label{cha:data-analys-grav}

 \chapterprecis{Identifying gravitational wave signals in
   interferometer data is not a trivial task, with signals appearing at
   low signal-to-noise ratios.}

%% \section{Bayesian inference}
%% \label{sec:bayesian-inference}
%% \input{tex/3-analysis/bayesian-inference}

\section{Matched filtering}
\label{sec:matched-filtering}
\input{tex/3-analysis/matched-filtering}

\section{Numerical Relativity}
\label{sec:numerical-relativity}
In order to produce templates for matched filtering an understanding
of the shape of the signal which is expected is required. For
gravitational wave events these signals usually have a complicated
form, and are analytically intractable, leading to the requirement
that these be calculated numerically. This has become possible in the
last decade or so, leading to the production of a number of catalogues
of
waveforms\cite{PhysRevLett.106.241101,gatechcat,2016PhRvD..93d4006H}. These
waveforms are computationally expensive to produce, and so ``post-Newtonian
approximants'' to them have been produced, following two separate
approaches, the ``Phenom'' classes of
approximant\cite{2016PhRvD..93d4007K,2007CQGra..24S.689A} and
``effective one-body'' classes\cite{2007PhRvD..76j4049B}.


%%% Local Variables: 
%%% mode: latex
%%% TeX-master: "../../document"
%%% End: 

\part{Data Analysis for Gravitational Wave Detectors}
\label{part:data-analysis}

\chapter{Gaussian processes}
\label{cha:gaussian-process}

\chapterprecis{ Gaussian processes are a powerful machine learning
  tool capable of performing multi-dimensional regression in a
  generalised Bayesian framework, which can be used as priors for further analysis operations.}

\input{tex/3-analysis/gaussian-processes}

% \part{Summary of work thus far}
% \label{part:work}

% \chapter{Presentations \& publications}
% \label{chap:pandp} \newpage


% \section{Work thus far}

% In the first six months of my PhD my work has been focussed primarily
% on the use of Gaussian Processes in gravitational astrophysics, with a
% specific focus on using Gaussian processes to estimate the uncertainty
% in post-Newtonian approximants to numerical relativity waveforms. The
% ultimate aim of this project is to identify the parameters at which
% new numerical relativity waveforms should be produced in an
% algorithmic and Bayesian manner.

% \section{Gaussian processes for gravitational wave detection}
% \label{sec:gauss-proc-grav}

% Gaussian processes have the attractive feature that they can be
% leveraged to produce a regression of data, which has both the
% equivalent of a ``best-fit'' line, which is represented by the mean
% function from the posterior, and an uncertainty, through the variance
% of the posterior. In matched filtering techniques used in LIGO data
% analysis it is often impractical to use numerical relativity (NR)
% waveforms, owing to their computational expense, and so approximants
% must be used to span the regions of parameter space between the
% ``true'' NR waveforms.

% Moore and Gair\cite{2014PhRvL.113y1101M,2016PhRvD..93f4001M} propose a
% method to account for the uncertainty which is introduced through the
% use of an approximant, by training a Gaussian process on the
% differences between NR and approximant waveforms. They then use the
% posterior from the Gaussian process to form the likelihood in their
% analysis. While this approach is shown to improve the efficiency of
% bayesian inference techniques on parameter estimation, it relies on
% the availability of an analytic or a semi-analytic approximant to the
% true values.

% Instead, I have attempted to take the approach of training a Gaussian
% process directly using the numerical relativity waveforms produced by
% the Relativity Group at Georgia Tech\cite{gatechcat}: here the input
% parameters are multi-dimensional, composed of the time, and the
% simulation parameters, and the strain values at each of these points
% composing the targets.

% This approach allows two useful results to be obtained: first, it
% allows the production of gravitational waveforms, with associated
% error estimates, directly from the NR data, bypassing the need for an
% approximant; second, a fully trained Gaussian process, through its
% ability to estimate the uncertainty in any prediction, provides a
% means to indetify regions of parameter space where predictions are
% likely to be particuarly poor, and is thus a means of identifying the
% optimal regions of parameter space to focus future simulation efforts
% on.

% An extra direction has recently appeared which I have actively
% pursued, relating to Gaussian processes. Where previously Gaussian
% processes had been trained to produce a single-dimensional output from
% a multi-dimensional input, I have been attempting to develop a
% technique where a multiple-dimensional input is used to prouduce a
% multiple-dimensional output. The specific intention of this method is
% to provide a timeseries which contains a gravitational wave signal to
% a Gaussian process, and use it to predict the model parameters which
% produced it. In effect, this technique would be a means of performing
% parameter estimation using purely NR data. Perfecting this technique
% is likely to be difficult, however, as it poses formidable problems
% with computability, and will require careful thought about memory
% optimisation. An ultimate aim for such an algorithm would be the
% abiltiy to produce a posterior on multiple waveform parameters
% directly from noisy data from a gravitational wave detector.

% \section{Minke: Producing MDCs for gravitational wave transient analysis}
% \label{sec:mink-prod-mdcs}

% In order to characterise the efficiency of any gravitational wave
% analysis algorithm it must be tested on injected signals---simulated
% signals with known parameters, and hence a known shape---which are
% inserted into either simulated noise, or timeseries containing real
% noise from the detector. By determining the number of these injected
% signals which are detected by an algorithm it is possible to determine
% the detection efficiency of the method; additionally, this method
% allows a controlled method of determining the quietest events the
% algorithm is capable of detecting in the noise environment, and hence
% of calculating the sensitive distance of the detector and analysis to
% a physical system producing a waveform with those parameters.

% An important component of the 


% \section{Burst companion paper}
% \label{sec:burst-comp-paper}

% During the period September 2015 --- April 2016 I contributed to one
% LSC author-list paper, the so-called \emph{Burst Companion Paper}, for
% which I was the figures editor. At present I am not on the general LSC
% author-list, and as such I am not a named contributor on any of the
% other collaboration papers produced over this time period.

% \section{Estimating beaming angles from sGRBs}
% \label{sec:estim-beam-angl}

% During the same period I have been a co-author on a short-author list
% paper, and the lead author on a paper composing the write-up of my
% MSci project.



% \part{Outline of future work}
% \label{part:future}
% \input{tex/5-future}

\bibliographystyle{unsrt}
\bibliography{bibliography/introduction,bibliography/relativity,bibliography/detectors,bibliography/gw150914,bibliography/sources,bibliography/analysis,bibliography/gaussian}

% The glossary
%\newglossaryentry{gravitational wave}{
  name = {gravitational wave},
  description = {A gravitational wave is a propogating perturbation of spacetime.}
}

\newglossaryentry{GW150914}{
  name = {GW150914},
  description = {The first gravitational wave event to be observed. The observation was made by the two \gls{LIGO} detectors in the United States of America, located at Hanford, Washington; and Livingston, Louisiana}
}

\newglossaryentry{VIRGO}{
  name = {VIRGO},
  description = {
    A gravitational wave detector in Italy.
  }
}

\newglossaryentry{LIGO}{
  name = LIGO,
  description = {
    The Laser Interferometer Gravitational-wave Observatory is a
    joint project of California Institute of Technology (CalTech) and
    the Massechusets Institute of Technology (MIT), in which laser
    interferometers of a similar design to the Michelson
    interferometer, famous for its use in disproving the existence of
    the ``luminous aether'', are used to detect small length
    perturbations over distances of 4km. }
}

\newglossaryentry{DetChar}{
  name = detector characterisation,
  description = {DetChar, or detector characterisation, is the process of analysing the noise sources and calibration of the detector, as well as identifying ``glitches'', transient noise events which can interfere with burst searches, and ``lines'', sources of noise which exist in a narrow frequency band which can interfere with long-integration time searches.}
}

\newglossaryentry{burst}{
  name = burst,
  description = {A burst is a short-lived, transient gravitational wave event.}
}

\newglossaryentry{chirp mass}{
  name = {chirp mass, $\mathcal{M}$},
  description = {
    The chirp mass of a binary system is defined as 
    \[ \mathcal{M} = \frac{ (m_1 m_2)^{3/5} }{(m_1 + m_2)^{1/5} } \]
    for $m_1$ and $m_2$ the masses of the two components of the binary,
    and determines the amplitude and the frequency evolution of a
    chirp from a coalescence%
  }}

\newglossaryentry{SNR}{
  name = {signal-to-noise ratio, SNR},
  description = {
    The ratio of signal power to noise power in a given signal. For a
    gravitational wave detector it is defined
    \[ \rho^2 = \int_0^\infty \frac{ 4 \left| h(f) \right|^2
    }{S_n(f)} \dd{f} \] for $h(f)$ the strain of the signal in the frequency
    domain, and $S_n$ is the power spectral density of the detector%
  }
}

\newglossaryentry{characteristic strain}{
  name = characteristic strain,
  description = {
    The characteristic strain is designed to take into account the change of a signal's frequency when it is integrated. It is defined
    \[h^2~c(f) = 4 f^2 \abs{h(f)}^2 \] for $h(f)$ the strain of the
    signal in the frequency domain, and $f$ the frequency%
  } }

\newglossaryentry{finesse}{
  name = finesse,
  description = {
The number of times a beam reflects between ends of a Fabry-Perot interferometer.
  } }

%%% Local Variables: 
%%% mode: latex
%%% TeX-master: "../document"
%%% End: 

%\newglossaryentry{cbc}{
  name = {compact binary coalescence},
  description = {
    The collision and merging of two dense, degenerate stellar
    remnants. The typical remnants considered in CBC searches are
    neutrin stars and black holes, although the range of compact
    objects extends to white dwarfs, and possibly exotic remnants,
    such as quark and boson stars
  }
}

%\glsaddall
%\useglossarystyle{altlist}
\printglossaries

\end{document}
