\documentclass{kentigern}

\usepackage{lipsum}
\usepackage[toc]{glossaries}

\makeglossaries


\title{Gravitational Waves : A Handbook}
\author{Daniel Williams}


\newglossaryentry{gravitational wave}{
  name = {gravitational wave},
  description = {A gravitational wave is a propogating perturbation of spacetime.}
}

\newglossaryentry{GW150914}{
  name = {GW150914},
  description = {The first gravitational wave event to be observed. The observation was made by the two \gls{LIGO} detectors in the United States of America, located at Hanford, Washington; and Livingston, Louisiana}
}

\newglossaryentry{LIGO}{
  name = LIGO,
  description = {
    The Laser Interferometer Gravitational-wave Observatory is a
    joint project of California Institute of Technology (CalTech) and
    the Massechusets Institute of Technology (MIT), in which laser
    interferometers of a similar design to the Michelson
    interferometer, famous for its use in disproving the existence of
    the ``luminous aether'', are used to detect small length
    perturbations over distances of 4km. }
}

\newglossaryentry{DetChar}{
  name = detector characterisation,
  description = {DetChar, or detector characterisation, is the process of analysing the noise sources and calibration of the detector, as well as identifying ``glitches'', transient noise events which can interfere with burst searches, and ``lines'', sources of noise which exist in a narrow frequency band which can interfere with long-integration time searches.}
}

\newglossaryentry{burst}{
  name = burst,
  description = {A burst is a short-lived, transient gravitational wave event.}
}

\newglossaryentry{chirp mass}{
  name = {chirp mass, $\mathcal{M}$},
  description = {What the chirp mass is.}
}


\begin{document}
\maketitle

\newpage

\tableofcontents

\chapter{Gravitational Waves}
\label{cha:grav-waves}

\chapterprecis{\Glspl{gravitational wave} were perhaps the last of the
  predictions of Einstein's General Theory of Relativity to be
  observed; their detection was the catalyst for the beginning of a
  new era of astrophysics}

\epigraph{I Guess we need the detection checklist...}{\textbf{Sergey Klimencko}, \emph{Internal communication}}

The 14 September 2016 will likely be remembered as one of the most
significant in the history of astronomy and of astrophysics. Early in
the morning of this autumn day, shortly after 9am UT, a gravitational
wave passed through the Earth, and on its way produced a sufficiently
large movement in the mirrors and test masses of the detectors of the
Laser Interferometer Gravitational-wave Observatory (\gls{LIGO}), as to be
detected.

Over five months' of data analysis, detector characterisation, and
detection verification were conducted by a global team of scientists,
in the LIGO and VIRGO Scientific Collaboration (LVC). This process
resulted in a slew of journal papers being written, vast quantities of
data produced, and an enormous public outreach effort to be
launched. Eventually, the collaboration found itself in a position to
make the announcement of the first detection of gravitational waves on
11 February 2016.


\section{Gravitational Waves and General Relativity}
\label{sec:grav-waves-gener}

Gravitational waves are one of the prediction of Einstein's 1915 Theory of General Relativity.

\chapter{Detectors}
\label{cha:detectors}

\chapterprecis{Gravitational wave detectors are one of the great
  achievements of Twentieth and early Twenty-First Century
  science. They are the most sensitive measuring devices ever
  constructed, but they face numerous technical challenges.}

Gravitational Wave detectors were first conceived in the 1960s by...

\section{An overview of noise sources}
\label{sec:an-overview-noise}

\section{Newtonian Noise}
\label{sec:newtonian-noise}

\section{Seismic Noise}
\label{sec:seismic-noise}

\section{Other noise sources}
\label{sec:other-noise-sources}

There are numerous additional noise sources within the interferometer.

\glsaddall
\glossarystyle{altlist}
\printglossaries

\end{document}
