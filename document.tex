\documentclass{kentigern}
\usepackage[utf8]{inputenc}
\makeatletter
%
%  These Macros are taken from the AAS TeX macro package version 5.2
%  and are compatible with the macros in the A&A document class
%  version 7.0
%  Include this file in your LaTeX source only if you are not using
%  the AAS TeX macro package or the A&A document class and need to
%  resolve the macro definitions in the TeX/BibTeX entries returned by
%  the ADS abstract service.
%
%  If you plan not to use this file to resolve the journal macros
%  rather than the whole AAS TeX macro package, you should save the
%  file as ``aas_macros.sty'' and then include it in your LaTeX paper
%  by using a construct such as:
%	\documentstyle[11pt,aas_macros]{article}
%
%  For more information on the AASTeX and A&A packages, please see:
%       http://journals.aas.org/authors/aastex.html	
%       ftp://ftp.edpsciences.org/pub/aa/readme.html
%  For more information about ADS abstract server, please see:
%       http://adsabs.harvard.edu/ads_abstracts.html
%

% Abbreviations for journals.  The object here is to provide authors
% with convenient shorthands for the most "popular" (often-cited)
% journals; the author can use these markup tags without being concerned
% about the exact form of the journal abbreviation, or its formatting.
% It is up to the keeper of the macros to make sure the macros expand
% to the proper text.  If macro package writers agree to all use the
% same TeX command name, authors only have to remember one thing, and
% the style file will take care of editorial preferences.  This also
% applies when a single journal decides to revamp its abbreviating
% scheme, as happened with the ApJ (Abt 1991).

\let\jnl@style=\rm
\def\ref@jnl#1{{\jnl@style#1}}

\def\aj{\ref@jnl{AJ}}                   % Astronomical Journal
\def\actaa{\ref@jnl{Acta Astron.}}      % Acta Astronomica
\def\araa{\ref@jnl{ARA\&A}}             % Annual Review of Astron and Astrophys
\def\apj{\ref@jnl{ApJ}}                 % Astrophysical Journal
\def\apjl{\ref@jnl{ApJ}}                % Astrophysical Journal, Letters
\def\apjs{\ref@jnl{ApJS}}               % Astrophysical Journal, Supplement
\def\ao{\ref@jnl{Appl.~Opt.}}           % Applied Optics
\def\apss{\ref@jnl{Ap\&SS}}             % Astrophysics and Space Science
\def\aap{\ref@jnl{A\&A}}                % Astronomy and Astrophysics
\def\aapr{\ref@jnl{A\&A~Rev.}}          % Astronomy and Astrophysics Reviews
\def\aaps{\ref@jnl{A\&AS}}              % Astronomy and Astrophysics, Supplement
\def\azh{\ref@jnl{AZh}}                 % Astronomicheskii Zhurnal
\def\baas{\ref@jnl{BAAS}}               % Bulletin of the AAS
\def\bac{\ref@jnl{Bull. astr. Inst. Czechosl.}}
                % Bulletin of the Astronomical Institutes of Czechoslovakia 
\def\caa{\ref@jnl{Chinese Astron. Astrophys.}}
                % Chinese Astronomy and Astrophysics
\def\cjaa{\ref@jnl{Chinese J. Astron. Astrophys.}}
                % Chinese Journal of Astronomy and Astrophysics
\def\icarus{\ref@jnl{Icarus}}           % Icarus
\def\jcap{\ref@jnl{J. Cosmology Astropart. Phys.}}
                % Journal of Cosmology and Astroparticle Physics
\def\jrasc{\ref@jnl{JRASC}}             % Journal of the RAS of Canada
\def\memras{\ref@jnl{MmRAS}}            % Memoirs of the RAS
\def\mnras{\ref@jnl{MNRAS}}             % Monthly Notices of the RAS
\def\na{\ref@jnl{New A}}                % New Astronomy
\def\nar{\ref@jnl{New A Rev.}}          % New Astronomy Review
\def\pra{\ref@jnl{Phys.~Rev.~A}}        % Physical Review A: General Physics
\def\prb{\ref@jnl{Phys.~Rev.~B}}        % Physical Review B: Solid State
\def\prc{\ref@jnl{Phys.~Rev.~C}}        % Physical Review C
\def\prd{\ref@jnl{Phys.~Rev.~D}}        % Physical Review D
\def\pre{\ref@jnl{Phys.~Rev.~E}}        % Physical Review E
\def\prl{\ref@jnl{Phys.~Rev.~Lett.}}    % Physical Review Letters
\def\pasa{\ref@jnl{PASA}}               % Publications of the Astron. Soc. of Australia
\def\pasp{\ref@jnl{PASP}}               % Publications of the ASP
\def\pasj{\ref@jnl{PASJ}}               % Publications of the ASJ
\def\rmxaa{\ref@jnl{Rev. Mexicana Astron. Astrofis.}}%
                % Revista Mexicana de Astronomia y Astrofisica
\def\qjras{\ref@jnl{QJRAS}}             % Quarterly Journal of the RAS
\def\skytel{\ref@jnl{S\&T}}             % Sky and Telescope
\def\solphys{\ref@jnl{Sol.~Phys.}}      % Solar Physics
\def\sovast{\ref@jnl{Soviet~Ast.}}      % Soviet Astronomy
\def\ssr{\ref@jnl{Space~Sci.~Rev.}}     % Space Science Reviews
\def\zap{\ref@jnl{ZAp}}                 % Zeitschrift fuer Astrophysik
\def\nat{\ref@jnl{Nature}}              % Nature
\def\iaucirc{\ref@jnl{IAU~Circ.}}       % IAU Cirulars
\def\aplett{\ref@jnl{Astrophys.~Lett.}} % Astrophysics Letters
\def\apspr{\ref@jnl{Astrophys.~Space~Phys.~Res.}}
                % Astrophysics Space Physics Research
\def\bain{\ref@jnl{Bull.~Astron.~Inst.~Netherlands}} 
                % Bulletin Astronomical Institute of the Netherlands
\def\fcp{\ref@jnl{Fund.~Cosmic~Phys.}}  % Fundamental Cosmic Physics
\def\gca{\ref@jnl{Geochim.~Cosmochim.~Acta}}   % Geochimica Cosmochimica Acta
\def\grl{\ref@jnl{Geophys.~Res.~Lett.}} % Geophysics Research Letters
\def\jcp{\ref@jnl{J.~Chem.~Phys.}}      % Journal of Chemical Physics
\def\jgr{\ref@jnl{J.~Geophys.~Res.}}    % Journal of Geophysics Research
\def\jqsrt{\ref@jnl{J.~Quant.~Spec.~Radiat.~Transf.}}
                % Journal of Quantitiative Spectroscopy and Radiative Transfer
\def\memsai{\ref@jnl{Mem.~Soc.~Astron.~Italiana}}
                % Mem. Societa Astronomica Italiana
\def\nphysa{\ref@jnl{Nucl.~Phys.~A}}   % Nuclear Physics A
\def\physrep{\ref@jnl{Phys.~Rep.}}   % Physics Reports
\def\physscr{\ref@jnl{Phys.~Scr}}   % Physica Scripta
\def\planss{\ref@jnl{Planet.~Space~Sci.}}   % Planetary Space Science
\def\procspie{\ref@jnl{Proc.~SPIE}}   % Proceedings of the SPIE

\let\astap=\aap
\let\apjlett=\apjl
\let\apjsupp=\apjs
\let\applopt=\ao


\makeatother
\usepackage{lipsum}
\usepackage[acronym,toc,nopostdot,xindy,style=index]{glossaries}
\usepackage{siunitx}

\usepackage{amsmath,amsfonts,amssymb}
\usepackage{tensor}
\makeglossaries
\usepackage{tabularx}
\usepackage{type1cm}
\usepackage{lettrine}
%\usepackage{physicsplus}
\usepackage{caption}
\usepackage{pgfplots}
\usepackage{rotating}
%\usepackage{pgfgantt}
\usepackage{environ}
\usepackage{setspace}

\usetikzlibrary{bayesnet}

\usepackage{doi}

  \usepackage[
  backend=biber,
  sorting=none,
  style=numeric,
  ]{biblatex}
  %\addbibresource{bibliography/observing-runs.bib}
  \addbibresource{bibliography/books.bib}
  \addbibresource{bibliography/mypapers.bib}
  \addbibresource{bibliography/sources.bib}
  \addbibresource{bibliography/relativity.bib}
  \addbibresource{bibliography/detectors.bib}
  \addbibresource{bibliography/events.bib}
  \addbibresource{bibliography/probability.bib}
  \addbibresource{bibliography/data-analysis.bib}
  \addbibresource{bibliography/sources.bib}
  \addbibresource{bibliography/gaussianprocess.bib}
  \addbibresource{bibliography/gammarayburst.bib}
  %
  
  \linespread{1.5}

   \setsecnumdepth{subsubsection}
   \settocdepth{subsection}
%\usepackage{geometry}
\title{Plausible inference methods for gravitational wave data analysis}
\author{Daniel Williams}

%% This file is intended to contain all of the macros defining the
%% various mathematical and physical quantities used in this document in
%% order to maintain notational consistency throughout the document and
%% its glossary.

% Margin notes

\newcommand{\marginnote}[1]{
        \refstepcounter{footnote}
\footnotemark\marginpar{\footnotemark}\footnotetext{#1}}


\NewEnviron{sidefigure}[2]{
  \sidebar{
    \BODY
    \captionof{figure}{#1 \label{#2}}
  }
}

\providecommand{\msolar}{\mathrm{M{_\odot}}}
\providecommand{\mat}[1]{\mathsf{#1}}
\providecommand{\ex}{\mathbb{E}\,}

% New Operators
\DeclareMathOperator{\vary}{var}
\DeclareMathOperator{\cov}{cov}
\providecommand{\rmse}{\mathrm{RMSE}}

% Calculus operators
\providecommand{\dd}{\,\mathrm{d}}

% Software Packages
\providecommand{\imrphenomp}{\texttt{IMRPhenomP}}
\providecommand{\lalsim}{\texttt{LALSimulation}}

% New Units
\providecommand{\solMass}{\ensuremath{\mathrm{M}_{\odot}}}
%\DeclareSIUnit\parsec{pc}

% Galactic astronomy
\providecommand{\numberGalaxies}{N_\mathrm{G}}

% Gravitational wave detectors
\providecommand{\horizonDistance}{\ensuremath\mathcal{D}_{\mathrm{hor}}}

\providecommand{\GP}{\gls{gp}}

% Pipelines
\providecommand{\olib}{\texttt{oLIB}}
\providecommand{\cwb}{\texttt{cWB}}
\providecommand{\bayeswave}{\texttt{Bayeswave}}
\providecommand{\minke}{\texttt{Minke}}

% Software
\providecommand{\lalsuite}{\texttt{LALSuite}}
\providecommand{\lalsimulation}{\texttt{LALSimulation}}
\providecommand{\imrp}{\texttt{IMRPhenom\,v2}}
\providecommand{\seobnr}{\texttt{SEOBNR}}
\providecommand{\heron}{\texttt{heron}}
% latin
\providecommand{\map}{maximum \emph{a posteriori}}

% Gaussian processes
\providecommand{\set}[1]{\mathcal{#1}}
\providecommand{\gp}{\mathcal{G\!P}}
\providecommand{\GP}{Gaussian Process\renewcommand{\GP}{GP}}
\providecommand{\NR}{Numerical Relativity\renewcommand{\NR}{NR}}
\providecommand{\PE}{parameter estimation\renewcommand{\PE}{PE}}
\providecommand{\GW}{gravitational wave\renewcommand{\GW}{GW}}
\providecommand{\EI}{\mathbb{E} \mathrm{I}}

\providecommand{\trainingpoints}{\mathcal{X}}
\providecommand{\trainingobservations}{\mathcal{Y}}
\providecommand{\trainingdata}{(\trainingpoints, \trainingobservations)}

\providecommand{\kernel}[1]{\mathsf{#1}}
\providecommand{\SE}{\kernel{SE}}
\providecommand{\Con}{\kernel{C}}
\providecommand{\Lin}{\kernel{Lin}}
\providecommand{\Per}{\kernel{Per}}
\providecommand{\RQ}{\kernel{RQ}}
\providecommand{\Mat}{\kernel{M52}}

\providecommand{\numbertrainingpoints}{$12,325$}
\providecommand{\numbertrainingwaveforms}{no. Waveforms}



\theoremstyle{definition}
\newtheorem{definition}{Definition}[section]
\newtheorem{theorem}{Theorem}[section]
\newtheorem{corollary}{Corollary}[theorem]
\newtheorem{lemma}[theorem]{Lemma}
% allows for temporary adjustment of side margins
%\usepackage{chngpage}


\usetikzlibrary{bayesnet}


% The glossary
\loadglsentries[main]{chapters/glossary/glossary}
\makeglossaries

\maxsecnumdepth{subsubsection}

\newcommand{\thesistitle}{%
  \thispagestyle{empty}
  %\begingroup%
  \hbox{%
    \hspace*{-0.2\textwidth}%
    \parbox[b]{1.25\textwidth}{%
      \vbox{%
        \begin{center}
          {\noindent\HUGE\bfseries%
            {Plausible Inference Methods\\\textit{for}\\ Gravitational Wave Data Analysis}}\\[2\baselineskip]
        \end{center}
        \begin{center}
          {\noindent\Large\itshape{On the application of Bayesian inference and modern modelling methods to astrophysical problems in the era of gravitational wave observation.}}\\[2\baselineskip]
        \end{center}
        \vspace*{3\baselineskip}
        {\Large Daniel Williams}\\
        {\Large MSci, FRAS, MInstP}\\[3\baselineskip]
        {\today}
      }
    }
    }
    \vfill
    \hspace*{-0.2\textwidth}%
    \parbox[b]{1.25\textwidth}{%
      \vbox{%
        {Submitted in fulfilment of the requirements for the degree of Doctor of Philosophy.}\\
        {School of Physics \& Astronomy}\\
        {College of Science and Engineering}\\
        {University of Glasgow $\cdot$ Oilthigh Glaschu}
      }%
    }%
  \null%
  %\endgroup%
  \newpage
}%


\begin{document}
\openleft
\frontmatter
\thesistitle

\begin{abstract}
  Einstein's publication of the \textit{General Theory of Relativity} in 1915, and the discovery of a wave-like solution to the field-equations of that theory sparked a century-long quest to detect \textit{gravitational waves},
  the illusive metric disturbances which were predicted to ripple-away from some of the most energetic events in the universe, such as supernovae and colliding black holes.
  While this quest was completed in September 2015, with the Laser Interfermeter Gravitational-wave Observatory (LIGO) observation of a gravitational wave produced by a pair of coalescing black holes,
  the age of gravitational wave detection has by no means come to an end,
  with the prospect of myriad detections in the near future to analyse.
\end{abstract}
\newpage

Copyright 2019 Daniel Williams, all rights reserved.\\

This document was built with version 3rdyear-23-g9a21f3f-*
 of the git repository, and has been assigned the identifier LIGO-P1900001 by the LIGO Document Control Centre.

\newpage
\tableofcontents
\newpage
\listoffigures
\newpage
\listoftables
\newpage

\printglossary[type=\acronymtype]


%\section{Acknowledgements}
\label{sec-1}
\begin{quote}
``Any road followed precisely to its end leads precisely nowhere. Climb the mountain just a little bit to test that it's a mountain. From the top of the mountain, you cannot see the mountain.'' --- Frank Herbert, \emph{Dune}.
\end{quote}

The last three-and-a-bit (well, let's be honest, almost four) years have been pretty busy around the Institute for Gravitational Research (IGR).
There have been a few events of some note over that time, and it's been very exciting to be around for them, starting out with what felt like a court-side seat for the adrenaline-fuelled months following the first detection in September 2015, through to being part of the team now that we've entered the era of ``routine detection''.

So my first thanks go to all of the members of the IGR, who kindly allowed me to spend that time working with them, and learning from them.
More generally, I extend my thanks to the members of the LIGO Scientific Collaboration who mentored me as I started to find my way into the field.

I had plenty of opportunity to see parts of the world I'd never have expected to see before starting on this endeavour, and my thanks go to all of the kind hosts who looked after me, whether it was at a world-class lab in the middle of a swamp in Louisiana; in humid Midtown, Atlanta; or the entirely exciting Daejong.

Without the people around me to keep me sane, or distract me from my work, I doubt this tome would ever have come about. 
Euan, Magnus, Rhys, Shona, laura, Alex, and David; you all bear some responsibility for this happening (and must accept some thanks and credit for the fact that it did). 
There were times when I doubted I'd finish at all, and without Andrew I'm sure I wouldn't: thanks for being around to listen to me complaining, and for letting yourself get embroiled in some of my harebrained schemes.

Then to my supervisors, without whom I'd have never discovered the joys of the Bayesian (correct) interpretation of probability: Siong and Graham, thank you for taking on the risk of taking me on as a student.

Finally, without my family this could not have happened.
Thank you Zoe, not least for all of the funny animal videos and pictures.
And thank you to my parents, who first showed me the stars.

\rule{\linewidth}{0.5pt}

My time as a PhD student, and my research, was kindly supported by the Science and Technology Facilities Council (STFC).

% \part{Outline \& Review of Gravitational Wave Literature}
% \label{part:introduction}
\newpage
\section{Notational conventions}
\label{sec:notation-conventions}

Throughout this work I take the convention that the metric tensor, $\ten{g}$ should be positive, having the signature $(-,+,+,+)$, and likewise the Riemann, $\ten{R}$, and Einstein, $\ten{E}$ tensors should also be positive, following the ``spacelike convention'' of Landau \& Lifshitz, and the convention of Misner, Thorne, and Wheeler (1973). I also adopt the convention of using greek indices for four-dimensional tensor quantities, such as 4-vectors, and latin indices otherwise. The Einstein summation convention is also assumed throughout for repeated indices. The reader should note that while the discussion of metrics in the context of general relativity is limited to four-dimensions, those metrics used in feature-space descriptions of data, especially in the context of Gaussian process regression, are not. 

\mainmatter
\chapter{Gravitational Waves: Generation, propagation, and detection}
\label{chapter:intro}

% gravitational waves: generation & propagation
\input{chapters/introduction/gravitational-waves}
% detectors
\input{chapters/introduction/detectors}

\chapter{Astrophysical sources of gravitational waves and their waveforms}
 \label{cha:sourc-grav-waves}
 
 % \chapterprecis{Gravitational waves are produced by any situation
 %   containing accelerating masses which are arranged in an asymmetrical
 %   manner, for example binary star systems, or non-spherical pulsars. A
 %   wide range of astrophysical sources are capable of producing
 %   gravitational waves, although only a handful of these are likely to
 %   be luminous enough to detect, or produce radiation over a frequency
 %   band which can be detected by current-generation detectors.}

 \input{chapters/sources/sources}

\part{Data Analysis for Gravitational Wave Detectors}
\label{part:data-analysis}
 
\chapter{Bayesian inference}
\label{cha:bayesian-inference}

\input{chapters/4-analysis/probability}


\chapter{Hierarchical Modelling of Gamma Ray Bursts}
\label{cha:gamma-ray-burst}

---
title: Gamma ray burst
abbreviation: GRB
---

Gamma ray burst. These are short-lived electromagnetic events which are highly luminous, especially within the gamma ray regime of the spectrum. Events lasting for less than around \si{2}{\second} are classified as ``short'' GRBs (sGRB), while the rest are long GRBs. sGRBs are believed to be the result of binary neutron star coalescence.

\chapter{Gaussian processes}
\label{cha:gaussian-process}

\input{chapters/4-analysis/gaussian-processes}

\chapter{Plausible surrogate modelling for gravitational waveforms}
\label{cha:heron}
\input{chapters/heron/heron}

% \appendices

% \chapter{Concepts from Graph Theory}
% \label{chap:graph-theory}

% \chapter{Gaussian Processes : The Gorey Details}
% \label{chap:gp-details}



%\bibliographystyle{plainnat}
%\bibliography
%\addbibresource{bibliography/books,bibliography/mypapers,bibliography/relativity,bibliography/detectors,bibliography/events,bibliography/probability,bibliography/data-analysis,bibliography/sources,bibliography/analysis,bibliography/gaussianprocess,bibliography/gammarayburst,bibliography/observing-runs}
% \bibliographystyle{unsrt85}




% The glossary
% \glsaddall
\printglossary

\medskip
\printbibliography[title={Bibliography}]

\end{document}
