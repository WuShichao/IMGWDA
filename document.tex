\documentclass[oneside]{kentigern}
\usepackage[utf8]{inputenc}
\makeatletter
%
%  These Macros are taken from the AAS TeX macro package version 5.2
%  and are compatible with the macros in the A&A document class
%  version 7.0
%  Include this file in your LaTeX source only if you are not using
%  the AAS TeX macro package or the A&A document class and need to
%  resolve the macro definitions in the TeX/BibTeX entries returned by
%  the ADS abstract service.
%
%  If you plan not to use this file to resolve the journal macros
%  rather than the whole AAS TeX macro package, you should save the
%  file as ``aas_macros.sty'' and then include it in your LaTeX paper
%  by using a construct such as:
%	\documentstyle[11pt,aas_macros]{article}
%
%  For more information on the AASTeX and A&A packages, please see:
%       http://journals.aas.org/authors/aastex.html	
%       ftp://ftp.edpsciences.org/pub/aa/readme.html
%  For more information about ADS abstract server, please see:
%       http://adsabs.harvard.edu/ads_abstracts.html
%

% Abbreviations for journals.  The object here is to provide authors
% with convenient shorthands for the most "popular" (often-cited)
% journals; the author can use these markup tags without being concerned
% about the exact form of the journal abbreviation, or its formatting.
% It is up to the keeper of the macros to make sure the macros expand
% to the proper text.  If macro package writers agree to all use the
% same TeX command name, authors only have to remember one thing, and
% the style file will take care of editorial preferences.  This also
% applies when a single journal decides to revamp its abbreviating
% scheme, as happened with the ApJ (Abt 1991).

\let\jnl@style=\rm
\def\ref@jnl#1{{\jnl@style#1}}

\def\aj{\ref@jnl{AJ}}                   % Astronomical Journal
\def\actaa{\ref@jnl{Acta Astron.}}      % Acta Astronomica
\def\araa{\ref@jnl{ARA\&A}}             % Annual Review of Astron and Astrophys
\def\apj{\ref@jnl{ApJ}}                 % Astrophysical Journal
\def\apjl{\ref@jnl{ApJ}}                % Astrophysical Journal, Letters
\def\apjs{\ref@jnl{ApJS}}               % Astrophysical Journal, Supplement
\def\ao{\ref@jnl{Appl.~Opt.}}           % Applied Optics
\def\apss{\ref@jnl{Ap\&SS}}             % Astrophysics and Space Science
\def\aap{\ref@jnl{A\&A}}                % Astronomy and Astrophysics
\def\aapr{\ref@jnl{A\&A~Rev.}}          % Astronomy and Astrophysics Reviews
\def\aaps{\ref@jnl{A\&AS}}              % Astronomy and Astrophysics, Supplement
\def\azh{\ref@jnl{AZh}}                 % Astronomicheskii Zhurnal
\def\baas{\ref@jnl{BAAS}}               % Bulletin of the AAS
\def\bac{\ref@jnl{Bull. astr. Inst. Czechosl.}}
                % Bulletin of the Astronomical Institutes of Czechoslovakia 
\def\caa{\ref@jnl{Chinese Astron. Astrophys.}}
                % Chinese Astronomy and Astrophysics
\def\cjaa{\ref@jnl{Chinese J. Astron. Astrophys.}}
                % Chinese Journal of Astronomy and Astrophysics
\def\icarus{\ref@jnl{Icarus}}           % Icarus
\def\jcap{\ref@jnl{J. Cosmology Astropart. Phys.}}
                % Journal of Cosmology and Astroparticle Physics
\def\jrasc{\ref@jnl{JRASC}}             % Journal of the RAS of Canada
\def\memras{\ref@jnl{MmRAS}}            % Memoirs of the RAS
\def\mnras{\ref@jnl{MNRAS}}             % Monthly Notices of the RAS
\def\na{\ref@jnl{New A}}                % New Astronomy
\def\nar{\ref@jnl{New A Rev.}}          % New Astronomy Review
\def\pra{\ref@jnl{Phys.~Rev.~A}}        % Physical Review A: General Physics
\def\prb{\ref@jnl{Phys.~Rev.~B}}        % Physical Review B: Solid State
\def\prc{\ref@jnl{Phys.~Rev.~C}}        % Physical Review C
\def\prd{\ref@jnl{Phys.~Rev.~D}}        % Physical Review D
\def\pre{\ref@jnl{Phys.~Rev.~E}}        % Physical Review E
\def\prl{\ref@jnl{Phys.~Rev.~Lett.}}    % Physical Review Letters
\def\pasa{\ref@jnl{PASA}}               % Publications of the Astron. Soc. of Australia
\def\pasp{\ref@jnl{PASP}}               % Publications of the ASP
\def\pasj{\ref@jnl{PASJ}}               % Publications of the ASJ
\def\rmxaa{\ref@jnl{Rev. Mexicana Astron. Astrofis.}}%
                % Revista Mexicana de Astronomia y Astrofisica
\def\qjras{\ref@jnl{QJRAS}}             % Quarterly Journal of the RAS
\def\skytel{\ref@jnl{S\&T}}             % Sky and Telescope
\def\solphys{\ref@jnl{Sol.~Phys.}}      % Solar Physics
\def\sovast{\ref@jnl{Soviet~Ast.}}      % Soviet Astronomy
\def\ssr{\ref@jnl{Space~Sci.~Rev.}}     % Space Science Reviews
\def\zap{\ref@jnl{ZAp}}                 % Zeitschrift fuer Astrophysik
\def\nat{\ref@jnl{Nature}}              % Nature
\def\iaucirc{\ref@jnl{IAU~Circ.}}       % IAU Cirulars
\def\aplett{\ref@jnl{Astrophys.~Lett.}} % Astrophysics Letters
\def\apspr{\ref@jnl{Astrophys.~Space~Phys.~Res.}}
                % Astrophysics Space Physics Research
\def\bain{\ref@jnl{Bull.~Astron.~Inst.~Netherlands}} 
                % Bulletin Astronomical Institute of the Netherlands
\def\fcp{\ref@jnl{Fund.~Cosmic~Phys.}}  % Fundamental Cosmic Physics
\def\gca{\ref@jnl{Geochim.~Cosmochim.~Acta}}   % Geochimica Cosmochimica Acta
\def\grl{\ref@jnl{Geophys.~Res.~Lett.}} % Geophysics Research Letters
\def\jcp{\ref@jnl{J.~Chem.~Phys.}}      % Journal of Chemical Physics
\def\jgr{\ref@jnl{J.~Geophys.~Res.}}    % Journal of Geophysics Research
\def\jqsrt{\ref@jnl{J.~Quant.~Spec.~Radiat.~Transf.}}
                % Journal of Quantitiative Spectroscopy and Radiative Transfer
\def\memsai{\ref@jnl{Mem.~Soc.~Astron.~Italiana}}
                % Mem. Societa Astronomica Italiana
\def\nphysa{\ref@jnl{Nucl.~Phys.~A}}   % Nuclear Physics A
\def\physrep{\ref@jnl{Phys.~Rep.}}   % Physics Reports
\def\physscr{\ref@jnl{Phys.~Scr}}   % Physica Scripta
\def\planss{\ref@jnl{Planet.~Space~Sci.}}   % Planetary Space Science
\def\procspie{\ref@jnl{Proc.~SPIE}}   % Proceedings of the SPIE

\let\astap=\aap
\let\apjlett=\apjl
\let\apjsupp=\apjs
\let\applopt=\ao


\makeatother
\usepackage{lipsum}
\usepackage[acronym,toc,nopostdot,xindy,style=index]{glossaries} %previously index
\usepackage{siunitx}

\usepackage{amsmath,amsfonts,amssymb}
\usepackage{tensor}
\usepackage{etoolbox}
\makeglossaries
\usepackage{tabularx}
\usepackage{type1cm}
%\usepackage{lettrine}
%\usepackage{physicsplus}
%\usepackage{caption}
\usepackage{pgfplots}
\usepackage{rotating}
%\usepackage{pgfgantt}
\usepackage{environ}
\usepackage{setspace}

\usepackage{pdflscape}

\usetikzlibrary{bayesnet}
\usepackage{doi}

\usepackage[
  backend=biber,
  sorting=none,
  style=numeric,
  ]{biblatex}
  
\addbibresource{bibliography/observing-runs.bib}
\addbibresource{bibliography/general-relativity.bib}
%\addbibresource{bibliography/books.bib}
\addbibresource{bibliography/mypapers.bib}
\addbibresource{bibliography/software.bib}
\addbibresource{bibliography/sources.bib}
\addbibresource{bibliography/relativity.bib}
\addbibresource{bibliography/detectors.bib}
\addbibresource{bibliography/events.bib}
\addbibresource{bibliography/probability.bib}
\addbibresource{bibliography/data-analysis.bib}
\addbibresource{bibliography/sources.bib}
\addbibresource{bibliography/gaussianprocess.bib}
\addbibresource{bibliography/gammarayburst.bib}
\DeclareFieldFormat{doi}{%
   \mkbibacro{DOI}\addcolon\space
   \ifhyperref
     {\href{http://doi.org/#1}{\nolinkurl{#1}}}
     {\nolinkurl{#1}}}
   %
   \DeclareBibliographyDriver{thesis}{%
  \usebibmacro{bibindex}%
  \usebibmacro{begentry}%
  \usebibmacro{author}%
  \setunit{\labelnamepunct}\newblock
  \printlist{language}%
  \newunit\newblock
  \usebibmacro{byauthor}%
  \newunit\newblock%Added by Marco
  \usebibmacro{title}%Added by Marco
  \newunit\newblock
  \printfield{note}%
  \newunit\newblock
  \printfield{type}%
  \newunit
  \usebibmacro{institution+location+date}%
  \newunit\newblock
  \usebibmacro{chapter+pages}%
  \newunit
  \printfield{pagetotal}%
  \newunit\newblock
  \iftoggle{bbx:isbn}
    {\printfield{isbn}}
    {}%
  \newunit\newblock
  \usebibmacro{doi+eprint+url}%
  \newunit\newblock
  \usebibmacro{addendum+pubstate}%
  \setunit{\bibpagerefpunct}\newblock
  \usebibmacro{pageref}%
  \usebibmacro{
finentry}%
}

\linespread{1.5}

\setsecnumdepth{subsubsection}
\settocdepth{subsection}

\title{Plausible inference methods for gravitational wave data analysis}
\author{Daniel Williams}

%% This file is intended to contain all of the macros defining the
%% various mathematical and physical quantities used in this document in
%% order to maintain notational consistency throughout the document and
%% its glossary.

% Margin notes

\newcommand{\marginnote}[1]{
        \refstepcounter{footnote}
\footnotemark\marginpar{\footnotemark}\footnotetext{#1}}


\NewEnviron{sidefigure}[2]{
  \sidebar{
    \BODY
    \captionof{figure}{#1 \label{#2}}
  }
}

\providecommand{\msolar}{\mathrm{M{_\odot}}}
\providecommand{\mat}[1]{\mathsf{#1}}
\providecommand{\ex}{\mathbb{E}\,}

% New Operators
\DeclareMathOperator{\vary}{var}
\DeclareMathOperator{\cov}{cov}
\providecommand{\rmse}{\mathrm{RMSE}}

% Calculus operators
\providecommand{\dd}{\,\mathrm{d}}

% Software Packages
\providecommand{\imrphenomp}{\texttt{IMRPhenomP}}
\providecommand{\lalsim}{\texttt{LALSimulation}}

% New Units
\providecommand{\solMass}{\ensuremath{\mathrm{M}_{\odot}}}
%\DeclareSIUnit\parsec{pc}

% Galactic astronomy
\providecommand{\numberGalaxies}{N_\mathrm{G}}

% Gravitational wave detectors
\providecommand{\horizonDistance}{\ensuremath\mathcal{D}_{\mathrm{hor}}}

\providecommand{\GP}{\gls{gp}}

% Pipelines
\providecommand{\olib}{\texttt{oLIB}}
\providecommand{\cwb}{\texttt{cWB}}
\providecommand{\bayeswave}{\texttt{Bayeswave}}
\providecommand{\minke}{\texttt{Minke}}

% Software
\providecommand{\lalsuite}{\texttt{LALSuite}}
\providecommand{\lalsimulation}{\texttt{LALSimulation}}
\providecommand{\imrp}{\texttt{IMRPhenom\,v2}}
\providecommand{\seobnr}{\texttt{SEOBNR}}
\providecommand{\heron}{\texttt{heron}}
% latin
\providecommand{\map}{maximum \emph{a posteriori}}

% Gaussian processes
\providecommand{\set}[1]{\mathcal{#1}}
\providecommand{\gp}{\mathcal{G\!P}}
\providecommand{\GP}{Gaussian Process\renewcommand{\GP}{GP}}
\providecommand{\NR}{Numerical Relativity\renewcommand{\NR}{NR}}
\providecommand{\PE}{parameter estimation\renewcommand{\PE}{PE}}
\providecommand{\GW}{gravitational wave\renewcommand{\GW}{GW}}
\providecommand{\EI}{\mathbb{E} \mathrm{I}}

\providecommand{\trainingpoints}{\mathcal{X}}
\providecommand{\trainingobservations}{\mathcal{Y}}
\providecommand{\trainingdata}{(\trainingpoints, \trainingobservations)}

\providecommand{\kernel}[1]{\mathsf{#1}}
\providecommand{\SE}{\kernel{SE}}
\providecommand{\Con}{\kernel{C}}
\providecommand{\Lin}{\kernel{Lin}}
\providecommand{\Per}{\kernel{Per}}
\providecommand{\RQ}{\kernel{RQ}}
\providecommand{\Mat}{\kernel{M52}}

\providecommand{\numbertrainingpoints}{$12,325$}
\providecommand{\numbertrainingwaveforms}{no. Waveforms}



\theoremstyle{definition}
\newtheorem{definition}{Definition}[section]
\newtheorem{theorem}{Theorem}[section]
\newtheorem{corollary}{Corollary}[theorem]
\newtheorem{lemma}[theorem]{Lemma}
% allows for temporary adjustment of side margins
%\usepackage{chngpage}


\usetikzlibrary{bayesnet}


% The glossary
\loadglsentries[main]{chapters/glossary/glossary}
\makeglossaries

\maxsecnumdepth{subsubsection}

\newcommand{\thesistitle}{%
  \thispagestyle{empty}
  %\begingroup%
  \hbox{%
    \hspace*{-0.2\textwidth}%
    \parbox[b]{1.25\textwidth}{%
      \vbox{%
        \begin{center}
          {\noindent\HUGE\bfseries%
            {Plausible Inference Methods\\\textit{for}\\ Gravitational Wave Data Analysis}}\\[2\baselineskip]
        \end{center}
        \begin{center}
          {\noindent\Large\itshape{On the application of Bayesian inference and modern modelling methods to astrophysical problems in the era of gravitational wave observation.}}\\[2\baselineskip]
        \end{center}
        \vspace*{3\baselineskip}
        {\Large Daniel Williams}\\
        {\Large MSci, FRAS, MInstP}\\[3\baselineskip]
        {\today}
      }
    }
    }
    \vfill
    \hspace*{-0.2\textwidth}%
    \parbox[b]{1.25\textwidth}{%
      \vbox{%
        {Submitted in fulfilment of the requirements for the degree of Doctor of Philosophy.}\\
        {School of Physics \& Astronomy}\\
        {College of Science and Engineering}\\
        {University of Glasgow $\cdot$ Oilthigh Glaschu}
      }%
    }%
  \null%
  %\endgroup%
  \newpage
}%


\begin{document}
\openright
\frontmatter
\thesistitle
\newpage \newpage

\begin{abstract}
  Einstein's publication of the \textit{general theory of relativity} in 1915, and the discovery of a wave-like solution to the field-equations of that theory sparked a century-long quest to detect \textit{gravitational waves}.
  These illusive metric disturbances were predicted to ripple-away from some of the most energetic events in the universe, such as supernovae and colliding black holes.
  
  The quest was completed in September 2015, with the \gls{ligo} observation of a gravitational wave produced by a pair of coalescing black holes, but work to continue detecting and interpretting the signals which are detected by \gls{ligo} and its brethren is by no means complete.
  The age of gravitational wave observation has arrived, and with it the difficulties of interpretting myriad signals, differentiating them from noise, and analysing them in order to gain insight into the astrophysical systems which produced them.

 This thesis provides overview of the history of the field of gravitational wave science: both in terms of the theoretical principles which frame it, and the attempts to build instruments which could measure them.
  It then provides a discussion of the morphologies of the signals which are searched for in current detectors' data, and the astrophysical systems which may produce such signals.
  It is of great importance that the sensitivity of both detectors and the signal analysis techniques which are used is well-understood.
  A substantial part of the novel work presented in this document discusses the development of a technique for assessing this sensitivity, through a software package called Minke.

  Knowing the sensitivity of a detector to signals from an astrophysical source allows robust limits to be placed on the rate at which these events occur.
  These rates can then be used to infer properties of astrophysical systems; this document contains a discussion of a technique which was developed by the author to allow the determination of the geometry of beamed emission from short gamma ray bursts which result from neutron star coalescences.
  This method finds that at its design sensitivity we expect the advanced LIGO detector to be able to place limits on the opening angle, $\theta$, of the beam within $\theta \in (8.10^\circ,14.95^\circ)$ under the assumption that all neutron star coalesences produce jets, and that gamma ray bursts occur at an illustrative rate of $\grbrate=10$\,Gpc$^{-3}$\,yr$^{-1}$}.

  The most efficient methods for extracting signals from noisy data, such as that produced by gravitational wave detectors, and then analysing these signals, requires robust prior knowledge of the signals' morphologies.
  The development of a new model for producing gravitational waveforms for coalescing binary black hole systems is discussed in detail in this work.
  The method which is used, Gaussian process regression, is introduced, with an overview of different methods for implementing models which use the method.
  The model, named Heron, is itself presented, and comparisons between the waveforms produced by Heron and other models which are currently used in analysis are made.
  Comparisons between the Heron model and highly accurate numerical relativity waveforms are also shown.

\end{abstract}
\newpage

Copyright 2019 Daniel Williams, all rights reserved.\\

This document has been assigned the identifier LIGO-P1900001 by the LIGO Document Control Center.
\newpage

%\input{chapters/0-frontmatter/declaration}

\section{Acknowledgements}
\label{sec-1}
\begin{quote}
``Any road followed precisely to its end leads precisely nowhere. Climb the mountain just a little bit to test that it's a mountain. From the top of the mountain, you cannot see the mountain.'' --- Frank Herbert, \emph{Dune}.
\end{quote}

The last three-and-a-bit (well, let's be honest, almost four) years have been pretty busy around the Institute for Gravitational Research (IGR).
There have been a few events of some note over that time, and it's been very exciting to be around for them, starting out with what felt like a court-side seat for the adrenaline-fuelled months following the first detection in September 2015, through to being part of the team now that we've entered the era of ``routine detection''.

So my first thanks go to all of the members of the IGR, who kindly allowed me to spend that time working with them, and learning from them.
More generally, I extend my thanks to the members of the LIGO Scientific Collaboration who mentored me as I started to find my way into the field.

I had plenty of opportunity to see parts of the world I'd never have expected to see before starting on this endeavour, and my thanks go to all of the kind hosts who looked after me, whether it was at a world-class lab in the middle of a swamp in Louisiana; in humid Midtown, Atlanta; or the entirely exciting Daejong.

Without the people around me to keep me sane, or distract me from my work, I doubt this tome would ever have come about. 
Euan, Magnus, Rhys, Shona, laura, Alex, and David; you all bear some responsibility for this happening (and must accept some thanks and credit for the fact that it did). 
There were times when I doubted I'd finish at all, and without Andrew I'm sure I wouldn't: thanks for being around to listen to me complaining, and for letting yourself get embroiled in some of my harebrained schemes.

Then to my supervisors, without whom I'd have never discovered the joys of the Bayesian (correct) interpretation of probability: Siong and Graham, thank you for taking on the risk of taking me on as a student.

Finally, without my family this could not have happened.
Thank you Zoe, not least for all of the funny animal videos and pictures.
And thank you to my parents, who first showed me the stars.

\rule{\linewidth}{0.5pt}

My time as a PhD student, and my research, was kindly supported by the Science and Technology Facilities Council (STFC).


\newpage
\tableofcontents
\newpage
\listoffigures
\newpage
\listoftables
\newpage

\printglossary[type=\acronymtype]



% \part{Outline \& Review of Gravitational Wave Literature}
% \label{part:introduction}
\newpage
\chapter{Notational conventions}
\label{sec:notation-conventions}

Throughout this work I take the convention that the metric tensor, $\ten{g}$ should be positive, having the signature $(-,+,+,+)$, and likewise the Riemann, $\ten{R}$, and Einstein, $\ten{E}$, tensors should also be positive, following the ``spacelike convention'' of Landau \& Lifshitz \cite{1975ctf..book.....L}, and the convention of Misner, Thorne, and Wheeler \cite{mtw}. I also adopt the convention of using greek indices for four-dimensional tensor quantities, such as 4-vectors, and latin indices otherwise. The reader should note that while the discussion of metrics in the context of general relativity is limited to four-dimensions, those metrics used in feature-space descriptions of data, especially in the context of Gaussian process regression, are not.
In chapter \ref{cha:intro}, and in places where general relativity is discussed it should be assumed that indexed tensor quantities follow the Einstein summation convention over repeated indices.

\mainmatter
\glsresetall
\chapter[Gravitational Waves: Generation, propagation, and detection]{Gravitational Waves: generation, propagation, and detection}
\label{cha:intro}

\chapterprecis{

\noindent This chapter introduces gravitational waves, and briefly discusses their description in general relativity, and their propagation through spacetime.
  The chapter then continues to discuss methods by which they might be detected, and a brief history of attempts to do so.

  Section \ref{sec:gw} contains an introduction and overview of the radiation predicted by general relativity: gravitational waves, and gives a very concise introduction to how these waves propagate.
  Section \ref{sec:gw:strain} briefly summarises a number of conventions which are used within gravitational wave astronomy, and this work, for describing the strain effect of gravitational waves, and related quantities.
  Section \ref{sec:detectors} gives an overview of the history of gravitational wave detection, including the early attempts to identify measurable quantities from the theory, to the operational constraints of modern detectors.
  Section \ref{sec:detectors:noise} discusses some of the major sources of noise which are introduced into detector data by environmental and instrumental phenomena.
  Section \ref{sec:detectors:network} provides an overview of the present network of gravitational wave detectors around the world.
  Section \ref{sec:detections} is a short discussion of the detections which have been made to date in the first two observing runs of the advanced-era detectors.
  
  The vast majority of material in this chapter is review material, however a number of sensitivity curve plots were produced by a Python package which I developed, \texttt{gravpy}, which is described in appendix \ref{app:gravpy}. This is noted in the captions for these figures.
}

\input{chapters/introduction/gravitational-waves}

\glsresetall % reset all of the acronyms so they show in full
\chapter{Astrophysical sources of gravitational waves and their waveforms}
\label{cha:sources}

\chapterprecis{

\noindent This chapter discusses the astrophysical sources of gravitational waves, and the form that the gravitational waves take, which we expect to detect as signals.
  Section \ref{sec:sources:continuous} introduces continuous sources of gravitational waves, such as gravitational wave pulsars.
  Section \ref{sec:sources:stochastic} provides a brief overview of the stochastic gravitational wave background.
  Section \ref{sec:sources:cbc} discusses compact binary coalescence events, which include binary black hole and binary neutron star merger events.
  This section contains discussion of the waveform of these events in section \ref{sec:sources:cbc:waveform}, and the  numerical relativity techniques used to produce the most accurate waveforms available in section \ref{sec:sources:cbc:waveform}; a discussion of the available catalogues of these data is contained within section \ref{sec:sources:cbc:catalogues}.
  Because of the computational expense of producing these waveforms a number of analytical approximant waveforms exist, and I discuss some of these in sections \ref{sec:sources:cbc:approximants} and \ref{sec:sources:cbc:surrogates}.
  
  There are many conceivable astrophysical situations where computing the waveform beforehand is likely to be impractical, or impossible.
  As a result a number of modern searches attempt to search for \emph{unmodelled} or poorly-modelled signals.
  Transient signals of this type are normally called \emph{burst} signals.
  Section \ref{sec:sources:burst} provides an overview of burst-like waveforms, and some of the potential sources of these signals.

  Finally, in section \ref{sources:burst:encounters} I provide discussion and results from a study I have conducted into the detectability of burst signals resulting from encounters between black holes which either result in glancing, parabolic encounters, or radiation driven capture and a subsequent merger event.
}

\input{chapters/sources/sources}
\glsresetall % reset all of the acronyms so they show in full

\chapter{Burst searches and mock data challenges}
\label{cha:searches}
\chapterprecis{

\noindent This chapter introduces searches for unmodelled gravitational wave transients (``bursts''), and then discusses the development of the Minke software library which I developed during the course of my PhD, which is used to produce mock data challenges, which are used to measure the sensitivity of burst search algorithms.

Section \ref{sec:sources:burstsearch} then gives an overview of techniques used to search for these signals in detector data.

Section \ref{sec:sources:mdc} discusses mock data challenges, and how these are used to characterise search algorithms.

Section \ref{sec:sources:minke} introduces Minke.

Section \ref{sec:sources:burstresults} then provides a discussion of the measured sensitivity of a variety of burst search algorithms which were employed on detector data from the first two advanced-era observing runs. This is largely a discussion of the results from \cite{2019PhRvD.100b4017A}, which used mock data challenges produced by the Minke library.
  }
\input{chapters/searches/searches}
% \part{Data Analysis for Gravitational Wave Detectors}
% \label{part:data-analysis}
\glsresetall % reset all of the acronyms so they show in full

\chapter{Bayesian inference}
\label{cha:bayesian-inference}

\chapterprecis{

\noindent This chapter presents an overview and review of statistical probability from a Bayesian perspective, and serves mostly to introduce concepts which will be relied upon in later chapters. While a few examples have been created by adapting classic examples (for example the example of structured distributions in section \ref{sec:probability:structured} is very nearly the classical ``sprinkler'' example, with small modifications to be more astrophysically relevant) the material is not novel.

  Section \ref{sec:probability:basic} introduces probability from an axiomatic basis, and section \ref{sec:probability:information} introduces concepts from information theory. Section \ref{sec:probability:priors} discusses how prior information can be treated within a Bayesian framework.
  Methods for dealing with complicated data using feature spaces are introduced in section \ref{sec:probability:features}, and structured probability distributions are discussed in section \ref{sec:probability:structured}.
  The process of statistical inference is introduced in section \ref{sec:probability:inference}, and stochastic processes are introduced in \ref{sec:probability:stochastic}.
  Approximate inference techniques are introduced in section \ref{sec:probability:approximate}, and hierarchical modelling, to which they are highly applicable, is discussed in section \ref{sec:probability:hierarchical}.
  Finally, Bayesian linear regression is briefly introduced in section \ref{sec:probability:regression}.
  }

\input{chapters/analysis/probability}
\glsresetall % reset all of the acronyms so they show in full

\chapter{Hierarchical Modelling of Gamma Ray Bursts}
\label{cha:gamma-ray-burst}

\chapterprecis{

\noindent This chapter is composed principally from the method presented in \cite{dwsgrbbayesianconstraint}, however the introductory material (in section \ref{sec:grb:sgrbs}) is new compared to that work. Section \ref{sec:grb:rate2beam} discusses how the beaming angle of short gamma ray bursts may be inferred from the observed astrophysical rate of these events.
  Section \ref{sec:grb:prospectsaligo} discusses what the plausible number of detections during the advanced LIGO observing runs will be, while section \ref{sec:grb:validation} provides a validation of the technique used to infer the beaming angle from these rates.
  Section \ref{sec:grb:jetposterior} provides the bulk of the results from this investigation; figure \ref{fig:grb:results:2017} illustrates the posterior which can be placed on the beaming angle after an observing run such as advanced LIGO's O2 run. Table \ref{tab:grb:results} contains a summary of the limits placed over a number of observing scenarios, under a number of different prior assumptions about the fraction of coalescences which produce a jet.  This discussion is extended in section \ref{sec:grb:beyond} with an investigation into the beaming angle inference at much higher sensitivities than are currently possible. Section \ref{sec:grb:conclusions} provides a summary of the contents of the chapter.}
  

---
title: Gamma ray burst
abbreviation: GRB
---

Gamma ray burst. These are short-lived electromagnetic events which are highly luminous, especially within the gamma ray regime of the spectrum. Events lasting for less than around \si{2}{\second} are classified as ``short'' GRBs (sGRB), while the rest are long GRBs. sGRBs are believed to be the result of binary neutron star coalescence.


\glsresetall % reset all of the acronyms so they show in full

\chapter{Gaussian processes for surrogate modelling}
\label{cha:gaussian-process}

\chapterprecis{
  
\noindentThis chapter introduces Gaussian processes, and Gaussian process regression, a probabilistic regression technique which is well suited to producing so-called surrogate models: statistical models which are informed primarily by observed data rather than prior physical knowledge.\\
  This chapter contains mainly introductory and review material, with section \ref{sec:gp:examples} containing a novel example of a Gaussian process regression model fitting a two-dimensional function.
  In section \ref{sec:gp:surrogate} I introduce the basic ideas of surrogate modelling, and briefly discuss various techniques which exist for producing such models.
  Then in section \ref{sec:gp:gp} I introduce the Gaussian process, a probabilistic model, and its use in Gaussian process regression.
  Section \ref{sec:gp:covariance} discusses the importance of covariance functions for Gaussian process models, and discusses a number of potential choices of covariance function for different situations. A number of examples are provided in section \ref{sec:gp:covariance:examples}.
  Section \ref{sec:gp:training} discusses the fitting (or training) of Gaussian process regression models.
  Section \ref{sec:gp:complexity} discusses some of the computational constraints which Gaussian process regression can encounter, and methods which can be used to overcome these.
  Section \ref{sec:gp:testing} contains a brief overview of conventional methods used for testing the accuracy of Gaussian process regression models.
  Section \ref{sec:gp:examples} contains a novel example of a Gaussian process regression model being used to determine contour lines on a geographical map, and shows examples of outputs from a range of different covariance functions.
  Section \ref{sec:gp:elliptical} briefly discusses more general elliptical processes, which can be seen as the generalisation of Gaussian processes.
  }

\input{chapters/analysis/gaussian-processes}
\glsresetall % reset all of the acronyms so they show in full

\chapter{Heron: A Gaussian process regression approach to modelling gravitational waveforms}
\label{cha:heron}

\chapterprecis{

\noindent In this chapter I introduce the use of Gaussian process regression as a surrogate modelling method for \gls{gw} waveforms from compact binary coalesensce events.
  Some of the work presented in this chapter has previously been made available on the ArXiv \cite{2019arXiv190309204W}, and at the time of writing are under review for publication. 
  The culmination of this effort is the development of the \textit{Heron} surrogate model, which is capable of producing waveforms with precession for arbitrarily spinning binary systems.
  Additional results are presented, specifically related to the comparison between the Heron model and pre-exisiting approximate models.

  In section \ref{sec:heron:nrdata} I discuss the training data used for the model, which is derived from the Georgia Tech waveform catalogue, and how the waveform data from the catalogue are prepared.
  In section \ref{sec:heron:waveformgpr} I extend the discussion of Gaussian process regression from chapter \ref{cha:gaussian-process} to the case of gravitational waveform data, and produce two simplified example waveform models using this technique.
  In section \ref{sec:heron:trainingdata} I describe the development of the Heron model itself, using numerical relativity data, and discuss some of the practical computational considerations for a model using this data.
  In section \ref{sec:heron:verification} I discuss a number of tests which can be performed on the Heron model in order to demonstrate its efficacy across its parameter space, and I present results from these tests.
  In section \ref{sec:heron:examples} I plot a number of example waveforms produced by the Heron model.
  Finally, section \ref{sec:heron:summary} provides an overview of the material in the chapter, and some discussion of future work which would be of value to the continued development of this model.

  This work was made possible thanks to data provided by Georgia Institute for Technology (Georgia Tech) and the Simulating Extreme Spacetimes (SXS) collaboration. 
}

\input{chapters/heron/heron}


\appendices
\chapter{Gravpy}
\label{app:gravpy}
\input{chapters/appendices/gravpy}

% % The glossary
% % \glsaddall
\printglossary

\medskip
\printbibliography[title={Bibliography}]

\end{document}
