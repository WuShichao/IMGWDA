
\newglossaryentry{gw150914}{
   name = {GW150914},
   description = {The first gravitational wave event to be observed. The observation was
made by the two \gls{ligo} detectors in the United States of America,
located at Hanford, Washington; and Livingston, Louisiana},
symbol = {}
}

\newglossaryentry{gamma ray burst}{
   name = {Gamma ray burst},
   description = {Gamma ray burst. These are short-lived electromagnetic events which are highly luminous, especially within the gamma ray regime of the spectrum. Events lasting for less than around \si{2}{\second} are classified as ``short'' GRBs (sGRB), while the rest are long GRBs. sGRBs are believed to be the result of binary neutron star coalescence.},
symbol = {}
}

% \newglossaryentry{horizon distance}{
%    name = {horizon distance},
%    description = {The horizon distance is the greatest luminosity distance at which a
% signal, typically an inspiral signal, can be detected by an
% interferometer with some given noise profile.

% The detection threshold is normally expressed in terms of a \gls{snr},
% $\glssymbol{snr}$, and in the case of a single detector a common
% choice is $\glssymbol{snr} = 8$.

% Given the \gls{psd} of the detector, $\glssymbol{psd}$, and a
% \gls{chirp mass}, $\glssymbol{chirp mass}$, then the horizon distance
% can be approximated as\cite{2012arXiv1203.2674T}
% % \begin{equation}
% %   %%\mathcal{D}_{\mathrm{hor}} = \frac{1}{\glssymbol{snr}}
% %   %% \left(
% %   %%  \frac{5 \pi}{24 c^3} \right)^{1/2} \left(G \glssymbol{chirp mass} \right)^{5/6} \pi^{-7/6} \sqrt{ 4 \int_f_{\mathrm low}^f_{\mathrm high} \frac{f^{-7/3}}{\glssymbol{psd}(f) } \dd{f} } 
% %  \end{equation}

% for an inspiral signal, where the SNR of the merger and ringdown is neglected.},
% symbol = {}
% }

\newglossaryentry{psd}{
   name = {power spectral density},
   description = {PSD},
symbol = {\ensuremath S}
}

\newglossaryentry{snr}{
   name = {signal to noise ratio},
   description = {SNR},
symbol = {\ensuremath \rho}
}

\newglossaryentry{virgo}{
   name = {VIRGO},
   description = {A gravitational-wave detector in Italy.},
symbol = {}
}

\newglossaryentry{finesse}{
   name = {finesse},
   description = {Finesse.},
symbol = {}
}

\newglossaryentry{milky way equivalent galaxy}{
   name = {Milky Way Equivalent Galaxy},
   description = {Given some horizon distance, $\horizonDistance$, we can approximate
the number of MWEGs, $\numberGalaxies$, within an observable volume
as\cite{2010CQGra..27q3001A}
\begin{equation}
\numberGalaxies = \frac{4}{3} \pi \qty(\frac{\horizonDistance}{\mega\parsec})^3 (2.26)^{-3} (0.0116)
\end{equation}
where the factor of $2.26$ is a correction from averaging over the
whole sky and all orientations, while $1.16\e{-2} \mega\parsec^{-3}$
is the extrapolated density of MWEGs in space.},
symbol = {}
}

\newglossaryentry{ligo}{
   name = {LIGO},
   description = {Laser interferometer gravitational-wave observatory.},
symbol = {}
}

\newglossaryentry{binary neutron star coalescence}{
   name = {Binary Neutron Star coalescence},
   description = {There are around 70 known NSs.},
symbol = {}
}

\newglossaryentry{gravitational wave}{
   name = {Gravitational Wave},
   description = {A gravitational wave is a propogating perturbation of spacetime.},
symbol = {}
}

\newglossaryentry{chirp mass}{
   name = {chirp mass},
   description = {Chirpmass},
symbol = {}
}
