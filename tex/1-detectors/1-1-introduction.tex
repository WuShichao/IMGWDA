The earliest attempts to develop a gravitational wave detector were
made in the 1960s with the experiments of Jospeh Weber (1919--2000) at
the University of Maryland, from which he claimed the first detection
in 1969\cite{1969PhRvL..22.1320W,1970PhRvL..24..276W} of signals
originating in the galactic centre\cite{1970PhRvL..25..180W}. Numerous
attempts to confirm his findings were unsuccessful, including searches
in Drever's group at the University of Glasgow
\cite{1973Natur.246..340D} in the UK, at Bell Labs
\cite{1973PhRvL..31..173L,1973PhRvL..31..176G,1974PhRvL..33..794L} in
the USA, at Munich\cite{1975NCimL..12..111B,1975NCimL..12..111B} in
Germany, at Moscow\cite{1973PhLA...45..271B} in Russia, and at
Tokyo\cite{1975PhRvL..35..890H} in Japan. While Weber's original
detections were soundly refuted by the community there is little doubt
that the announcement led to a flurry of activity in the field. This
lead to the development of modern cryogenic resonant bars, such as
ALTAIR\cite{1992NCimC..15..943B}, ALLEGRO\cite{2000IJMPD...9..229M},
NAUTILUS\cite{1997APh.....7..231A}, and
EXPLORER\cite{1993PhRvD..47..362A}; and laser interferometers.


\sidebar{
  \includegraphics{figures/bar-detector-psd.pdf}
  \captionof{figure}{The ASDs of the generation of modern, cryogenic resonant mass detectors. Note the narrow band-width in comparison to interferometric detectors, such as Advanced LIGO (see figure \ref{fig:aligo-design-asd})}
  }

\sidebar{
  \includegraphics{figures/first-gen-asd.pdf}
  \captionof{figure}{The ASD of the first generation gravitational wave detectors.}
  }

Laser interferometers, of which advanced LIGO is an implementation,
were the result of a quest for both higher sensitivities and
bandwidth. The possibility of using a Michelson interferometer to
measure the distance between test masses in order to detect
gravitational radiation originated in Moscow\cite{1963JETP...16..433G}
in 1963.  This approach was followed early-on by Scottish and German
groups as a means of improving on resonant bar sensitivities, with a
3-meter and later a 30-meter prototype detector constructed at
Garching in the late
1970s\cite{1979JPhE...12.1043B,1988PhRvD..38..423S} which used optical
delay lines, and a 1-meter prototype, and later a 10-meter instrument
were built at Glasgow in the early
1980s\cite{1979RSPSA.368...11D,1995RScI...66.4447R}, which used
Fabry-Perot cavities . The Glasgow detector was the spiritual
predecessor to the CalTech 40-meter
prototype\cite{1996PhLA..218..157A}. The increasing maturity of
technology developed by these prototypes lead to the construction of
the first generation of long-baseline detectors, starting with TAMA in
Tokyo\cite{1996JKASS..29..279K}, the joint UK-German GEO600
detector\cite{1997CQGra..14.1471L} near Hannover, and the two
kilometre-scale detectors, the joint CalTech-MIT detectors
LIGO\cite{1992Sci...256..325A}, located at two sites in the USA, and
the joint Italian-French detector VIRGO\cite{1990NIMPA.289..518B},
near Cascina. These detectors were operated during the 2000s, and
while none of them made a detection of gravitational waves, they
provided valuable astrophysical results by placing astrophysical
limits on the strength of the stochastic gravitational wave background
\cite{2014PhRvL.113w1101A}, production of gravitational waves by
pulsars\cite{2014ApJ...785..119A} and gamma ray
bursts\cite{2012ApJ...760...12A}, and the rate of compact binary
coalescence in the local
universe\cite{2012PhRvD..85h2002A,2013PhRvD..87b2002A}.

The initial-generation of detectors were upgraded during the first
half of the 2010s, leading to Advanced LIGO\cite{2015CQGra..32g4001L}
which resumed observations in September 2015, and the imminent start
of observations from the Advanced VIRGO
detector\cite{2015CQGra..32b4001A}, with the prospect of a joint run
occuring during the second half of 2016. The GEO detector was the
first of the initial detectors to be fully upgraded becoming GEO-HF
\cite{2006CQGra..23S.207W}, with improved sensitivity at high
frequencies. Japanese efforts have focussed on the development of
KAGRA (formerly LCGT), a cryogenic interferometer located deep
underground in the Kamioka mine\cite{1999IJMPD...8..557K}, although
the project has suffered from a number of set-backs. The construction
of a third LIGO detector interferometer in India using the mothballed
second detector from the Washington site has now moved into its
initial stages, with the prospect of this detector joining the network
around the end of the decade.

\sidebar{
  \includegraphics{figures/aligo-asd.pdf}
  \captionof{figure}{The ASD of LIGO at its design sensitivity. \label{fig:aligo-design-asd}}
  }

The second-generation detectors, specifically the two Advanced LIGO
detectors were responsible for the first discovery of gravitational
waves\cite{2016PhRvL.116m1103A}, and have successfully demonstrated
the ability of interferometry to observe the gravitational
universe. This said, future improvements in sensitivity are highly
desirable, but are likely to be even more technically challenging than
the transition from resonant bars to laser interferometers. In order
to improve the bandwidth of detectors a location free of
\emph{Newtonian noise} must be found, which ultimately mandates the
placement of an interferometer in space. There have been a number of
proposals for a space-based interferometer, with
eLISA\cite{2013GWN.....6....4A} likely to be the first to launch in
the 2020s, following the completion of a pathfinder mission during
2016\cite{2015JPhCS.610a2005A}. NEED TO UPDATE THIS. The eLISA detector will be sensitive
in the milli-hertz region of the gravitational wave spectrum, and will
be capable of observing binary inspirals at a much earlier stage in
their evolution than the advanced ground-based detectors, as well as
the galactic population of low-mass binaries, such as binary white
dwarfs. A Japanese proposal, DECIGO\cite{2011CQGra..28i4011K}, would
observe in the decihertz regime using a complex arrangement of six
spacecraft in a star-of-David confugration. There are also plans for
more sensitive detectors on the ground. The Einstein telescope is a
European proposal for an underground kilometre-scale detector in a
triangular configuration, using a ``xylophone'' configuration to
improve broadband sensitivity compared to the second-generation of
detectors; its scientific aims include providing more sensitive tests
of general relativity than are possible with the advanced
detectors\cite{2012CQGra..29l4013S}. There are also proposals for
upgrades of the advanced detectors to use squeezed light to reduce
quantum noise\cite{2015PhRvD..91f2005M}, the use of
speedmeters\cite{2014MUPB...69..519V,2002gr.qc....11088K}, or atom
interferometry\cite{2013PhRvL.110q1102G,2016PhRvD..93b1101C,2008PhRvD..78l2002D}.


\sidebar{
  \includegraphics{figures/ipta-psd.pdf}
  \captionof{figure}{The effective PSD of the International Pulsar Timing Array (IPTA).}
  }

At the very low-frequency limit of the gravitational wave spectrum the
bulk of detection efforts rotate around pulsar timing arrays, which
promise the detection of gravitational waves by precision measurements
of pulse arrival times from a number of pulsars distributed across the
sky. By observing correlated delays\cite{1983ApJ...265L..39H} in
arrival times the presence of a very long wavelength gravitational
wave can be inferred. There are a number of collaborations actively
producing pulsar observations with the aim of detecting gravitational
waves: the European Pulsar Timing Array
(EPTA)\cite{2013CQGra..30v4009K}, NANOGrav\cite{2009arXiv0909.1058J},
the Parkes Pulsar Timing Array (PPTA)\cite{2013PASA...30...17M}, and
the International Pulsar Timing Array (IPTA)
collaboration\cite{2013CQGra..30v4010M}.

%%% Local Variables: 
%%% mode: latex
%%% TeX-master: "../../document"
%%% End: 
