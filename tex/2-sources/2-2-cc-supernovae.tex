
\subsection{Core-collapse supernovae}
\label{sec:core-coll-supern}


Core collapse supernova (CCSNe) are driven by the release of
gravitational energy as a massive star's core collapses. Progenitor
stars of CCSNe have zero-age-main-sequence (ZAMS) masses in the range
$8\,\msolar \leq M \leq 130\,\msolar$. Much of this energy is stored
as heat in the protoneutron star remnant, around 99\% of the released
energy is carried-off by neutrinos, around 1\% provides the kinetic
energy of the explosion, while less than $0.01\%$ of the energy is
extracted as electromagnetic and gravitational radiation
\cite{2009CQGra..26f3001O}.

When the iron core of a star exceeds the Chandrasekhar mass it becomes
unstable, and undegoes gravitational collapse, and is compressed until
the neutron degeneracy pressure is able to halt arrest the
collapse. At this point the core becomes stiff, and the inner core
rebounds---a phase of the supernova known as ``core bounce''. The
stiff, ultra-dense remnant of the collapse is a proto-neutronstar
(PNS).

Gravitational waves are expected to be emitted in a number of periods
during the collapse, for example during a rotating collapse, and the
core-bounce which follows it; pulsations of the PNS
\cite{1966ApJ...145..514M}; and anisotropic neutrino emission
\cite{1979ApJ...231Q.644E ,1978ApJ...223.1037E ,1978Natur.274..565T}.

In order to predict the gravitational waveforms which would be
produced by a CCSN detailed numerical modelling must be completed,
with the most modern results from
Scheidegger\cite{2010A&A...514A..51S}, modelling rotating,
axisymmetric collapses in three dimensions, and
Dimmelmeier\cite{2008PhRvD..78f4056D} in two dimensions; and
M\"uller\cite{2012A&A...537A..63M} and Ott\cite{2013ApJ...768..115O},
modelling neutrino-driven supernovae in three dimensions.

It is possible that core-collapse supernovae could have been detected
with the initial LIGO detector\cite{2009LRR....12....2S}, although
none were. At design sensitivity the three-detector network of
Advanced LIGO and Advanced VIRGO should be able to detect CCSNe to a
distance of around $\SI{5.5}{\kilo pc}$, in the case of
neutrino-driven explosions, while rapidly-rotating core-collapses will
be detectable to $\SI{50}{\kilo pc}$, the distance to the Large
Magellanic Cloud. Extreme emission scenarios may be detectable as far
as $\SI{0.77}{\mega pc}$, the distance to M31\cite{2016PhRvD..93d2002G}.

\subsection{Type Ia supernovae}
\label{sec:type-ia-supernovae}

Type Ia supernovae (SNe Ia) are believed to be the result of
white-dwarfs in binary systems accreting enough matter to exceed the
Chandrasekhar-mass, and undergoing catastrophic
coore-collapse\cite{2013MNRAS.429.1156S}, however the evolution of the
binary systems which are the progenitors of Type Ia supernovae is
poorly understood. Recent work\cite{2015PhRvD..92l4013S} implies that
the gravitational wave emission from a Type Ia supernova would produce
decihertz gravitational-waves, peaking at a frequency around
\SI{0.4}{\hertz}. This would position SNe Ia as a target for the
proposed DECIGO and BBO space-based observatories.

%%% Local Variables: 
%%% mode: latex
%%% TeX-master: "../../document"
%%% End: 
