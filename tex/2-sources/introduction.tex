The astrophysical sources of gravitational waves can be divided
roughly into three categories\cite{2009LRR....12....2S}:
\begin{description}
\item[Continuous] sources are expected to produce radiation
  constantly, or at least over long periods of time. The primary
  source of continuous sources for advanced LIGO are expected to be
  gravitational wave pulsars, but in detectors which are sensitive at
  lower frequencies, such as the proposed eLISA mission, the radiation
  from inspiralling binary systems should also be detectable.
\item[Transient] sources produce a strong \emph{burst} of
  gravitational waves over a period of seconds or less. These are
  sources which are primarily expected in the advanced LIGO passband,
  with compact binary coalesences and supernovae being major targets
  for burst searches in the advanced observing runs, however there are
  prospects for burst sources in the eLISA regime, for example from
  hyperbolic encounters between compact objects and stars or other
  compact objects
  \cite{2012PhRvD..86l4012B,2012PhRvD..86d4017D,2008MPLA...23...99C,2008APh....30..105C,2010MmSAI..81...87D,2005PhRvD..72h4009G,2010PhRvD..82j7501B,2011ApJ...729L..23G}.
\item[Stochastic] sources are expected to produce a background of
  gravitational waves, from the black holes at the centres of
  galaxies\cite{1980Natur.287..307B,2001astro.ph..8028P,2003ApJ...583..616J,2008MNRAS.390..192S},
  and from the universe's inflationary period\cite{1988PhRvD..37.2078A}.
\end{description}


%%% Local Variables: 
%%% mode: latex
%%% TeX-master: "../../document"
%%% End: 
