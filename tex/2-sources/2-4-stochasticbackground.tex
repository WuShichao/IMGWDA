
While the gravitational wave detections which we expect to make with
the aLIGO and aVIRGO detectors in the next five years all correspond
to radiation from specific sources, in the LISA era we expect to
encounter a new phenomenon: the gravitational wave
background\cite{2009LRR....12....2S}. This is likely to come from a
number of sources. LISA will be sensitive to compact binary systems
for a much longer period of their evolution, and so rather than just
detecting the final few orbits and the coalescense of objects, we will
be able to observe decades or more of their inspiral. Given the large
population of binary objects in the universe, and the ability of LISA
to detect white dwarf binaries, we can expect the existence of a rich
gravitational wave background at low frequencies: so much so that the
data-handling task will be many orders of magnitude more complicated
than that for the advanced era detectors.  Super-massive binary black
holes throughout the universe will also contribute to this background.

Currently-favoured theories of cosmology favour the existence of a
period of cosmological inflation, a process which should have produced
very low-frequency gravitational waves.  These were the focus of the
ultimately-refuted BICEP2 announcement\cite{2014PhRvL.112x1101B} of
the discovery of gravitational waves.

%%% Local Variables: 
%%% mode: latex
%%% TeX-master: "../../document"
%%% End: 
