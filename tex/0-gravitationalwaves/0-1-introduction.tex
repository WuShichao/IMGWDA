\chapterprecis{\Glspl{gw} were perhaps the last of the
  predictions of Einstein's General Theory of Relativity to be
  observed; their detection was the catalyst for the beginning of a
  new era of astrophysics}

\epigraph{I guess we need to do the detection checklist...}{\textbf{Sergey Klimencko}, \emph{Internal LSC communication}, 14 September 2015}

\lettrine[lines=3]{T}{he 14 September 2016} will likely be remembered as one of the most
significant in the history of astronomy and of astrophysics. Early in
the morning of this autumn day, shortly after 9am UT, a gravitational
wave passed through the Earth, and on its way produced a sufficiently
large movement in the mirrors and test masses of the detectors of the
\gls{ligo}, as to be
detected.

Over five months of data analysis, detector characterisation, and
detection verification were conducted by a global team of scientists,
in the \gls{ligo} and \gls{virgo} Scientific Collaboration (LVC). This process
resulted in a slew of journal papers being written, vast quantities of
data produced, and an enormous public outreach effort to be
launched. Eventually, the collaboration found itself in a position to
make the announcement of the first direct detection of gravitational
waves, \gls{gw150914}, on 11 February 2016.

\gls{gw150914} is not the first evidence for gravitational waves, with the
Hulse-Taylor pulsar\cite{1975ApJ...195L..51H,2005ASPC..328...25W}
having provided compelling indirect evidence which lead to its
discoverers receiving the 1993 Nobel Prize.
